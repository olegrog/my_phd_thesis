%%% Обоснование выбора названия диссертации
Молекулярная газодинамика "--- это газовая динамика, построенная на основе кинетической теории газа.
Под последней обычно понимают теорию неравновесных свойств газа.
Ключевую роль при описании газа играет отношение длины свободного пробега молекул газа \(\ell\)
к характерному размеру течения \(L\) "--- так называемое число Кнудсена \(\Kn=\ell/L\).
В континуальном пределе (\(\Kn\to0\)) обычно используют законы классической гидродинамики,
основанной на модели сплошной среды, и только в случае конечных \(\Kn\)
учитывают молекулярную структуру газа. Таким образом, в литературе можно встретить
разделение на континуальную гидрогазодинамику и динамику разреженного газа.
Однако имеется достаточно широкий круг задач, для которых уравнения Навье"--~Стокса
некорректно описывают поведение газа даже при \(\Kn\to0\).
Поэтому в настоящем исследовании используется термин \emph{молекулярная газовая динамика},
подчёркивая тот факт, что методы и представления кинетической теории газа используются
как для разреженного газа, так и для его континуального предела.
Этот термин, по-видимому, впервые предложен в 1970 году М.\,Н.~Коганом~\autocite{Kogan1971review},
позже подхвачен в фундаментальных трудах Г.~Бёрдом~\autocite{Bird1981} и Ё.~Соне~\autocite{Sone2007}.

{\actuality}
%%% Современные проблемы и вызовы
Молекулярная газодинамика, как наука, получила своё бурное развитие с середины XX века
в связи в активным освоением космоса. Первые исследования носили в основном экспериментальный
характер. В XXI веке превалирующую роль играет компьютерное моделирование, что
говорит о зрелости теоретических представлений дисциплины.
Неравновесное состояние газа описывается в общем случае шестимерной функцией распределения,
её эволюция подчиняется уравнению Больцмана.
Входящий в него нелинейный интеграл столкновений представляет собой нелокальный квадратичный оператор,
что создаёт существенные трудности, как для математического, так и численного анализа.
За последние три десятилетия строгая математическая теория пополнилась множеством фундаментальных результатов,
а стремительный рост суперкомпьютерных мощностей, доступных исследователям и инженерам,
спровоцировал системное развитие численных методов.

%%% Прикладные области
На сегодняшний день можно выделить несколько прикладных областей, где активно применяется молекулярная газодинамика:
\begin{enumerate}
    \item \emph{Аэрокосмические исследования}.
    Движение аппаратов в верхних слоях атмосферы сопровождаются сильно неравновесными течениями
    и достаточно большими числами Кнудсена.
    \item \emph{Микроэлектромеханические системы} (МЭМС).
    Эта относительно молодая отрасль обуславливает основную волну интереса
    к изучению разреженного газа в начале XXI века и включает широкий круг задач:
    от протекания газа через неравномерно нагретые микроканалы до процессов испарения и конденсации.
    \item \emph{Экологические проблемы}.
    Конечная фаза существования атмосферных загрязнений "--- это аэрозольные частицы.
    Процесс их образования, а также изменение их дисперсного состава описываются уравнением Больцмана.
    \item \emph{Вакуумные технологии}.
    Моделирование течений газа, когда число Кнудсена значительно меняется в пространственно-временных масштабах,
    представляет собой особенно трудную задачу, однако современный уровень развития вычислительных средств
    позволяет во многих случаях обходиться без дорогостоящих экспериментальных прототипов.
\end{enumerate}

Таким образом, актуальность данного исследования обусловлена
\begin{itemize}
    \item активным развитием прикладных областей;
    \item потребностью в высокоточных численных методах;
    \item быстрым ростом доступных вычислительных ресурсов.
\end{itemize}

{\object} "--- движение одноатомного газа твёрдых сфер различной степени разреженности.
В исследовании одновременно изучается два {\subject}:
\begin{enumerate}
    \item Физические свойства неравновесных течений, включая поля макроскопических величин,
    функции распределения молекулярных скоростей, а также граничные профили сил и теплопотока.
    \item Методы численного и асимптотического анализа.
\end{enumerate}

{\progress}

%%% формальная асимптотическая теория
Формальная асимптотическая теория уравнения Больцмана была заложена с трудах Д.~Гильберта~\autocite{Hilbert1912},
С.~Чепмена~\autocite{Chapman1916}, Д.~Энскога~\autocite{Enskog1917},
позже развита Д.~Барнеттом~\autocite{Burnett1935}, Х.~Грэдом~\autocite{Grad1963a} и Ё.~Соне~\autocite{Sone2002}.
Решение уравнения Больцмана для слаборазреженного газа допускает отделение гидродинамической части
от существенно неравновесных пространственно"=временн\'{ы}х кинетических слоёв.
Большой цикл работ Киотской группы (Ё.~Соне, К.~Аоки, Ш.~Таката, Т.~Овада и др.)
посвящён высокоточному численному анализу кнудсеновского слоя первого~\autocite{Ohwada1989creep, Ohwada1989jump}
и второго порядка~\autocite{Ohwada1992, Takata2015second, Takata2015curvature}
для диффузного отражения и газа твёрдых сфер.
Различные системы гидродинамических уравнений могут быть получены в зависимости от способа асимптотического масштабирования.
В частности, для медленных неизотермических течений справедливы
\emph{уравнения Когана"--~Галкина"--~Фридлендера}~\autocite{Kogan1976} (КГФ),
содержащие некоторые ненавье"--~стоксовские члены.

%%% численные методы
Огромное множество исследований посвящено численному решению уравнения Больцмана.
Среди них можно выделить три магистральных направления в зависимости от способа аппроксимации
функции распределения скоростей:
\begin{itemize}
    \item \emph{методы прямого статистического моделирования} (ПСМ)
    строятся на основе некоторого случайного процесса марковского типа,
    способного аппроксимировать больцмановскую динамику;
    \item \emph{методы дискретных скоростей} подразумевают фиксированный конечный набор доступных молекулярных скоростей;
    \item \emph{проекционные методы} используют разложение по базису в определённом функциональном пространстве.
\end{itemize}
Методы ПСМ в силу своей универсальности и простоты нашли широкое применение в прикладных областях,
однако присущие им флуктуации сильно ограничивают точность получаемых результатов.
Проекционные методы, напротив, обладают наилучшим соотношением погрешности к размерности аппроксимационного пространства,
но, как правило, в достаточно узком классе решений.
Оказалось возможным добиться второго порядка точности в рамках метода дискретных скоростей,
однако для этого потребовался длинный исторический путь.

%%% методы дискретных скоростей
Метод дискретных скоростей был впервые использован Дж.~Хэвилендом~\autocite{Haviland1965},
а также А.~Нордсиком и Б.~Хиксом~\autocite{Nordsieck1966}.
Для вычисления интеграла столкновения они использовали кубатуры Монте"=Карло
с последующей консервативной коррекцией функции распределения.
В дальнейшем метод дискретных скоростей развивался С.~Йеном~\autocite{Yen1984},
В.\,В.~Аристовым и Ф.\,Г.~Черемисиным~\autocite{Tcheremissine1980}.
Модели газа, допускающие столкновения только в дискретном пространстве, берут начало с работ
Т.~Карлемана~\autocite{Carleman1957} и Дж.~Бродуэлла~\autocite{Broadwell1964shock, Broadwell1964shear}.
Методы построения таких моделей изучались С.\,К.~Годуновым и У.\,М.~Султангазиным~\autocite{Sultangazin1971},
Р.~Гатиньоль~\autocite{Gatignol1975}, А.~Кабанном~\autocite{Cabannes1980},
А.\,В.~Бобылевым и К.~Черчиньяни~\autocite{Bobylev1999dvm}.
Сходимость дискретного газа к решению Ди-Перна"--~Лионса доказана С.~Мишлером~\autocite{Mischler1997}.

%%% современные методы дискретных скоростей
А.~Пальцевский, Ж.~Шнайдер и А.\,В.~Бобылев показали, что классические модели дискретного газа,
несмотря на присущие им \emph{консервативность} на микроскопическом уровне и \emph{энтропийность}
(выполнение \(\mathcal{H}\)-теоремы), обладают дробным порядком сходимости к уравнению Больцмана~\autocite{Palczewski1997}.
В.~Панфёров и А.~Гейнц показали, как специальная замена переменных
позволяет улучшить сходимость, но лишь вплоть до первого порядка~\autocite{Panferov2002}.
\emph{Размазывание (mollification) столкновительной сферы} позволяет естественным образом
решить проблему консервативной аппроксимации, избегая решения целочисленных уравнений.
К.~Бюе, С.~Кордье и П.~Дегон продемонстрировали, как с её помощью обеспечить консервативность
на макроскопическом уровне (для столкновительного оператора целиком)~\autocite{Buet1998};
Х.~Бабовски построил простейшую схему с консервативностью на мезоскопическом уровне
(для всей столкновительной сферы)~\autocite{Babovsky1998}, его подход позже развил Д.~Гёрш~\autocite{Goersch2002};
наконец, Ф.\,Г.~Черемисин предложил новый класс методов, сохраняющих консервативность
на микроскопическом уровне (для отдельной столкновительной пары)~\autocite{Tcheremissine1997}.
Микроскопическая консервативность, достигнутая Ф.\,Г.~Черемисиным, может быть интерпретирована
как проекционная процедура Петрова"--~Галёркина,
в которой столкновительные инварианты образуют ортогональную оболочку.
Поэтому такой метод будем называть \emph{проекционным методом дискретных скоростей} (ПМДС).

%%% неравномерные сетки в пространстве скоростей
Во многих прикладных задачах эффективная аппроксимация уравнения Больцмана
требует существенно неоднородной дискретизации в скоростном пространстве.
Неравномерные сетки активно используются
как в методах дискретных скоростей~\autocite{Kolobov2011, Morris2012},
так и проекционных~\autocite{Heintz2008, Wu2014}.

В настоящем исследовании выделены две основные {\aim}:
\begin{enumerate}
    \item Развитие ПМДС для неравномерных сеток, его верификация в широком диапазоне неравновесности.
    \item Численный анализ некоторых классических течений разреженного газа на основе
    как уравнения Больцмана, так и соответствующих уравнений гидродинамического типа.
    Оценка области применимости последних при различных граничных условиях.
\end{enumerate}
Для достижения поставленных целей поставлены следующие {\tasks}:
\begin{enumerate}
    \item Оценка точности ПМДС на различных численных примерах и выделение круга задач,
    где использование неравномерных сеток необходимо и оправдано.\label{tasks:first1}
    \item Изучение и сравнительный анализ многоточечных проекционных шаблонов,
    необходимых для консервативного вычисления интеграла столкновений на неравномерных сетках.
    \item Построение асимптотического решения второго порядка
    для пограничного слоя Прандтля для газа твёрдых сфер. \label{tasks:first2}
    \item Сравнительный анализ численных решений задачи Куэтта в широком диапазоне параметров,
    получаемых с помощью ПМДС и других общепризнанных методов.
    \item Исследование сходимости численного решения уравнения Больцмана к асимптотическому
    для широкого класса течений между параллельными пластинами.
    \item Исследование различных подходов к постановке граничных условий для уравнений КГФ,
    сравнительный анализ с решением уравнения Больцмана.\label{tasks:last1}
    \item Разработка комплекса программ для решения уравнений КГФ в произвольной геометрии.
    \item Параметрический анализ течений между некоаксиальными и эллиптическими цилиндрами
    в континуальном пределе.\label{tasks:last2}
\end{enumerate}
Задачи \ref{tasks:first1}--\ref{tasks:last1} позволяют достичь первой цели,
задачи \ref{tasks:first2}--\ref{tasks:last2} раскрывают содержание второй цели.

%%% Новые методики и идеи, но не перечислять результаты! В настоящем времени.
{\novelty}
\begin{enumerate}
    \item ПМДС применяется для неравномерных сеток в пространстве скоростей. % Rogozin2016
    \item Достижения современной нелинейной асимптотической теории используются
    для верификации численного метода решения уравнения Больцмана. % Rogozin2016
    \item Уравнения КГФ решаются с граничными условиями, содержащими члены отличные от теплового скольжения. % Rogozin2017
    \item Изучаются новые эффекты и свойства течений разреженного газа. % Rogozin2014, Rogozin2016, Rogozin2017
\end{enumerate}

{\influence}
\begin{enumerate}
    \item ПМДС на неравномерных сетках может быть использован
    как инструмент для численного анализа течений разреженного газа с повышенной точностью.
    \item Результаты анализа нелинейной задачи Куэтта могут служить эталоном
    для верификации других численных методов.
    \item Разработанный солвер \verb+snitSimpleFoam+ предоставляет широкие возможности
    для моделирования медленных неизотермических течений слаборазреженного газа в произвольной геометрии
    как в академических целях, так и для инженерных приложений.
\end{enumerate}

{\methods}
В качестве математической модели неравновесного газа используется кинетическая теория,
высокий уровень развития которой позволяет настоящему исследованию обходиться без эмпирической базы.
Методологическая база включает специальные математические и вычислительные методы:
\begin{itemize}
    \item асимптотические методы нелинейной теории возмущения;
    \item численные методы интегрирования систем дифференциальных уравнений в частных производных,
    специальные численные методы вычислительной гидродинамики;
    \item численные методы многомерного интегрирования;
    \item квадратурные методы решения интегральных уравнений;
    \item проекционные методы решения операторных уравнений;
    \item вариационное исчисление.
\end{itemize}
Численные результаты получены с использованием широкого спектра современных компьютерных технологий и программных комплексов, включая
\begin{itemize}
    \item системы компьютерной алгебры (SymPy~\autocite{sympy}),
    \item генерацию расчётных сеток (gmsh~\autocite{gmsh}),
    \item организацию параллельных вычислений (MPI~\autocite{mpi}),
    \item инструментарий вычислительной гидродинамики (OpenFOAM~\autocite{openfoam}),
    \item визуализацию полей (matplotlib~\autocite{matplotlib}).
\end{itemize}

%%% Результаты исследования
В соответствии с результатами решения поставленных задач выдвигаются {\defpositions}
\begin{enumerate}
    \item\label{defpos:uniform_grid} % Rogozin2010, Rogozin2011
    На численных примерах показано, что ПМДС на равномерной сетке
    способен адекватно и эффективно описывать течения разреженного газа~\cite{Rogozin2010, Rogozin2011, Rogozin2017}.
    Использование неравномерной сетки оправдано при необходимости детального разрешения резких перепадов функции распределения.
    \item\label{defpos:stencils} % Rogozin2016
    Для многоточечных проекционных шаблонов выявлены критерии,
    минимизирующие требования к мощности множества кубатурных точек~\cite{Rogozin2016}.
    Изучены оптимальные пяти- и семиточечный шаблоны.
    \item\label{defpos:transport_coeffs} % Rogozin2016
    С точностью 8--10 знаков вычислены неизвестные ранее транспортные коэффициенты для газа твёрдых сфер,
    необходимые для вычисления тензора напряжений и вектора потока тепла в пограничном слое Прандтля~\cite{Rogozin2016}.
    \item\label{defpos:Couette_flow} % Rogozin2016
    Получено решение плоской задачи Куэтта для широкого диапазона чисел Кнудсена вплоть до гиперзвуковых скоростей.
    Абсолютная погрешность первых 13-ти моментов функции распределения не выше \(10^{-4}\)~\cite{Rogozin2016}.
    \item\label{defpos:asymptotic_verification} % Rogozin2016+2017
    ПМДС на неравномерных прямоугольных сетках "--- надёжный инструмент для высокоточного анализа
    нелинейных плоских кинетических слоёв. Продемонстрировано отклонение от асимптотического
    решения не более \(10^{-4}\) для нелинейных течений между параллельными пластинами с температурой,
    распределённой а) константно~\cite{Rogozin2016}, б) синусоидально~\cite{Rogozin2017}.
    \item\label{defpos:boundary_conditions} % Rogozin2017
    На численных примерах показано, что использование совместимых граничных условий первого и второго порядка
    для уравнений КГФ существенно улучшает точность асимптотического решения~\cite{Rogozin2017}.
    Исследованы, в том числе, граничные условия, учитывающие кривизну граничной поверхности.
    \item\label{defpos:snit_solver} % Rogozin2014
    В рамках платформы OpenFOAM разработан солвер уравнений КГФ
    на основе метода конечных объёмов и модифицированного алгоритма SIMPLE~\cite{Rogozin2014}.
    \item\label{defpos:snit_forces} % Rogozin2014 + unpublished
    На основе численного параметрического анализа некоторых нелинейных течений газа
    между равномерно нагретыми телами в континуальном пределе была показана
    электростатическая аналогия~\cite{Rogozin2014}:
    обтекаемые тела притягиваются подобно электрически заряженным телам,
    при этом сила \(F\propto\Kn^2(T_2^s-T_1^s)(T_2^{1+s}-T_1^{1+s})\),
    когда вязкость и теплопроводность газа \(\mu, \lambda \propto T^s\).
\end{enumerate}

{\reliability} полученных результатов обеспечивается следующими обстоятельствами:
\begin{enumerate}
    \item Кинетическое уравнение Больцмана выводится из первых принципов и
    содержит минимальное количество дополнительных допущений.
    В настоящем исследовании в качестве потенциала межмолекулярного взаимодействия
    повсеместно используется модель твёрдых сфер,
    которая достаточно адекватно отражает реальные кинетические процессы в широком диапазоне неравновесности.
    В качестве модели взаимодействия газа с поверхностью используется полное диффузное отражение.
    \item Проводится систематический сравнительный анализ результатов,
    полученных с помощью ПМДС, прямого статистического моделирования
    и асимптотическиго анализа уравнения Больцмана.
    \item Проводится анализ сходимости численных методов на основе
    множества решений на разностных сетках различной мелкости.
    \item Верификация используемых солверов и систем обработки данных
    выполнена на тестовых задачах, решение которых с высокой точностью представлено в литературе.
    Результаты находятся в полном соответствии с результатами, полученными другими авторами.
\end{enumerate}

{\probation} Результаты диссертации докладывались лично соискателем на
\begin{itemize}
    \item 54 научной конференции МФТИ (Долгопрудный, 2011),
    \item 9 Международной конференции по неравновесным процессам в соплах и струях (Алушта, 2012),
    \item семинаре сектора кинетической теории отдела механики ВЦ ФИЦ ИУ РАН (Москва, 2016),
    \item 2 Международном симпозиуме по аэродинамике, охватывающем различные режимы течений (Маньян, Китай, 2017).
\end{itemize}

{\contribution} соискателя в работах с соавторами заключается в следующем:
разработка алгоритмов, программная реализация, проведение вычислительных экспериментов, анализ результатов.

\ifnumequal{\value{bibliosel}}{0}{% Встроенная реализация с загрузкой файла через движок bibtex8
    \publications\ Основные результаты по теме диссертации изложены в XX печатных изданиях,
    X из которых изданы в журналах, рекомендованных ВАК,
    X "--- в тезисах докладов.%
}{% Реализация пакетом biblatex через движок biber
%Сделана отдельная секция, чтобы не отображались в списке цитированных материалов
    \begin{refsection}[vak,papers,conf]% Подсчет и нумерация авторских работ. Засчитываются только те, которые были прописаны внутри \nocite{}.
        %Чтобы сменить порядок разделов в сгрупированном списке литературы необходимо перетасовать следующие три строчки, а также команды в разделе \newcommand*{\insertbiblioauthorgrouped} в файле biblio/biblatex.tex
        \printbibliography[heading=countauthorvak, env=countauthorvak, keyword=biblioauthorvak, section=1]%
        \printbibliography[heading=countauthorconf, env=countauthorconf, keyword=biblioauthorconf, section=1]%
        \printbibliography[heading=countauthornotvak, env=countauthornotvak, keyword=biblioauthornotvak, section=1]%
        \printbibliography[heading=countauthor, env=countauthor, keyword=biblioauthor, section=1]%
        \nocite{Rogozin2010, Rogozin2011, Rogozin2014, Rogozin2016, Rogozin2017}
        \publications\ Основные результаты по теме диссертации изложены в \arabic{citeauthorvak} печатных изданиях,
        рекомендованных ВАК.
    \end{refsection}
    \begin{refsection}[vak,papers,conf]%Блок, позволяющий отобрать из всех работ автора наиболее значимые, и только их вывести в автореферате, но считать в блоке выше общее число работ
        \printbibliography[heading=countauthorvak, env=countauthorvak, keyword=biblioauthorvak, section=2]%
        \printbibliography[heading=countauthornotvak, env=countauthornotvak, keyword=biblioauthornotvak, section=2]%
        \printbibliography[heading=countauthorconf, env=countauthorconf, keyword=biblioauthorconf, section=2]%
        \printbibliography[heading=countauthor, env=countauthor, keyword=biblioauthor, section=2]%
    \end{refsection}
}
