%% Согласно ГОСТ Р 7.0.11-2011:
%% 5.3.3 В заключении диссертации излагают итоги выполненного исследования, рекомендации, перспективы дальнейшей разработки темы.
%% 9.2.3 В заключении автореферата диссертации излагают итоги данного исследования, рекомендации и перспективы дальнейшей разработки темы.
\begin{enumerate}

\item Все полученные результаты могут быть обобщены для произвольного потенциала
практически без увеличения вычислительных затрат.

\item в расширении области применимости асимтотического решения за счёт
особой методики постановки граничных условий;

\item в комплексной верификации проекционного метода дискретных скоростей
для широкого круга практических задач;

\item С точки зрения точности результатов и вычислительных затрат асимптотическое решение
является оптимальным выбором для моделирования течений в области \emph{малых чисел Кнудсена}.
Однако такое решение доступно только для узкого класса краевых задач с гладкой границей,
поэтому в большей степени оно должно служить для верификации численных методов общего назначения.
Проекционный метод на равномерной скоростной сетке "--- один из таких.
Полученные с его помощью результаты демонстрируют хорошую сходимость к асимптотическому решению,
однако требуют большого числа итераций для достижения стационарного состояния.
\item Для \emph{чисел Кнудсена близких к единице} проекционный метод позволяет получить
более точное решение по сравнению с методами прямого статистического моделирования,
для которых характерны значительные флуктуации функции распределения.
Особенно это касается малых чисел Маха, где статистические флуктуации не позволяют
получить адекватную картину течений.
\item Эффективность проекционного метода, как и любого метода дискретных скоростей,
на равномерной сетке в скоростном пространстве снижается для \emph{больших чисел Кнудсена}
ввиду значительных градиентов функции распределения во всей области, занимаемой газом.
Использование в этом случае неравномерной скоростной сетки даёт возможность
аппроксимировать функцию распределения с высокой точностью. Однако поскольку методы
дискретных скоростей подразумевают постоянство скоростной сетки во всём физическом
пространстве, то такой подход не может быть применён непосредственно к задачам
с произвольной геометрией и требует дальнейшего существенного развития.
\item Для \emph{медленных неизотермических течений} численное решение уравнения Больцмана
сходится к решению уравнений КГФ, получаемых в ходе асимптотического анализа.
При этом смешивание граничных условий разных порядков позволяет получить
температурное поле не только в континуальном пределе, но и для малых чисел Кнудсена.
В последнем случае оно является приближённым, но не асимптотическим решением
следующего порядка.

\item Проекционный метод, как и любой метод дискретных скоростей, сталкивается с проблемой
эффективной дискретизации. Области пространства скоростей,
где наблюдаются значительные перепады функции распределения, представляют известные трудности
для достижения высокой точности аппроксимации.

\item По сравнению с другими методами, проекционный метод на неравномерных сетках
позволяет достичь повышенной точности численного анализа плоского приграничного слоя
от линейных и вплоть до гиперзвуковых течений для широкого диапазона чисел Кнудсена.

\item Использование всех совместимых с уравнениями КГФ граничных условий первого и второго порядка
позволяет существенно улучшить точность асимптотического решения

\end{enumerate}
