%% Согласно ГОСТ Р 7.0.11-2011:
%% 5.3.3 В заключении диссертации излагают итоги выполненного исследования, рекомендации, перспективы дальнейшей разработки темы.
%% 9.2.3 В заключении автореферата диссертации излагают итоги данного исследования, рекомендации и перспективы дальнейшей разработки темы.
\begin{enumerate}[wide]

%%% полученные решения классических задач
\item Рассмотрены классические задачи молекулярной газовой динамики,
такие как течения между параллельными пластинами, некоаксиальными цилиндрами и сферами, эллиптическими цилиндрами.
Полученные решения обладают высокой точностью, верифицированы и могут считаться эталонными.
Их детальный анализ обнаружил ряд новых физических эффектов.
Некоторые из них практически недоступны для ПСМ,
что служит веским основанием к дальнейшему развитию численных методов решения уравнения Больцмана.

%%% плюсы и минусы KПИМДС для неравномерных сеток
\item Обобщение KПИМДС для неравномерных сеток приводит к дополнительным вычислительным трудностям.
В частности, усложняется алгоритм консервативного проецирования в интеграле столкновений,
повышаются требования к мощности множества кубатурных точек,
что в целом приводит к увеличению вычислительных затрат.
Кроме того, на неравномерной сетке, в общем случае, снижается точность кубатур функций близких к максвелловским.
Тем не менее в настоящем исследовании на численных примерах продемонстрировано,
как в рамках KПИМДС неравномерная прямоугольная сетка позволяет достичь высокой точности и эффективности
а) для детального разрешения плоских кинетических слоёв,
б) для медленных, но сильно неизотермических течений.
Настоящая область применения метода значительно шире, включая
гиперзвуковые течения и задачи при очень больших числах Кнудсена.
Неравномерные сетки позволяют эффективно аппроксимировать
как большой объём скоростного пространства в первом случае,
так и высокие градиенты функции распределения во втором.

%%% сложности и направления развития KПИМДС для неравномерных сеток
\item Важной задачей математического анализа KПИМДС остаётся вопрос сходимости
и особенно влияния проекционного шаблона на её скорость.
Неравномерные сетки неизбежно приводят к отрицательным проекционным весам,
которые могут стать причиной аномальных численных флуктуаций решения.
Эта проблема требует детального анализа.

%%% польза асимптотического анализа
\item Асимптотическая теория уравнения Больцмана для малых чисел Кнудсена
играет важнейшую роль в моделировании разреженного газа.
С её помощью можно получить не только значения транспортных коэффициентов из знания молекулярного потенциала,
но также истинные граничные условия для гидродинамических уравнений и, что немаловажно,
корректно описать существенно неравновесное поведение газа в слое Кнудсена.
На численных примерах было показано, как использование граничных условий первого и второго порядка
позволяет улучшить точность и качество асимптотического решения.
В настоящем исследовании применение асимптотической теории оказалось ещё шире.
Главным образом, она послужила надёжным инструментом верификации численного метода решения уравнения Больцмана.
Кроме того, использование асимптотического решения в качестве начального приближения позволило
значительно ускорить решение стационарных задач с малыми числами Кнудсена.

%%% ограниченность асимптотического анализа
\item Гидродинамическое описание газа может оказаться некорректным на масштабах существенно больше длины свободного пробега,
если градиенты макроскопических величин в некоторых областях сравнимы с обратным числом Кнудсена.
Достоверно описать поведение газа в этих существенно неравновесных областях
возможно только в рамках кинетического подхода.
Подобная ситуация встречается во многих реальных задачах.
В настоящем исследовании было продемонстрировано кардинальное изменение картины
медленного неизотермического течения при больших градиентах температуры.

%%% прикладное значение медленных неизотермических течений
\item Медленные неизотермические течения представляют интерес в набирающей обороты индустрии МЭМС.
В настоящем исследовании показано, что численное решение уравнения Больцмана в континуальном пределе
сходится к решению уравнений КГФ с соответствующими граничными условиями,
которые, таким образом, верно учитывают влияние сильных температурных
неоднородностей на процессы переноса в слаборазреженном газе.

\end{enumerate}
