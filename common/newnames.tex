% Новые переменные, которые могут использоваться во всём проекте
% ГОСТ 7.0.11-2011
% 9.2 Оформление текста автореферата диссертации
% 9.2.1 Общая характеристика работы включает в себя следующие основные структурные
% элементы:
% актуальность темы исследования;
\newcommand{\actualityTXT}{Актуальность темы.}
% степень ее разработанности;
\newcommand{\progressTXT}{Степень разработанности темы.}
% цели и задачи;
\newcommand{\aimTXT}{цели}
\newcommand{\tasksTXT}{задачи}
% научную новизну;
\newcommand{\noveltyTXT}{Научная новизна:}
% теоретическую и практическую значимость работы;
\newcommand{\influenceTXT}{Теоретическая и практическая значимость:}
% или чаще используют просто
%\newcommand{\influenceTXT}{Практическая значимость}
% методологию и методы исследования;
\newcommand{\methodsTXT}{Методология и методы исследования.}
% положения, выносимые на защиту;
\newcommand{\defpositionsTXT}{основные положения, выносимые на~защиту:}
% степень достоверности и апробацию результатов.
\newcommand{\reliabilityTXT}{Достоверность}
\newcommand{\probationTXT}{Апробация работы.}

\newcommand{\contributionTXT}{Личный вклад}
\newcommand{\publicationsTXT}{Публикации.}

\newcommand{\objectTXT}{Объект исследования}
\newcommand{\subjectTXT}{предмета}

\newcommand{\authorbibtitle}{Публикации автора по теме диссертации}
\newcommand{\vakbibtitle}{В изданиях из списка ВАК РФ}
\newcommand{\notvakbibtitle}{В прочих изданиях}
\newcommand{\confbibtitle}{В сборниках трудов конференций}
\newcommand{\fullbibtitle}{Список литературы} % (ГОСТ Р 7.0.11-2011, 4)

% математические сокращения
\DeclareMathOperator{\spann}{span}
\newcommand{\Kn}{\mathrm{Kn}}
\newcommand{\Ma}{\mathrm{Ma}}
\newcommand{\NS}{N\!S}
\newcommand{\dd}{d} %{\mathrm{d}}
\newcommand{\der}[2][]{\frac{\dd#1}{\dd#2}}
\newcommand{\derdual}[2][]{\frac{\dd^2#1}{\dd#2^2}}
\newcommand{\Der}[2][]{\dd#1/\dd#2}
\newcommand{\pder}[2][]{\frac{\partial#1}{\partial#2}}
\newcommand{\pderdual}[2][]{\frac{\partial^2#1}{\partial#2^2}}
\newcommand{\pderder}[3][]{\frac{\partial^2#1}{\partial#2\partial#3}}
\newcommand{\Pder}[2][]{\partial#1/\partial#2}
\newcommand{\Pderder}[3][]{\partial^2#1/\partial#2\partial#3}
\newcommand{\varder}[2][]{\frac{\delta#1}{\delta#2}}
\newcommand{\dzeta}{\boldsymbol{\dd\zeta}}
\newcommand{\bzeta}{{\boldsymbol{\zeta}}}
\newcommand{\dx}{\boldsymbol{\dd{x}}}
\newcommand{\bx}{{\boldsymbol{x}}}
\newcommand{\by}{{\boldsymbol{y}}}
\newcommand{\bv}{{\boldsymbol{v}}}
\newcommand{\bh}{{\boldsymbol{h}}}
\newcommand{\be}{\boldsymbol{e}}
\newcommand{\Nu}{\mathcal{N}}
\newcommand{\Mu}{\mathcal{M}}
\newcommand{\OO}[1]{O(#1)}
\newcommand{\oo}[1]{o(#1)}
\newcommand{\onwall}[1]{\left(#1\right)_0}
\newcommand{\deltann}[2]{(\delta_{#1#2}-n_#1 n_#2)}
\newcommand{\Set}[2]{\{\,{#1}:{#2}\,\}}
\newcommand{\inner}[2]{\left\langle{#1},{#2}\right\rangle}
\newcommand{\eqdef}{\mathrel{\overset{\makebox[0pt]{\mbox{\normalfont\tiny\sffamily def}}}{=}}}
% {\overset{\mathrm{def}}{=\joinrel=}}

% use Russian tradition
\renewcommand{\phi}{\varphi}
\renewcommand{\epsilon}{\varepsilon}
\renewcommand{\kappa}{\varkappa}

\newcommand{\muHS}{\hat{\mu}_{\mathrm{HS}}}
\newcommand{\QHS}{\hat{Q}_{\mathrm{HS}}}

% математические сокращения для приложения
\newcommand{\bxi}{{\boldsymbol{\xi}}}
\newcommand{\dxi}{\boldsymbol{\dd\xi}}
\newcommand{\B}{\ensuremath{\mathcal{B}^{(4)}}}
\newcommand{\Q}{\ensuremath{\mathcal{Q}^{(0)}}}
\newcommand{\T}[1]{\ensuremath{\mathcal{T}^{(#1)}}}
\newcommand{\TT}{\ensuremath{\tilde{\mathcal{T}}^{(0)}}}
\newcommand{\QQ}{\ensuremath{\tilde{\mathcal{Q}}^{(0)}}}
\newcommand{\IF}[2][0]{\ensuremath{I{#2}^{(#1)}}}
\newcommand{\IFF}[1]{\ensuremath{I\tilde{#1}^{(0)}}}
\newcommand{\ZD}[2]{\zeta_{#1}\delta_{#2}}
\newcommand{\ZZD}[3]{\zeta_{#1}\zeta_{#2}\delta_{#3}}
\newcommand{\Z}{\zeta_i}
\newcommand{\ZZ}{\zeta_i\zeta_j}
\newcommand{\ZZZ}{\zeta_i\zeta_j\zeta_k}
\newcommand{\ZZZZ}{\zeta_i\zeta_j\zeta_k\zeta_l}
\newcommand{\DD}[2]{\delta_{#1}\delta_{#2}}
\DeclareMathOperator{\sgn}{sgn}
\DeclareMathOperator{\erf}{erf}

% gnuplot 5+ default colors
\definecolor{gnuplot1}{HTML}{9400d3}
\definecolor{gnuplot2}{HTML}{009e73}
\definecolor{gnuplot3}{HTML}{56b4e9}
\definecolor{gnuplot4}{HTML}{e69f00}
\definecolor{gnuplot5}{HTML}{f0e442}
\definecolor{gnuplot6}{HTML}{0072b2}
\definecolor{gnuplot7}{HTML}{e51e10}
% gnuplot markers
\pgfdeclareplotmark{star9}{\pgfuseplotmark{+}\pgftransformscale{1.5}\pgfuseplotmark{x}}
% tikz parameters
\newcommand{\mytikzscale}{1.5}
