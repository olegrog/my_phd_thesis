%%% Основные сведения %%%
\newcommand{\thesisAuthorLastName}{Рогозин}
\newcommand{\thesisAuthorOtherNames}{Олег Анатольевич}
\newcommand{\thesisAuthorInitials}{О.\,А.}
\newcommand{\thesisAuthor}             % Диссертация, ФИО автора
{%
    \texorpdfstring{% \texorpdfstring takes two arguments and uses the first for (La)TeX and the second for pdf
        \thesisAuthorLastName~\thesisAuthorOtherNames% так будет отображаться на титульном листе или в тексте, где будет использоваться переменная
    }{%
        \thesisAuthorLastName, \thesisAuthorOtherNames% эта запись для свойств pdf-файла. В таком виде, если pdf будет обработан программами для сбора библиографических сведений, будет правильно представлена фамилия.
    }
}
\newcommand{\thesisAuthorShort}        % Диссертация, ФИО автора инициалами
{\thesisAuthorInitials~\thesisAuthorLastName}
%\newcommand{\thesisUdk}                % Диссертация, УДК
%{\todo{533.72}}
\newcommand{\thesisTitle}              % Диссертация, название
{Численный и асимптотический анализ некоторых классических задач молекулярной газодинамики}
\newcommand{\thesisSpecialtyNumber}    % Диссертация, специальность, номер
{01.02.05}
\newcommand{\thesisSpecialtyTitle}     % Диссертация, специальность, название
{Механика жидкости, газа и плазмы}
\newcommand{\thesisDegree}             % Диссертация, ученая степень
{кандидата физико-математических наук}
\newcommand{\thesisDegreeShort}        % Диссертация, ученая степень, краткая запись
{канд. физ.-мат. наук}
\newcommand{\thesisCity}               % Диссертация, город написания диссертации
{Москва}
\newcommand{\thesisYear}               % Диссертация, год написания диссертации
{2017}
\newcommand{\thesisOrganization}       % Диссертация, организация
{Федеральный исследовательский центр <<Информатика и управление>> Российской академии наук}
\newcommand{\thesisOrganizationShort}  % Диссертация, краткое название организации для доклада
{ФИЦ ИУ РАН}

\newcommand{\thesisInOrganization}     % Диссертация, организация в предложном падеже: Работа выполнена в ...
{Федеральном исследовательском центре <<Информатика и управление>> Российской академии наук}

\newcommand{\supervisorFio}            % Научный руководитель, ФИО
{Черемисин Феликс Григорьевич}
\newcommand{\supervisorRegalia}        % Научный руководитель, регалии
{доктор физико-математических наук, профессор}
\newcommand{\supervisorFioShort}       % Научный руководитель, ФИО
{Ф.\,Г.~Черемисин}
\newcommand{\supervisorRegaliaShort}   % Научный руководитель, регалии
{д.ф.-м.н., проф.}


\newcommand{\opponentOneFio}           % Оппонент 1, ФИО
{\todo{Фамилия Имя Отчество}}
\newcommand{\opponentOneRegalia}       % Оппонент 1, регалии
{\todo{доктор физико-математических наук, профессор}}
\newcommand{\opponentOneJobPlace}      % Оппонент 1, место работы
{\todo{Не очень длинное название для места работы}}
\newcommand{\opponentOneJobPost}       % Оппонент 1, должность
{\todo{старший научный сотрудник}}

\newcommand{\opponentTwoFio}           % Оппонент 2, ФИО
{\todo{Фамилия Имя Отчество}}
\newcommand{\opponentTwoRegalia}       % Оппонент 2, регалии
{\todo{кандидат физико-математических наук}}
\newcommand{\opponentTwoJobPlace}      % Оппонент 2, место работы
{\todo{Основное место работы c длинным длинным длинным длинным названием}}
\newcommand{\opponentTwoJobPost}       % Оппонент 2, должность
{\todo{старший научный сотрудник}}

\newcommand{\leadingOrganizationTitle} % Ведущая организация, дополнительные строки
{\todo{Федеральное государственное бюджетное образовательное учреждение высшего профессионального образования с~длинным длинным длинным длинным названием}}

\newcommand{\defenseDate}              % Защита, дата
{\todo{DD mmmmmmmm YYYY~г.~в~XX часов}}
\newcommand{\defenseCouncilNumber}     % Защита, номер диссертационного совета
{Д\,002.073.03}
\newcommand{\defenseCouncilTitle}      % Защита, учреждение диссертационного совета
{Федеральном исследовательском центре <<Информатика и управление>> Российской академии наук}
\newcommand{\defenseCouncilAddress}    % Защита, адрес учреждение диссертационного совета
{119333, Москва, ул.~Вавилова, д.~44, кор.~2}
\newcommand{\defenseCouncilPhone}      % Телефон для справок
{+7~(499)~135-51-64}

\newcommand{\defenseSecretaryFio}      % Секретарь диссертационного совета, ФИО
{Безродных Сергей Игоревич}
\newcommand{\defenseSecretaryRegalia}  % Секретарь диссертационного совета, регалии
{доктор физико-математических наук}    % Для сокращений есть ГОСТы, например: ГОСТ Р 7.0.12-2011 + http://base.garant.ru/179724/#block_30000

\newcommand{\synopsisLibrary}          % Автореферат, название библиотеки
{ФИЦ ИУ РАН или на сайте института \url{www.frccsc.ru}}
\newcommand{\synopsisDate}             % Автореферат, дата рассылки
{\todo{DD mmmmmmmm YYYY года}}

% To avoid conflict with beamer class use \providecommand
\providecommand{\keywords}%            % Ключевые слова для метаданных PDF диссертации и автореферата
{}
