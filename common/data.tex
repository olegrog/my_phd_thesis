%%% Основные сведения %%%
\newcommand{\thesisAuthorLastName}{Рогозин}
\newcommand{\thesisAuthorOtherNames}{Олег Анатольевич}
\newcommand{\thesisAuthorInitials}{О.\,А.}
\newcommand{\thesisAuthor}             % Диссертация, ФИО автора
{%
    \texorpdfstring{% \texorpdfstring takes two arguments and uses the first for (La)TeX and the second for pdf
        \thesisAuthorLastName~\thesisAuthorOtherNames% так будет отображаться на титульном листе или в тексте, где будет использоваться переменная
    }{%
        \thesisAuthorLastName, \thesisAuthorOtherNames% эта запись для свойств pdf-файла. В таком виде, если pdf будет обработан программами для сбора библиографических сведений, будет правильно представлена фамилия.
    }
}
\newcommand{\thesisAuthorShort}        % Диссертация, ФИО автора инициалами
{\thesisAuthorInitials~\thesisAuthorLastName}
%\newcommand{\thesisUdk}                % Диссертация, УДК
%{\todo{533.72}}
\newcommand{\thesisTitle}              % Диссертация, название
{Численный и асимптотический анализ некоторых классических задач молекулярной газодинамики}
\newcommand{\thesisSpecialtyNumber}    % Диссертация, специальность, номер
{01.02.05}
\newcommand{\thesisSpecialtyTitle}     % Диссертация, специальность, название
{Механика жидкости, газа и плазмы}
\newcommand{\thesisDegree}             % Диссертация, ученая степень
{кандидата физико-математических наук}
\newcommand{\thesisDegreeShort}        % Диссертация, ученая степень, краткая запись
{канд. физ.-мат. наук}
\newcommand{\thesisCity}               % Диссертация, город написания диссертации
{Москва}
\newcommand{\thesisYear}               % Диссертация, год написания диссертации
{2018}
\newcommand{\thesisOrganization}       % Диссертация, организация
{Федеральный исследовательский центр <<Информатика и управление>> Российской академии наук}
\newcommand{\thesisOrganizationShort}  % Диссертация, краткое название организации для доклада
{ФИЦ ИУ РАН}

\newcommand{\thesisInOrganization}     % Диссертация, организация в предложном падеже: Работа выполнена в ...
{ФГУ <<Федеральный исследовательский центр <<Информатика и управление>> Российской академии наук>> (ФИЦ ИУ РАН)}

\newcommand{\supervisorFio}            % Научный руководитель, ФИО
{Черемисин Феликс Григорьевич}
\newcommand{\supervisorRegalia}        % Научный руководитель, регалии
{доктор физико-математических наук, профессор}
\newcommand{\supervisorFioShort}       % Научный руководитель, ФИО
{Ф.\,Г.~Черемисин}
\newcommand{\supervisorRegaliaShort}   % Научный руководитель, регалии
{д.ф.-м.н., проф.}
\newcommand{\supervisorJobPlace}       % Научный руководитель, место работы
{ФГУ <<\thesisOrganization>>}
\newcommand{\supervisorJobPost}        % Научный руководитель, должность
{главный научный сотрудник}


\newcommand{\opponentOneFio}           % Оппонент 1, ФИО
{Кузнецов Михаил Михайлович}
\newcommand{\opponentOneRegalia}       % Оппонент 1, регалии
{доктор физико-математических наук, доцент}
\newcommand{\opponentOneJobPlace}      % Оппонент 1, место работы
{ГОУ ВО МО <<Московский государственный областной университет>>}
\newcommand{\opponentOneJobPost}       % Оппонент 1, должность
{профессор кафедры теоретической физики}

\newcommand{\opponentTwoFio}           % Оппонент 2, ФИО
{Горелов Сергей Львович}
\newcommand{\opponentTwoRegalia}       % Оппонент 2, регалии
{доктор физико-математических наук, доцент}
\newcommand{\opponentTwoJobPlace}      % Оппонент 2, место работы
{ФГУП <<Центральный аэрогидродинамический институт имени профессора Н.\,Е.~Жуковского>>}
\newcommand{\opponentTwoJobPost}       % Оппонент 2, должность
{ведущий научный сотрудник}

\newcommand{\leadingOrganizationTitle} % Ведущая организация, дополнительные строки
{ФГБОУ ВО <<Национальный исследовательский университет <<Московский энергетический институт>>}

\newcommand{\defenseDate}              % Защита, дата
{<<\underline{\hspace{.5cm}}>> февраля 2018 г. в 15 часов}
%{19 апреля 2018 г. в 15 часов}
\newcommand{\defenseCouncilNumber}     % Защита, номер диссертационного совета
{Д\,002.073.03}
\newcommand{\defenseCouncilTitle}      % Защита, учреждение диссертационного совета
{ФИЦ ИУ РАН}
\newcommand{\defenseCouncilAddress}    % Защита, адрес учреждение диссертационного совета
{119333, г.~Москва, ул.~Вавилова, д.~42}
\newcommand{\defenseAddress}    % Защита, адрес учреждение диссертационного совета
{119333, г.~Москва, ул.~Вавилова, д.~40, конференц-зал}

\newcommand{\defenseSecretaryFio}      % Секретарь диссертационного совета, ФИО
{Безродных С.\,И.}
\newcommand{\defenseSecretaryRegalia}  % Секретарь диссертационного совета, регалии
{доктор физико-математических наук}    % Для сокращений есть ГОСТы, например: ГОСТ Р 7.0.12-2011 + http://base.garant.ru/179724/#block_30000

\newcommand{\synopsisLibrary}          % Автореферат, название библиотеки
{Вычислительного центра им.~А.\,А.~Дородницына ФИЦ ИУ РАН по адресу: г.~Москва, ул.~Вавилова, д.~42}
\newcommand{\synopsisURL}              % Автореферат, сайт
{\url{http://www.frccsc.ru/diss-council/00207303/diss/list/rogozin_oa}}
\newcommand{\synopsisDate}             % Автореферат, дата рассылки
{<<\underline{\hspace{.5cm}}>> февраля 2018 г}
%{16 февраля 2018 г}

% To avoid conflict with beamer class use \providecommand
\providecommand{\keywords}%            % Ключевые слова для метаданных PDF диссертации и автореферата
{}
