%&pdflatex
\documentclass[mathserif]{beamer} %, handout
\usetheme[progressbar=foot]{metropolis}
\setbeamertemplate{caption*}[numbered]

\usepackage[utf8x]{inputenc}
\usepackage[T2A]{fontenc}
\usepackage[english,russian]{babel}

\usepackage{amssymb, amsmath, amsfonts, mathtools, mathrsfs}
\usepackage[normalem]{ulem} % for sout
\usepackage{changepage} % for adjustwidth
\usepackage{tabu, tabulary}
\usepackage{comment}

%%% Основные сведения %%%
\newcommand{\thesisAuthorLastName}{Рогозин}
\newcommand{\thesisAuthorOtherNames}{Олег Анатольевич}
\newcommand{\thesisAuthorInitials}{О.\,А.}
\newcommand{\thesisAuthor}             % Диссертация, ФИО автора
{%
    \texorpdfstring{% \texorpdfstring takes two arguments and uses the first for (La)TeX and the second for pdf
        \thesisAuthorLastName~\thesisAuthorOtherNames% так будет отображаться на титульном листе или в тексте, где будет использоваться переменная
    }{%
        \thesisAuthorLastName, \thesisAuthorOtherNames% эта запись для свойств pdf-файла. В таком виде, если pdf будет обработан программами для сбора библиографических сведений, будет правильно представлена фамилия.
    }
}
\newcommand{\thesisAuthorShort}        % Диссертация, ФИО автора инициалами
{\thesisAuthorInitials~\thesisAuthorLastName}
%\newcommand{\thesisUdk}                % Диссертация, УДК
%{\todo{533.72}}
\newcommand{\thesisTitle}              % Диссертация, название
{Численный и асимптотический анализ некоторых классических задач молекулярной газодинамики}
\newcommand{\thesisSpecialtyNumber}    % Диссертация, специальность, номер
{01.02.05}
\newcommand{\thesisSpecialtyTitle}     % Диссертация, специальность, название
{Механика жидкости, газа и плазмы}
\newcommand{\thesisDegree}             % Диссертация, ученая степень
{кандидата физико-математических наук}
\newcommand{\thesisDegreeShort}        % Диссертация, ученая степень, краткая запись
{канд. физ.-мат. наук}
\newcommand{\thesisCity}               % Диссертация, город написания диссертации
{Москва}
\newcommand{\thesisYear}               % Диссертация, год написания диссертации
{2017}
\newcommand{\thesisOrganization}       % Диссертация, организация
{Московский физико-технический институт (государственный университет)}
\newcommand{\thesisOrganizationShort}  % Диссертация, краткое название организации для доклада
{МФТИ}

\newcommand{\thesisInOrganization}     % Диссертация, организация в предложном падеже: Работа выполнена в ...
{Московском физико-техническом институте (государственном университете)}

\newcommand{\supervisorFio}            % Научный руководитель, ФИО
{Черемисин Феликс Григорьевич}
\newcommand{\supervisorRegalia}        % Научный руководитель, регалии
{доктор физико-математических наук}
\newcommand{\supervisorFioShort}       % Научный руководитель, ФИО
{Ф.\,Г.~Черемисин}
\newcommand{\supervisorRegaliaShort}   % Научный руководитель, регалии
{д.ф.-м.н.}


\newcommand{\opponentOneFio}           % Оппонент 1, ФИО
{\todo{Фамилия Имя Отчество}}
\newcommand{\opponentOneRegalia}       % Оппонент 1, регалии
{\todo{доктор физико-математических наук, профессор}}
\newcommand{\opponentOneJobPlace}      % Оппонент 1, место работы
{\todo{Не очень длинное название для места работы}}
\newcommand{\opponentOneJobPost}       % Оппонент 1, должность
{\todo{старший научный сотрудник}}

\newcommand{\opponentTwoFio}           % Оппонент 2, ФИО
{\todo{Фамилия Имя Отчество}}
\newcommand{\opponentTwoRegalia}       % Оппонент 2, регалии
{\todo{кандидат физико-математических наук}}
\newcommand{\opponentTwoJobPlace}      % Оппонент 2, место работы
{\todo{Основное место работы c длинным длинным длинным длинным названием}}
\newcommand{\opponentTwoJobPost}       % Оппонент 2, должность
{\todo{старший научный сотрудник}}

\newcommand{\leadingOrganizationTitle} % Ведущая организация, дополнительные строки
{\todo{Федеральное государственное бюджетное образовательное учреждение высшего профессионального образования с~длинным длинным длинным длинным названием}}

\newcommand{\defenseDate}              % Защита, дата
{\todo{DD mmmmmmmm YYYY~г.~в~XX часов}}
\newcommand{\defenseCouncilNumber}     % Защита, номер диссертационного совета
{Д\,212.125.14}
\newcommand{\defenseCouncilTitle}      % Защита, учреждение диссертационного совета
{Московском авиационном институте (национальном исследовательском университете)}
\newcommand{\defenseCouncilAddress}    % Защита, адрес учреждение диссертационного совета
{125993, Москва, A-80, ГСП-3, Волоколамское шоссе, д.~4}
\newcommand{\defenseCouncilPhone}      % Телефон для справок
{+7~499~158-58-62}

\newcommand{\defenseSecretaryFio}      % Секретарь диссертационного совета, ФИО
{Гидаспов Владимир Юрьевич}
\newcommand{\defenseSecretaryRegalia}  % Секретарь диссертационного совета, регалии
{кандидат физико-математических наук, доцент}            % Для сокращений есть ГОСТы, например: ГОСТ Р 7.0.12-2011 + http://base.garant.ru/179724/#block_30000

\newcommand{\synopsisLibrary}          % Автореферат, название библиотеки
{МАИ или на сайте института \url{www.mai.ru}}
\newcommand{\synopsisDate}             % Автореферат, дата рассылки
{\todo{DD mmmmmmmm YYYY года}}

% To avoid conflict with beamer class use \providecommand
\providecommand{\keywords}%            % Ключевые слова для метаданных PDF диссертации и автореферата
{}

\graphicspath{{../images/}{images/}}

\title{\thesisTitle}
\author{%
    \emph{соискатель}:~\thesisAuthor\\ %Рогозин Олег Анатольевич\\
    \emph{руководитель}:~\supervisorRegaliaShort~\supervisorFio %д.ф.-м.н., проф. Черемисин Феликс Григорьевич\\
}
\institute{%
    \thesisOrganization % Вычислительный центр ФИЦ ИУ РАН
}
\date{}

\newcommand{\Kn}{\mathrm{Kn}}
\newcommand{\St}{\mathrm{St}}
\newcommand{\Ma}{\mathrm{Ma}}
\newcommand{\loc}{\mathrm{loc}}
\newcommand{\eqdef}{\overset{\mathrm{def}}{=}}
\newcommand{\dd}{\:\mathrm{d}}
\newcommand{\pder}[2][]{\frac{\partial#1}{\partial#2}}
\newcommand{\pderdual}[2][]{\frac{\partial^2#1}{\partial#2^2}}
\newcommand{\pderder}[3][]{\frac{\partial^2#1}{\partial#2\partial#3}}
\newcommand{\Pder}[2][]{\partial#1/\partial#2}
\newcommand{\dxi}{\boldsymbol{\dd\xi}}
\newcommand{\domega}{\boldsymbol{\dd\omega}}
\newcommand{\bomega}{\boldsymbol{\omega}}
\newcommand{\bxi}{\boldsymbol{\xi}}
\newcommand{\bh}{\boldsymbol{h}}
\newcommand{\be}{\boldsymbol{e}}
\newcommand{\Nu}{\mathcal{N}}
\newcommand{\Mu}{\mathcal{M}}
\newcommand{\OO}[1]{O(#1)}
\newcommand{\Set}[2]{\{\,{#1}:{#2}\,\}}
\newcommand{\xoverbrace}[2][\vphantom{\int}]{\overbrace{#1#2}}
\newcommand{\Cite}[2][]{\alert{\textsc{#2 #1}}}
\newcommand{\inner}[2]{\left\langle{#1},{#2}\right\rangle}

\newcommand\pro{\item[$+$]}
\newcommand\con{\item[$-$]}

% use Russian tradition
\renewcommand{\phi}{\varphi}
\renewcommand{\epsilon}{\varepsilon}

\begin{document}

\frame{\titlepage}

\begin{frame}
    \frametitle{Цели и задачи}
    \textbf{Объект исследования:} движение одноатомного газа
    \textbf{Предмет исследования:}
    \begin{itemize}
        \item методы численного и асимптотического анализа,
        \item физические свойства стационарных течений.
    \end{itemize}
    \textbf{Цель исследования:}
    \begin{itemize}
        \item развитие и апробация КПИМДС для неравномерных сеток,
        \item численный анализ некоторых течений разреженного газа.
    \end{itemize}
    \textbf{Актуальность:}
    \begin{itemize}
        \item активным развитием прикладных областей,
        \item потребностью в высокоточных численных методах,
        \item быстрым ростом доступных вычислительных ресурсов.
    \end{itemize}
\end{frame}

\begin{frame}
    \frametitle{Кинетическое описание газа}
    Функция распределения молекулярных скоростей
    \begin{equation}
        f(x,\xi): \Omega\times\mathbb{R}^d\mapsto\mathbb{R}_+
        \quad (\Omega\subset\mathbb{R}^d),
    \end{equation}
    Стационарное уравнение Больцмана
    \begin{equation}
        \xi\cdot\pder[f]{x} = \int_{\mathbb{R}^d}\dxi_* \int_{S^{d-1}} \dd\sigma
        \overbrace{B(\xi-\xi_*,\sigma)}^\text{collisional kernel}
        \Big( \overbrace{f'f'_*}^\text{gain} - \overbrace{ff_*}^\text{loss} \Big).
    \end{equation}
    Макроскопические переменные:
    \begin{gather*}
        \rho = \int_{\mathbb{R}^d} f\dxi, \\
        v = \frac1\rho\int_{\mathbb{R}^d} \xi f\dxi, \\
        T = \frac1{d\rho}\int_{\mathbb{R}^d} \left(|\xi|^2 - |v|^2\right) f\dxi.
    \end{gather*}
\end{frame}

\begin{frame}
    \frametitle{Функция распределения при \(\Delta{v}=2\) и \(Kn=1\)}
    Логарифмические сингулярности (\(x_{Bi}\) "--- граница):
    \begin{gather}
        \pder[f]{x_i}n_i = \frac{C}{\Kn}\ln\frac{(x_i-x_{Bi})n_i}{\Kn} + \OO{1}, \\
        \pder[f]{\xi_i}n_i = C\ln\xi_in_i + \OO{1}.
    \end{gather}
    \vspace{-20pt}
    \begin{columns}
        \column{.55\textwidth}
        \begin{figure}
            \includegraphics[width=\linewidth]{{{couette/distrib_f/kn1.0-boundary}}}\vspace{-10pt}
            \caption{Возле пластины}
        \end{figure}
        \column{.55\textwidth}
        \begin{figure}
            \includegraphics[width=\linewidth]{{{couette/distrib_f/kn1.0-center}}}\vspace{-10pt}
            \caption{Вблизи центра}
        \end{figure}
    \end{columns}
\end{frame}

\begin{frame}
    \frametitle{Зависимость сдвигового напряжения от \(\Kn\)}
    \vspace{-10pt}
    \[ P_{xy} = -\Gamma_1(T_H)\pder[v_H]{y}k + \OO{k^3} \]
    \vspace{-40pt}
    \centering\hspace{-1.5cm}
    \includegraphics[width=1.1\linewidth]{couette/integrated/shear}
    \hspace{-1.5cm}
\end{frame}

\begin{frame}
    \frametitle{Зависимость продольного теплового потока от \(\Kn\)}
    \vspace{-20pt}
    \[ q_x = q_{xK1}k
        + \frac{T_H}{p_H}\left( \frac{\Gamma_3}2 \pderdual[v_H]{y}
            + 4\Gamma_{10} \pder[T_H]{y}\pder[v_H]{y} \right)k^2
        + q_{xK2}k^2 + \OO{k^3} \]
    \vspace{-40pt}
    \centering\hspace{-1.5cm}
    \includegraphics[width=1.1\linewidth]{couette/integrated/qflow}
    \hspace{-1.5cm}
\end{frame}


\begin{frame}
    \frametitle{Транспортные коэффициенты для модели твёрдых сфер}
    \begin{alignat*}{2}
        \gamma_1 &= 1.270042427, \quad &\text{\Cite[1957]{Pekeris, Alterman}} \\
        \gamma_2 &= 1.922284065, \quad &\text{\Cite[1957]{Pekeris, Alterman}} \\
        \gamma_3 &= 1.947906335, \quad &\text{\Cite[1992]{Ohwada, Sone}} \\
        \gamma_7 &= 0.189200,    \quad &\text{\Cite[1996]{\footnotesize Sone, Aoki, Takata, Sugimoto, Bobylev}} \\
        \gamma_8 &= 1.495941968, \quad &\text{\Cite[2016]{Rogozin}} \\
        \gamma_9 &= 1.636073458, \quad &\text{\Cite[2016]{Rogozin}} \\
        \gamma_{10} &= 2.4497795.\quad &\text{\Cite[2016]{Rogozin}}
    \end{alignat*}
\end{frame}

\begin{frame}
    \frametitle{Поле скоростей при \(\Kn=0.02\)}
    \begin{figure}
        \hspace{-.5cm}
        \begin{overprint}
            \onslide<1| handout:3>
                \includegraphics[width=\linewidth]{{{snit/elliptic/kgf-0.02-flow}}}
                \vspace{-20pt}
                \caption{уравнения КГФ с граничными условиями ведущего порядка}
            \onslide<2| handout:2>
                \includegraphics[width=\linewidth]{{{snit/elliptic/first-0.02-flow}}}
                \vspace{-20pt}
                \caption{уравнения КГФ с граничными условиями первого порядка}
            \onslide<3| handout:1>
                \includegraphics[width=\linewidth]{{{snit/elliptic/curv-0.02-flow}}}
                \vspace{-20pt}
                \caption{уравнения КГФ с граничными условиями второго порядка}
            \onslide<4| handout:0>
                \includegraphics[width=\linewidth]{{{snit/elliptic/kes-0.02-flow}}}
                \vspace{-20pt}
                \caption{численное решение уравнения Больцмана}
        \end{overprint}
    \end{figure}
\end{frame}

\begin{frame}
    \frametitle{Заключение}
    Основные положения, выносимые на защиту:
    \begin{enumerate}
        \item\label{defpos:stencils}
        Критерии положительности для многоточечных проекционных шаблонов.
        \item\label{defpos:transport_coeffs}
        Транспортные коэффициенты для газа твёрдых сфер.
        \item\label{defpos:Couette_flow}
        Решение нелинейной плоской задачи Куэтта в широком диапазоне \(\Kn\) и \(\Ma\).
        \item\label{defpos:asymptotic_verification}
        Разница между асимптотическим и кинетическим решениями не более \(10^{-4}\).
        \item\label{defpos:boundary_conditions}
        Граничные условия первого и второго порядка для уравнений КГФ.
        \item\label{defpos:snit_forces}
        Электростатическая аналогия при обтекании равномерно нагретых тел.
    \end{enumerate}
\end{frame}

\begin{frame}
    \frametitle{References}
    \newcommand\zerodoi{10.1007/s00162-014-0334-5}
    \newcommand\firstdoi{10.1016/j.euromechflu.2016.06.011}
    \newcommand\seconddoi{10.1134/S0965542517060112}
    \newcommand\secondarxiv{1701.05811}
    \begin{itemize}
        \item {[Rogozin 2014]} \\
        “Computer simulation of slightly rarefied gas flows driven by significant temperature variations and their continuum limit”.
        In:~\textit{Theoretical and Computational Fluid Dynamics} 28.6 (2014), pp.~573–587.
        \textsc{doi}:~\href{http://dx.doi.org/\zerodoi}{\alert{\texttt{\zerodoi}}}.
        \item {[Rogozin 2016]} \\
        “Numerical analysis of the nonlinear plane Couette-flow problem of a rarefied gas for hard-sphere molecules”.
        In:~\textit{European Journal of Mechanics B/Fluids} 60 (2016), pp.~148–163.
        \textsc{doi}:~\href{http://dx.doi.org/\firstdoi}{\alert{\texttt{\firstdoi}}}.
        \item {[Rogozin 2017]} \\
        “Slow non-isothermal flows: numerical and asymptotic analysis of the Boltzmann equation”.
        In:~\textit{Computational Mathematics and Mathematical Physics} 57.7 (2017), pp.~1205–1229.
        \textsc{doi}:~\href{http://dx.doi.org/\seconddoi}{\alert{\texttt{\seconddoi}}}.
        \textsc{arXiv}:~\href{http://arxiv.org/abs/\secondarxiv}{\alert{\texttt{\secondarxiv}}}.
    \end{itemize}
\end{frame}

\end{document}
