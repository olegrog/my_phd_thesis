%&pdflatex
\documentclass[mathserif]{beamer} %, handout
\usetheme[progressbar=foot]{metropolis}
\setbeamertemplate{caption*}[numbered]

\usepackage[utf8x]{inputenc}
\usepackage[T2A]{fontenc}
\usepackage[english,russian]{babel}

\usepackage{amssymb, amsmath, amsfonts, mathtools, mathrsfs}
\usepackage[normalem]{ulem} % for sout
\usepackage{changepage} % for adjustwidth
\usepackage{tabu, tabulary}
\usepackage{comment}

%%% Основные сведения %%%
\newcommand{\thesisAuthorLastName}{Рогозин}
\newcommand{\thesisAuthorOtherNames}{Олег Анатольевич}
\newcommand{\thesisAuthorInitials}{О.\,А.}
\newcommand{\thesisAuthor}             % Диссертация, ФИО автора
{%
    \texorpdfstring{% \texorpdfstring takes two arguments and uses the first for (La)TeX and the second for pdf
        \thesisAuthorLastName~\thesisAuthorOtherNames% так будет отображаться на титульном листе или в тексте, где будет использоваться переменная
    }{%
        \thesisAuthorLastName, \thesisAuthorOtherNames% эта запись для свойств pdf-файла. В таком виде, если pdf будет обработан программами для сбора библиографических сведений, будет правильно представлена фамилия.
    }
}
\newcommand{\thesisAuthorShort}        % Диссертация, ФИО автора инициалами
{\thesisAuthorInitials~\thesisAuthorLastName}
%\newcommand{\thesisUdk}                % Диссертация, УДК
%{\todo{533.72}}
\newcommand{\thesisTitle}              % Диссертация, название
{Численный и асимптотический анализ некоторых классических задач молекулярной газодинамики}
\newcommand{\thesisSpecialtyNumber}    % Диссертация, специальность, номер
{01.02.05}
\newcommand{\thesisSpecialtyTitle}     % Диссертация, специальность, название
{Механика жидкости, газа и плазмы}
\newcommand{\thesisDegree}             % Диссертация, ученая степень
{кандидата физико-математических наук}
\newcommand{\thesisDegreeShort}        % Диссертация, ученая степень, краткая запись
{канд. физ.-мат. наук}
\newcommand{\thesisCity}               % Диссертация, город написания диссертации
{Москва}
\newcommand{\thesisYear}               % Диссертация, год написания диссертации
{2017}
\newcommand{\thesisOrganization}       % Диссертация, организация
{Федеральный исследовательский центр <<Информатика и управление>> Российской академии наук}
\newcommand{\thesisOrganizationShort}  % Диссертация, краткое название организации для доклада
{ФИЦ ИУ РАН}

\newcommand{\thesisInOrganization}     % Диссертация, организация в предложном падеже: Работа выполнена в ...
{Федеральном исследовательском центре <<Информатика и управление>> Российской академии наук}

\newcommand{\supervisorFio}            % Научный руководитель, ФИО
{Черемисин Феликс Григорьевич}
\newcommand{\supervisorRegalia}        % Научный руководитель, регалии
{доктор физико-математических наук, профессор}
\newcommand{\supervisorFioShort}       % Научный руководитель, ФИО
{Ф.\,Г.~Черемисин}
\newcommand{\supervisorRegaliaShort}   % Научный руководитель, регалии
{д.ф.-м.н., проф.}


\newcommand{\opponentOneFio}           % Оппонент 1, ФИО
{\todo{Фамилия Имя Отчество}}
\newcommand{\opponentOneRegalia}       % Оппонент 1, регалии
{\todo{доктор физико-математических наук, профессор}}
\newcommand{\opponentOneJobPlace}      % Оппонент 1, место работы
{\todo{Не очень длинное название для места работы}}
\newcommand{\opponentOneJobPost}       % Оппонент 1, должность
{\todo{старший научный сотрудник}}

\newcommand{\opponentTwoFio}           % Оппонент 2, ФИО
{\todo{Фамилия Имя Отчество}}
\newcommand{\opponentTwoRegalia}       % Оппонент 2, регалии
{\todo{кандидат физико-математических наук}}
\newcommand{\opponentTwoJobPlace}      % Оппонент 2, место работы
{\todo{Основное место работы c длинным длинным длинным длинным названием}}
\newcommand{\opponentTwoJobPost}       % Оппонент 2, должность
{\todo{старший научный сотрудник}}

\newcommand{\leadingOrganizationTitle} % Ведущая организация, дополнительные строки
{\todo{Федеральное государственное бюджетное образовательное учреждение высшего профессионального образования с~длинным длинным длинным длинным названием}}

\newcommand{\defenseDate}              % Защита, дата
{\todo{DD mmmmmmmm YYYY~г.~в~XX часов}}
\newcommand{\defenseCouncilNumber}     % Защита, номер диссертационного совета
{Д\,002.073.03}
\newcommand{\defenseCouncilTitle}      % Защита, учреждение диссертационного совета
{Федеральном исследовательском центре <<Информатика и управление>> Российской академии наук}
\newcommand{\defenseCouncilAddress}    % Защита, адрес учреждение диссертационного совета
{119333, Москва, ул.~Вавилова, д.~44, кор.~2}
\newcommand{\defenseCouncilPhone}      % Телефон для справок
{+7~(499)~135-51-64}

\newcommand{\defenseSecretaryFio}      % Секретарь диссертационного совета, ФИО
{Безродных Сергей Игоревич}
\newcommand{\defenseSecretaryRegalia}  % Секретарь диссертационного совета, регалии
{доктор физико-математических наук}    % Для сокращений есть ГОСТы, например: ГОСТ Р 7.0.12-2011 + http://base.garant.ru/179724/#block_30000

\newcommand{\synopsisLibrary}          % Автореферат, название библиотеки
{ФИЦ ИУ РАН или на сайте института \url{www.frccsc.ru}}
\newcommand{\synopsisDate}             % Автореферат, дата рассылки
{\todo{DD mmmmmmmm YYYY года}}

% To avoid conflict with beamer class use \providecommand
\providecommand{\keywords}%            % Ключевые слова для метаданных PDF диссертации и автореферата
{}

\graphicspath{{../images/}{images/}}

\title{\thesisTitle}
\author{%
    \emph{соискатель}:~\thesisAuthor\\ %Рогозин Олег Анатольевич\\
    \emph{руководитель}:~\supervisorRegaliaShort~\supervisorFio %д.ф.-м.н., проф. Черемисин Феликс Григорьевич\\
}
\institute{%
    \thesisOrganization % Вычислительный центр ФИЦ ИУ РАН
}
\date{}

\newcommand{\Kn}{\mathrm{Kn}}
\newcommand{\St}{\mathrm{St}}
\newcommand{\Ma}{\mathrm{Ma}}
\newcommand{\loc}{\mathrm{loc}}
\newcommand{\eqdef}{\overset{\mathrm{def}}{=}}
\newcommand{\dd}{\:\mathrm{d}}
\newcommand{\der}[2][]{\frac{\dd#1}{\dd#2}}
\newcommand{\derdual}[2][]{\frac{\dd^2#1}{\dd#2^2}}
\newcommand{\pder}[2][]{\frac{\partial#1}{\partial#2}}
\newcommand{\pderdual}[2][]{\frac{\partial^2#1}{\partial#2^2}}
\newcommand{\pderder}[3][]{\frac{\partial^2#1}{\partial#2\partial#3}}
\newcommand{\Pder}[2][]{\partial#1/\partial#2}
\newcommand{\dxi}{\boldsymbol{\dd\xi}}
\newcommand{\domega}{\boldsymbol{\dd\omega}}
\newcommand{\bomega}{\boldsymbol{\omega}}
\newcommand{\bxi}{\boldsymbol{\xi}}
\newcommand{\bh}{\boldsymbol{h}}
\newcommand{\be}{\boldsymbol{e}}
\newcommand{\Nu}{\mathcal{N}}
\newcommand{\Mu}{\mathcal{M}}
\newcommand{\OO}[1]{O(#1)}
\newcommand{\Set}[2]{\{\,{#1}:{#2}\,\}}
\newcommand{\xoverbrace}[2][\vphantom{\int}]{\overbrace{#1#2}}
\newcommand{\Cite}[2][]{\alert{\textsc{#2 #1}}}
\newcommand{\inner}[2]{\left\langle{#1},{#2}\right\rangle}
\newcommand{\onwall}[1]{\left(#1\right)_0}
\newcommand{\deltann}[2]{(\delta_{#1#2}-n_#1 n_#2)}

% use Russian tradition
\renewcommand{\phi}{\varphi}
\renewcommand{\epsilon}{\varepsilon}

\begin{document}

\frame{\titlepage}

\begin{frame}
    \frametitle{Цели и задачи}
    \textbf{Объект исследования:}\vspace{-5pt}
    \begin{itemize}
        \item динамика разреженного одноатомного газа
    \end{itemize}
    \textbf{Предмет исследования:}\vspace{-5pt}
    \begin{itemize}
        \item методы численного и асимптотического анализа,
        \item физические свойства стационарных течений.
    \end{itemize}
    \textbf{Цель исследования:}\vspace{-5pt}
    \begin{itemize}
        \item развитие и апробация КПИМДС для неравномерных сеток,
        \item численный анализ некоторых известных течений.
    \end{itemize}
    \textbf{Актуальность:}\vspace{-5pt}
    \begin{itemize}
        \item активным развитием прикладных областей,
        \item потребностью в высокоточных численных методах,
        \item быстрым ростом доступных вычислительных ресурсов.
    \end{itemize}
\end{frame}

\begin{frame}
    \frametitle{Кинетическое описание газа}
    Функция распределения молекулярных скоростей
    \begin{equation}
        f(x_i,\xi_i): \Omega\times\mathbb{R}^D\mapsto\mathbb{R}_+
        \quad (\Omega\subset\mathbb{R}^D),
    \end{equation}
    \pause
    Стационарное уравнение Больцмана:
    \begin{equation}
        \xi_i\pder[f]{x_i} = \int_{\mathbb{R}^D}\dxi_* \int_{S^{D-1}} \dd\Omega(\boldsymbol{\alpha})
        \overbrace{B(\bxi-\bxi_*,\boldsymbol{\alpha})}^\text{collisional kernel}
        \Big( \overbrace{f'f'_*}^\text{gain} - \overbrace{ff_*}^\text{loss} \Big),
    \end{equation}
    \begin{gather*}
        f=f(\bxi), \quad f_*=f(\bxi_*), \quad f'=f(\bxi'), \quad f'_*=f(\bxi'_*), \\
        \xi_i' = \xi_i+\alpha_i\alpha_j \left(\xi_{j*}-\xi_j\right), \quad
        \xi_{i*}' = \xi_{i*}-\alpha_i\alpha_j \left(\xi_{j*}-\xi_j\right).
    \end{gather*}
    \pause
    Макроскопические переменные:
    \begin{equation}
        \begin{gathered}
        \rho = \int_{\mathbb{R}^D} f\dxi, \quad
        v_i = \frac1\rho\int_{\mathbb{R}^D} \xi_i f\dxi, \\
        T = \frac1{D\rho}\int_{\mathbb{R}^D} \left(\xi_i - v_i\right)^2 f\dxi.
        \end{gathered}
    \end{equation}
\end{frame}

\begin{frame}
    \frametitle{Дискретизация пространства скоростей}
    % Группа для УБ, просто множество для модельных типа БГК
    Решётка \(\mathcal{V} = \Set{\xi_\gamma}{\gamma\in\Gamma}\) построена так, что
    \begin{equation}\label{eq:xi_cubature}
        \int F(\bxi) \dxi \approx \sum_{\gamma\in\Gamma} F_\gamma w_\gamma =
            \sum_{\gamma\in\Gamma} \hat{F_\gamma},
            \quad F_\gamma = F(\bxi_\gamma),
    \end{equation}\vspace{-10pt}

    где \(\sum_{\gamma\in\Gamma} w_\gamma = V_\Gamma\) "--- объём, покрываемый скоростной сеткой.
    \pause\vspace{10pt}

    Дискретизация интеграла столкновений:
    \begin{equation}\label{eq:discrete_symm_ci}
        \footnotesize
        \hat{J}_\gamma = \frac{\pi V_\Gamma^2}{\displaystyle\sum_{\nu\in\Nu} w_{\nu}w_{*\nu}}
            \sum_{\nu\in\Nu} \bigg(
                \delta_{\gamma\nu} + \delta_{*\gamma\nu} -
                \xoverbrace{ \alert{\delta'_{\gamma\nu}} - \alert{\delta'_{*\gamma\nu}} }^\text{projection}
            \bigg)\bigg(
                \xoverbrace{ \frac{w_{\nu}w_{*\nu}}{\alert{w'_{\nu}w'_{*\nu}}}
                \alert{\hat{f}'_{\nu}\hat{f}'_{*\nu}} }^\text{interpolation} - \hat{f}_{\nu}\hat{f}_{*\nu}
            \bigg)B_\nu.
    \end{equation}\vspace{-10pt}
    % микроскопическая консервативность
    Цель --- чтобы каждый член суммы\vspace{5pt}
    \begin{itemize}
        \item сохраняет массу, импульс, энергию,
        \item не уменьшает энтропию.
    \end{itemize}
\end{frame}

\begin{frame}
    \frametitle{Консервативное проецирование}
    \centering{\Large\bf Почему метод называется проекционным?}
    \vspace{10pt}

    Проекционная процедура Петрова"--~Галёркина:
    \begin{equation}\label{eq:Petrov-Galerkin}
        \int \xoverbrace{ \psi_s(\bxi_\gamma) }^\text{test space} \bigg(
            \alert{\delta(\bxi'-\bxi_\gamma)} - \sum_{a\in\Lambda} r_{\lambda,a}
            \xoverbrace{ \delta(\bxi_{\lambda+s_a}-\bxi_\gamma) }^\text{trial space}
        \bigg) \dxi_\gamma = 0.
    \end{equation}
    \begin{equation*}
        \psi_0 = 1,\;\psi_i = \xi_i,\;\psi_4 = \xi_i^2
        \quad\implies\quad\text{консервативность}
    \end{equation*}
\end{frame}

\section{Нелинейное течение Куэтта}

\begin{frame}
    \frametitle{Плоское течение Куэтта}
    \begin{columns}
        \column{.5\textwidth}
        \hspace{-10pt}\includegraphics{couette/tikz/geometry}
        \vspace{-10pt}
        \[ \Delta{T} = 0 \]
        \column{.6\textwidth}
        \begin{itemize}
            \item Линейная задача \(\Delta{v} = o(1)\): \[ x_i\to\mathbb{R}^1, \zeta_i\to\mathbb{R}^{\alert{2}}\]
            Решена в \newline \Cite[1990]{Sone, Takata, Ohwada}
            \bigskip
            \item Нелинейная задача \(\Delta{v} = \OO{1}\): \[ x_i\to\mathbb{R}^1, \zeta_i\to\mathbb{R}^{\alert{3}} \]
            Решена в \Cite[2016]{Rogozin}
        \end{itemize}
    \end{columns}
\end{frame}

\begin{frame}
    \frametitle{Функция распределения при \(\Delta{v}=2\) и \(Kn=1\)}
    Логарифмическая сингулярность \Cite[2016]{Chen, Funagane, Liu, Takata}:
    \begin{equation}
        \pder[f]{\xi_i}n_i = C\ln\xi_in_i + \OO{1}.
    \end{equation}
    \vspace{-20pt}
    \begin{columns}
        \column{.55\textwidth}
        \begin{figure}
            \includegraphics[width=\linewidth]{{{couette/distrib_f/kn1.0-boundary}}}\vspace{-10pt}
            \caption{Возле пластины}
        \end{figure}
        \column{.55\textwidth}
        \begin{figure}
            \includegraphics[width=\linewidth]{{{couette/distrib_f/kn1.0-center}}}\vspace{-10pt}
            \caption{Вблизи центра}
        \end{figure}
    \end{columns}
\end{frame}

\begin{frame}
    \frametitle{Профили макроскопических величин}
    Логарифмическая сингулярность \Cite[2014]{Chen, Liu, Takata}:
    \begin{equation}
        \pder[f]{x_i}n_i = \frac{C}{\Kn}\ln\frac{(x_i-x_{Bi})n_i}{\Kn} + \OO{1},
        \quad x_{Bi}\text{ "--- граница}.
    \end{equation}
    \vspace{-20pt}
    \begin{columns}
        \column{.55\textwidth}
        \begin{figure}
            \includegraphics[width=\linewidth]{couette/profiles/qx}\vspace{-10pt}
            \caption{Продольный поток тепла}
        \end{figure}
        \column{.55\textwidth}
        \begin{figure}
            \includegraphics[width=\linewidth]{couette/profiles/Pzz}\vspace{-10pt}
            \caption{Величина \(2\int(\xi_z^2-\xi_y^2)f\dxi\)}
        \end{figure}
    \end{columns}
\end{frame}

\begin{frame}
    \frametitle{Зависимость сдвигового напряжения от \(\Kn\)}
    \vspace{-10pt}
    \[ P_{xy} = -\Gamma_1(T_H)\der[v_H]{y}k + \OO{k^3} \]
    \vspace{-10pt}

    \centering\hspace{-1.5cm}
    \includegraphics[width=1.0\linewidth]{couette/integrated/shear}
    \hspace{-1.5cm}
\end{frame}

\begin{frame}
    \frametitle{Зависимость продольного теплового потока от \(\Kn\)}
    \vspace{-20pt}
    \[ q_x = q_{xK1}k
        + \frac{T_H}{p_H}\left( \frac{\Gamma_3}2 \derdual[v_H]{y}
            + 4\Gamma_{10} \der[T_H]{y}\der[v_H]{y} \right)k^2
        + q_{xK2}k^2 + \OO{k^3} \]
    \vspace{-10pt}

    \centering\hspace{-1.5cm}
    \includegraphics[width=1.0\linewidth]{couette/integrated/qflow}
    \hspace{-1.5cm}
\end{frame}

\begin{frame}
    \frametitle{Зависимость \(2\int(\xi_z^2-\xi_y^2)f\dxi\) от \(\Kn\)}
    \vspace{-15pt}
    \[ P_{zzH} - P_{yyH} = \frac{2\Gamma_8}{p_H}\left(\der[v_H]{y}\right)^2 k^2 + \OO{k^3} \]
    \vspace{-15pt}

    \centering\hspace{-1.5cm}
    \includegraphics[width=1.0\linewidth]{couette/integrated/pzz}
    \hspace{-1.5cm}
\end{frame}

\begin{frame}
    \frametitle{Транспортные коэффициенты для модели твёрдых сфер}
    \begin{alignat*}{2}
        \gamma_1 &= 1.270042427, \quad &\text{\Cite[1957]{Pekeris, Alterman}} \\
        \gamma_2 &= 1.922284065, \quad &\text{\Cite[1957]{Pekeris, Alterman}} \\
        \gamma_3 &= 1.947906335, \quad &\text{\Cite[1992]{Ohwada, Sone}} \\
        \gamma_7 &= 0.189200,    \quad &\text{\Cite[1996]{\footnotesize Sone, Aoki, Takata, Sugimoto, Bobylev}} \\
        \gamma_8 &= 1.495941968, \quad &\text{\Cite[2016]{Rogozin}} \\
        \gamma_9 &= 1.636073458, \quad &\text{\Cite[2016]{Rogozin}} \\
        \gamma_{10} &= 2.4497795.\quad &\text{\Cite[2016]{Rogozin}}
    \end{alignat*}
\end{frame}

\section{Задача Соне--Бобылева}

\begin{frame}
    \frametitle{Задача Соне--Бобылева}
    \begin{columns}
        \column{.5\textwidth}
        \hspace{-10pt}\includegraphics{snit/tikz/sone_bobylev}
        \[ T_B = 1 - \frac{\cos(2\pi x)}{2} \]
        \column{.55\textwidth}
        \begin{itemize}
            \item \(\Kn\to0\) для модели БГК и твёрдых сфер \\ \Cite[1996]{Sone, ..., Bobylev}
            \smallskip
            \item \(\Kn=\OO{1}\) для модели БГК \Cite[1996]{Sone, ..., Bobylev}
            \smallskip
            \item \(\Kn=\OO{1}\) для твёрдых сфер \Cite[2017]{Rogozin}
        \end{itemize}
    \end{columns}
\end{frame}

\begin{frame}
    \frametitle{Уравнения гидродинамического типа}
    \Cite[1970, 1971, 1976]{Коган, Галкин, Фридлендер}:
    \begin{align*}
        \pder{x_i}\left(\frac{u_{i1}}{T_0}\right) &= 0, \\
        \pder{x_j}\left(\frac{u_{i1}u_{j1}}{T_0}\right)
            &-\frac12\pder{x_j}\left[\Gamma_1\left(
                \pder[u_{i1}]{x_j} + \pder[u_{j1}]{x_i} - \frac23\pder[u_{kH1}]{x_k}\delta_{ij}
            \right)\right] \notag\\
            &-\alert{\left[
                \frac{\Gamma_7}{\Gamma_2}\frac{u_{j1}}{T_0}\pder[T_0]{x_j}
                + \frac{\Gamma_2^2}4 \left(\frac{\Gamma_7}{\Gamma_2^2}\right)'
                    \left(\pder[T_0]{x_j}\right)^2
            \right]\pder[T_0]{x_i}} \notag\\
            &= -\frac12\pder[p_2^\dag]{x_i}, \\
        \pder[u_{i1}]{x_i} &= \frac12\pder{x_i}\left(\Gamma_2\pder[T_0]{x_i}\right),
    \end{align*}
    где
    \begin{equation*}
        p_2^\dag = p_0 p_2
            + \frac23\pder{x_k}\left(\Gamma_3\pder[T_0]{x_k}\right)
            - \frac{\Gamma_7}{6}\left(\pder[T_0]{x_k}\right)^2, \quad u_{i1} = p_0v_{i1}.
    \end{equation*}
\end{frame}

\begin{frame}
    \frametitle{Слой Кнудсена первого порядка}
    \begin{equation}
        \alert{T_0} = T_{B0}. \label{eq:bc_T0}
    \end{equation}

    \begin{gather}
        \frac{p_0}{T_{B0}}
            \begin{bmatrix} \alert{T_1} - T_{B1} \\ T_{K1} \\ T_{B0}^2\rho_{K1} \end{bmatrix} =
            \onwall{\pder[T_0]{x_j}} n_j \begin{bmatrix} d_1 \\ \Theta_1(\eta) \\ p_0\Omega_1(\eta) \end{bmatrix}, \\
        \left\{
        \begin{aligned}
            & \frac1{\sqrt{T_{B0}}}\begin{bmatrix} (\alert{u_{j1}} - u_{Bj1}) \\ u_{jK1} \end{bmatrix} \deltann{i}{j} =
                \onwall{\pder[T_0]{x_j}}\begin{bmatrix} k_0 \\ Y_0(\eta) \end{bmatrix}\deltann{i}{j}, \\
            & \alert{u_{j1}} n_j = 0.
        \end{aligned}
        \right.
    \end{gather}
    \vspace{-20pt}
    \begin{itemize}
        \item \(\eta\) "--- растянутая координата вдоль нормали
        \item \(\Theta_1, \Omega_1, Y_0 = \OO{\eta^{-\infty}}\)
        \item \(\onwall{\cdots}\) "--- значение на границе (\(\eta=0\))
    \end{itemize}
\end{frame}

\begin{frame}
    \frametitle{Изотермические линии при \(\Kn=0.01\)}
    \begin{columns}
        \column{.55\textwidth}
        \begin{figure}
            \centering
            \includegraphics[width=\linewidth]{{{snit/contours/asym-0.01-temp}}}
            \caption{уравнения КГФ с учётом температурного скачка}
        \end{figure}
        \column{.55\textwidth}
        \begin{figure}
            \centering
            \includegraphics[width=\linewidth]{{{snit/contours/kes-0.01-temp}}}
            \caption{уравнение Больцмана \hspace{100pt}\vphantom{.}}
        \end{figure}
    \end{columns}
\end{frame}

\begin{frame}
    \frametitle{Поле скоростей при \(\Kn=0.01\)}
    \begin{columns}
        \column{.55\textwidth}
        \begin{figure}
            \includegraphics[width=\linewidth]{{{snit/contours/asym-0.01-vel}}}
            \caption{уравнения КГФ с учётом температурного скачка}
        \end{figure}
        \column{.55\textwidth}
        \begin{figure}
            \includegraphics[width=\linewidth]{{{snit/contours/kes-0.01-vel}}}
            \caption{уравнение Больцмана \hspace{100pt}\vphantom{.}}
        \end{figure}
    \end{columns}
\end{frame}

\begin{frame}
    \frametitle{Зависимость температурного поля от \(\Kn\)}
    \begin{figure}
        \includegraphics{snit/vs_kn/top_temp}
        \caption{интеграл температуры вдоль оси симметрии}
    \end{figure}
\end{frame}

\section{Течение между некоаксиальными цилиндрами}

\begin{frame}
    \frametitle{Поле скоростей в континуальном пределе}
    \centering
    \begin{figure}
    \begin{overprint}
        \onslide<1| handout:1>
            \[ T_1 = 1, \quad T_2 = 5.\]
            \hspace{-1cm}
            \includegraphics[width=1.15\linewidth]{kgf/noncoaxial/cylinder5}
        \onslide<2| handout:0>
            \[ T_1 = 1, \quad T_2 = 50.\]
            \hspace{-1cm}
            \includegraphics[width=1.15\linewidth]{kgf/noncoaxial/cylinder50}
    \end{overprint}
    \hspace{-.5cm}
    \end{figure}
\end{frame}

\begin{frame}
    \frametitle{Зависимость поля скоростей от \(\tau = T_2-T_1\)}
    \centering
    \begin{columns}
        \column{.7\textwidth}
        \begin{figure}
            \includegraphics[width=\linewidth]{kgf/temper/U-tau-cylinders}
            \vspace{-.5cm}\caption{максимальное значение скорости}
        \end{figure}
        \column{.4\textwidth}
        \[ u_{i1} = \OO{\tau^3}, \quad \tau\to0, \]
        \[ u_{i1} = \OO{\tau^{3/2}}, \quad \tau\to\infty. \]
    \end{columns}
\end{frame}

\begin{frame}
    \frametitle{Зависимость действующей силы от \(\tau = T_2-T_1\)}
    \vspace{-.2cm}
    \[ p_0 \oint_S F_{x2}\dd{S} = \OO{\tau^2}. \]
    \vspace{-.7cm}
    \begin{columns}
        \column{.6\textwidth}
        \begin{figure}
            \includegraphics[width=\linewidth]{kgf/temper/F-tau-cylinders-inner}
            \vspace{-.5cm}\caption{внутренний цилиндр}
        \end{figure}
        \column{.6\textwidth}
        \begin{figure}
            \includegraphics[width=\linewidth]{kgf/temper/F-tau-cylinders-outer}
            \vspace{-.5cm}\caption{внешний цилиндр}
        \end{figure}
    \end{columns}
\end{frame}

\begin{frame}
    \frametitle{Зависимость действующей силы от расстояния между осями}
    \vspace{-.5cm}
    \[ p_0 \oint_S F_{x2}\dd{S} = \begin{cases}
        \OO{(1-d)^{-1}} & \quad\text{для cфер}, \\
        \OO{(1-d)^{-3/2}} & \quad\text{для цилиндров}, \\
        \end{cases} \quad d\to1. \]
    \vspace{-.7cm}
    \begin{columns}
        \column{.6\textwidth}
        \begin{figure}
            \includegraphics[width=\linewidth]{kgf/force/forces}
        \end{figure}
        \column{.6\textwidth}
        \begin{figure}
            \includegraphics[width=\linewidth]{kgf/force/forces-close}
        \end{figure}
    \end{columns}
\end{frame}


\section{Течение между эллиптическими цилиндрами}

\begin{frame}
    \frametitle{Слой Кнудсена второго порядка}
    \footnotesize
    \begin{equation*}
        \begin{aligned}
            \frac1{\sqrt{T_{B0}}}&
                \begin{bmatrix} (u_{j2} - u_{Bj2}) \\ u_{jK2} \end{bmatrix}\deltann{i}{j} = \\
            &- \left.\frac{\sqrt{T_{B0}}}{p_0}\onwall{\pder[u_{j1}]{x_k}} \deltann{i}{j}n_k
                \begin{bmatrix} k_0 \\ Y_0(\eta) \end{bmatrix} \quad\right\}\text{velocity jump}\\
            &- \left.\frac{T_{B0}}{p_0}\onwall{\pderder[T_0]{x_j}{x_k}} \deltann{i}{j}n_k
                \begin{bmatrix} a_4 \\ Y_{a4}(\eta) \end{bmatrix} \quad\right\}\text{thermal slip} \\
            &\left.\begin{aligned}
                &- \bar\kappa\frac{T_{B0}}{p_0}\onwall{\pder[T_0]{x_j}} \deltann{i}{j}
                \begin{bmatrix} a_5 \\ Y_{a5}(\eta) \end{bmatrix} \\
                &- \kappa_{jk}\frac{T_{B0}}{p_0}\onwall{\pder[T_0]{x_k}} \deltann{i}{j}
                \begin{bmatrix} a_6 \\ Y_{a6}(\eta) \end{bmatrix}
            \end{aligned} \quad\right\}\text{curvature terms}\\
            &- \left.\pder[T_{B1}]{x_j} \deltann{i}{j}
                \begin{bmatrix} K_1 \\ \frac12 Y_1(\eta) \end{bmatrix}. \quad\right\}\text{thermal creep}
        \end{aligned}\label{eq:boundary_u2t}
    \end{equation*}
    \vspace{-10pt}
    \begin{itemize}
        \item \(\kappa_{ij} = \kappa_1 l_i l_j + \kappa_2 m_i m_j\) is the curvature tensor, \(\bar\kappa = (\kappa_1+\kappa_2)/2\)
    \end{itemize}
\end{frame}

\begin{frame}
    \frametitle{Слой Кнудсена второго порядка}
    \footnotesize
    \begin{gather*}
        \begin{multlined}
            \frac1{\sqrt{T_{B0}}}
                \begin{bmatrix} (u_{j2} - u_{Bj2}) \\ u_{jK2} \end{bmatrix} n_j = \\
            - \frac{T_{B0}}{p_0}\left[ \onwall{\pderder[T_0]{x_i}{x_j}}n_i n_j
                + 2\bar\kappa\onwall{\pder[T_0]{x_i}}n_i \right]
                \begin{bmatrix} \frac12\int_0^\infty Y_1(\eta_0)\dd\eta_0 \\
                    \frac12\int_\infty^{\eta} Y_1(\eta_0)\dd\eta_0 \end{bmatrix}.
        \end{multlined}\label{eq:boundary_u2n}\\
        \begin{aligned}
            \frac{p_0}{T_{B0}}
                \begin{bmatrix} T_2 - T_{B2} \\ T_{K2} \\ T_{B0}^2\rho_{K2} \end{bmatrix}
            &= \onwall{\pder[T_1]{x_j}} n_j
                \begin{bmatrix} d_1 \\ \Theta_1(\eta) \\ p_0\Omega_1(\eta) \end{bmatrix} \\
            &+ \frac{T_{B0}}{p_0}\onwall{\pderder[T_0]{x_i}{x_j}} n_i n_j
                \begin{bmatrix} d_3 \\ \Theta_3(\eta) \\ p_0\Omega_3(\eta) \end{bmatrix} \\
            &+ \bar\kappa\frac{T_{B0}}{p_0}\onwall{\pder[T_0]{x_i}} n_i
                \begin{bmatrix} d_5 \\ \Theta_5(\eta) \\ p_0\Omega_5(\eta) \end{bmatrix}.
        \end{aligned}\label{eq:boundary_T2}
    \end{gather*}
\end{frame}


\begin{frame}
    \frametitle{Поле скоростей при \(\Kn=0.02\)}
    \begin{figure}
        \hspace{-.5cm}
        \begin{overprint}
            \onslide<1| handout:3>
                \includegraphics[width=\linewidth]{{{snit/elliptic/kgf-0.02-flow}}}
                \vspace{-20pt}
                \caption{уравнения КГФ с граничными условиями ведущего порядка}
            \onslide<2| handout:2>
                \includegraphics[width=\linewidth]{{{snit/elliptic/first-0.02-flow}}}
                \vspace{-20pt}
                \caption{уравнения КГФ с граничными условиями первого порядка}
            \onslide<3| handout:1>
                \includegraphics[width=\linewidth]{{{snit/elliptic/curv-0.02-flow}}}
                \vspace{-20pt}
                \caption{уравнения КГФ с граничными условиями второго порядка}
            \onslide<4| handout:0>
                \includegraphics[width=\linewidth]{{{snit/elliptic/kes-0.02-flow}}}
                \vspace{-20pt}
                \caption{численное решение уравнения Больцмана}
        \end{overprint}
    \end{figure}
\end{frame}

\begin{frame}
    \frametitle{Заключение}
    \textbf{Основные положения, выносимые на защиту:}
    \begin{enumerate}
        \item\label{defpos:stencils}
        Критерии положительности для многоточечных проекционных шаблонов.
        \item\label{defpos:transport_coeffs}
        Транспортные коэффициенты для газа твёрдых сфер.
        \item\label{defpos:Couette_flow}
        Решение нелинейной плоской задачи Куэтта в широком диапазоне \(\Kn\) и \(\Ma\).
        \item\label{defpos:asymptotic_verification}
        Разница между асимптотическим и кинетическим решениями не более \(10^{-4}\).
        \item\label{defpos:boundary_conditions}
        Граничные условия первого и второго порядка для уравнений КГФ.
        \item\label{defpos:snit_forces}
        Электростатическая аналогия при обтекании равномерно нагретых тел.
    \end{enumerate}
\end{frame}

\begin{frame}
    \frametitle{Публикации с соавторами}
    \begin{itemize}
        \item \textit{Computing of gas flows in micro- and nanoscale channels on the base of the Boltzmann kinetic equation}
        / Yu. A. Anikin, E. P. Derbakova, O. I. Dodulad, Yu. Yu. Kloss, D. V. Martynov,
            \textbf{O. A. Rogozin}, P. V. Shuvalov, F. G. Tcheremissine. \\
        \alert{Procedia Comput. Sci.} 1.1 (2010), 735--744.
        \item \textit{Проблемно-моделирующая среда для расчётов и анализа газокинетических процессов}
        / О. И. Додулад, Ю. Ю. Клосс, Д. В. Мартынов,
            \textbf{О. А. Рогозин}, В. В. Рябченков, П. В. Шувалов, Ф. Г. Черемисин. \\
        \alert{Нано- и микросистемная техника} 2 (2011), 12--17.
    \end{itemize}
\end{frame}

\begin{frame}
    \frametitle{Основные публикации}
    \begin{itemize}
        \item \textit{Computer simulation of slightly rarefied gas flows driven
            by significant temperature variations and their continuum limit.} \\
        \alert{Theor. Comput. Fluid Dyn.} 28.6 (2014), 573--587.
        \item \textit{Numerical analysis of the nonlinear plane Couette-flow problem of a rarefied gas for hard-sphere molecules.} \\
        \alert{Eur. J. Mech. B/Fluids} 60 (2016), 148--163.
        \item \textit{Медленные неизотермические течения: численный и асимптотический анализ уравнения Больцмана.} \\
        \alert{Ж. вычисл. матем. и матем. физ.} 57.7 (2017), 1205--1229.
    \end{itemize}
\end{frame}


\end{document}
