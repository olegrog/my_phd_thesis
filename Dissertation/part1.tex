\chapter{Кинетическая теория газов} \label{chapt:theory}

\section{Основные понятия и уравнения} \label{sect:fundamentals}

Кинетическая теория основывается на представлении о молекулярном строении вещества.
Газом называется совокупность молекул, находящихся на столь больших расстояниях друг от друга,
что молекулы большую часть времени слабо взаимодействуют друг с другом.
Короткие промежутки времени, в течение которых молекулы сильно взаимодействуют, рассматриваются как \emph{столкновения}.
Если усредненной по времени потенциальной энергией взаимодействия молекул можно пренебречь
по сравнению с их кинетической энергией, то газ называется \emph{идеальным}.
Практически газы из нейтральных молекул при давлениях до сотен атмосфер могут рассматриваться как идеальные.
До этих же давлений вероятность тройных столкновений мала по сравнению с вероятностью двойных (или парных).
В идеальном газе объём, занятый молекулами, мал по сравнению с объёмом, занятым газом.
Другими словами, если \(d_m\) "--- эффективный диаметр молекулы,
\(n_0\) "--- число молекул в единице объёма, то в пределе \(d_m\to0\), \(n_0\to\infty\)
в идеальном газе \(n_0 d_m^3\to0\).
Если при этом конечной остаётся \emph{длина свободного пробега} молекул между столкновениями
\begin{equation}\label{eq:ell}
    \ell = \frac1{\sqrt2\pi d_m^2 n_0},
\end{equation}
то такой предельный континуум принято называть \emph{газом Больцмана}.

\subsection{Границы применимости физической модели}

Предполагается, что движение молекул может быть описано с помощью классической ньютоновской механики.
Квантовые эффекты существенны лишь при очень низких температурах и для легких молекул (водород, гелий, электроны).
Для водорода и гелия квантовые поправки существенны уже при нормальных условиях.
Большинство же газов сжижается при температуре,
при которой ещё нет необходимости применять квантовую теорию столкновения молекул.
Квантовые эффекты необходимо учитывать при неупругих столкновениях атомов и молекул
(возбуждение внутренних степеней свободы молекул, возбуждение электронных уровней и т.\,п.).
Потенциалы упругих взаимодействий молекул также могут быть вычислены лишь с помощью квантовой механики.
Однако при известном потенциале взаимодействия упругие столкновения могут быть рассмотрены классически.

Релятивистские эффекты существенны лишь при очень больших температурах (больших скоростях молекул).
Практически эти эффекты можно не учитывать при температурах порядка десятков и сотен тысяч градусов.
Для водорода, например, средняя скорость молекул при температуре в 105~K равняется 0.0001 скорости света.
Даже скорость электрона при такой температуре составляет тысячные доли скорости света.

Таким образом, рассматриваемая теория идеального газа с учётом парных столкновений в рамках классической механики
удовлетворительно описывает движение газа в широком диапазоне температур и давлений
(для температур от десятков градусов Кельвина до сотен тысяч и для давлений до сотен атмосфер).

% Nordheim1928, Landau1936, Vlasov1938, ZNG, Cerci%Kremer2002
% Bobylev,Illner -- Manev Vlasov--Boltzmann

\subsection{Функция распределения скоростей}

В диссертации повсеместно (если не оговорено иначе) используются безразмерные переменные.
Соответствующие размерные референсные значения содержат верхний индекс \({}^{(0)}\).
Пусть \(x_iL\) (или \(\bx L\)) "--- прямоугольные координаты в физическом пространстве,
а \(\zeta_i\sqrt{2RT^{(0)}}\) (или \(\bzeta\sqrt{2RT^{(0)}}\)) "--- молекулярная скорость.
Здесь \(L\) и \(T^{(0)}\) "--- референсные длина и температура,
\(R = k_B/m\) (\(=1.380658\times 10^{-23} \text{Дж·К}^{-1}\))
"--- удельная газовая постоянная, где \(k_B\) "--- постоянная Больцмана,
\(m\) "--- масса отдельной молекулы.
В шестимерном объёме \(\dx\dzeta\) в момент времени \(t\) находятся
\begin{equation}\label{eq:distrib_function}
    \dd{N} = \frac{\rho^{(0)}L^3}{m} f(\bx,\bzeta,t)\dx\dzeta
\end{equation}
молекул. \(f\) "--- \emph{функция распределения} скоростей молекул в газе.
\(\rho^{(0)}\) "--- референсная плотность, \(f^{(0)} = \rho^{(0)}/(2RT^{(0)})^{3/2}\).

Макроскопические переменные: плотность \(\rho\rho^{(0)}\), скорость \(v_i\sqrt{2RT^{(0)}}\),
температура \(TT^{(0)}\), тензор напряжений \(p_{ij}p^{(0)} = p_{ij}\rho^{(0)}RT^{(0)}\)
и вектор теплового потока \(q_i p^{(0)}\sqrt{2RT^{(0)}}\), "--- определяются
как соответствующие моменты функции распределения:
\begin{gather}\label{eq:macro_variables}
    \rho = \int f(\bx,\bzeta,t)\dzeta, \\
    \rho v_i = \int \zeta_i f(\bx,\bzeta,t)\dzeta, \\
    \rho T = \frac23 \int (\zeta_i-v_i)^2 f(\bx,\bzeta,t)\dzeta, \\
    p_{ij} = 2\int (\zeta_i-v_i)(\zeta_j-v_j) f(\bx,\bzeta,t)\dzeta, \\
    q_i = \int (\zeta_i-v_i)(\zeta_j-v_j)^2 f(\bx,\bzeta,t)\dzeta.
\end{gather}
Трехмерное интегрирование по \(\bzeta\) здесь и далее проводится во всём пространстве \(\bzeta\).
Безразмерное давление \(p = p_{ii}/3 = \rho T\).

\subsection{Уравнение Больцмана}

Поведение функции распределения \(f\) определяется \emph{уравнением Больцмана}
\begin{equation}\label{eq:Boltzmann}
    \pder[f]{t} + \zeta_i\pder[f]{x_i} + F_i\pder[f]{\zeta_i} = \frac1k J(f,f),
\end{equation}
где билинейный оператор \(J(f,g)\) выражается как
\begin{equation}\label{eq:integral}
    J(f,g) = \frac12 \int(f'g'_*+f'_*g'-fg_*-f_*g) B\left(\frac{|\alpha_j V_j|}{V},V\right)
        \dd\Omega(\boldsymbol\alpha)\dzeta_*,
\end{equation}
\begin{equation*}
    \begin{array}{l l}
        f = f(\bx,\bzeta,t), & f_* = f(\bx,\bzeta_*,t), \\
        f' = f(\bx,\bzeta',t), & f'_* = f(\bx,\bzeta'_*,t), \\
        \zeta_i' = \zeta_i+\alpha_i\alpha_j V_j, & \zeta_{i*}' = \zeta_{i*}-\alpha_i\alpha_j V_j, \\
        V_i = \zeta_{i*}-\zeta_i, & V = \sqrt{V_i^2} = |V_i|.
    \end{array}
\end{equation*}
Предполагается, что внешняя сила \(2F_i k_B T^{(0)}/L\), действующая на молекулу,
не зависит от молекулярной скорости \(\bzeta\),
\(\boldsymbol\alpha\) "--- единичный вектор,
выражающий изменение направления молекулярной скорости из-за столкновения молекул,
\(\Omega(\boldsymbol\alpha)\) "--- элемент телесного угла в направлении~\(\boldsymbol\alpha\),
\emph{столкновительное ядро} \(B(|\alpha_i V_i|/V,V)\) "--- неотрицательная функция,
определяемая межмолекулярным потенциалом. В частности,
\begin{equation}\label{eq:ci_kernel}
    B = \frac{|\alpha_i V_i|}{4\sqrt{2\pi}}
\end{equation}
для газа, состоящего из твёрдых сфер.\footnote{
    В настоящей работе другие потенциалы не рассматриваются,
    хотя большинство формулировок приводятся для общего случая.
}
Интегрирование в~\eqref{eq:integral} проводятся по всему пространству \(\zeta_{i*}\) и
по всем направлениям \(\alpha_i\) (всей сферической поверхности) соответственно.
Интеграл \(J(f,f)\) называется \emph{интегралом столкновения}
или \emph{столкновительным членом} уравнения Больцмана.
Модифицированное число Кнудсена \(k\) вычисляется как
\begin{equation}\label{eq:modified_kn}
    k = \frac{\sqrt\pi}2\Kn = \frac{\sqrt\pi}2 \frac{\ell^{(0)}}L = \frac{m}{2\sqrt{2\pi} d_m^2 \rho^{(0)}L}.
\end{equation}
В безразмерном виде~\eqref{eq:Boltzmann} неявно учтено, что число Струхаля равно единице,
т.\,е. референсный отрезок времени \(t^{(0)} = L/\sqrt{2RT^{(0)}}\).

\subsubsection{Классификация столкновительных ядер}

В основании молекулярной газовой динамики лежит кинетическая теория уравнения Больцмана
\begin{gather*}
    \pder[f]{t} + \zeta_i\pder[f]{x_i} = J(f,f), \\
    J(f,f) \eqdef \int_{\mathbb{R}^d}\dzeta_*\int_{\mathbb{S}^{d-1}}\boldsymbol{\dd\sigma}
    B\left(|\bzeta - \bzeta_*|, \frac{(\zeta_i - \zeta_{i*})\sigma_i}{|\bzeta - \bzeta_*|}\right) \left(f'f'_* - ff_*\right),
\end{gather*}
решение которого сильно зависит от больцмановского \emph{столкновительного ядра} \(B(|\bzeta - \bzeta_*|, \cos\theta)\).
В упрощённом виде его можно записать как
\begin{equation*}
    B = \left|\bzeta - \bzeta_*\right|^\gamma b(\cos\theta), \quad b(\cos\theta) \underset{\theta\to0+}{\sim} \theta^{-2-\nu},
\end{equation*}
где \(\theta\) "--- угол отклонения частиц, сталкивающихся со скоростями \(\bzeta\) и \(\bzeta_*\).
Степенному потенциалу межмолекулярного взаимодействия \( U(r)=r^{1-s}\:(s>2) \) соответствуют параметры
\begin{equation*}
    \gamma = \frac{s-5}{s-1}\in(-3,1), \quad \nu = \frac{2}{s-1}\in(0,2).
\end{equation*}
В зависимости из значения \(\gamma\) содержательна следующая классификация:
\begin{itemize}\label{eq:power_classification}
    \item модель твёрдых сфер \(\gamma=1\),
    \item жёсткие потенциалы \(\gamma\in(0,1)\),
    \item максвелловский потенциал \(\gamma=0\),
    \item мягкие потенциалы \(\gamma\in(-3,0)\),
    \item кулоновский потенциал \(\gamma=-3\).
\end{itemize}
В отличие от модели твёрдых сфер все степенные потенциалы порождают неинтегрируемую \emph{угловую сингулярность}
вследствие слишком большого числа \emph{скользящих} столкновений (при малых \(\theta\)).
Рассмотрение таких сингулярных потенциалов, называемых \emph{дальнодействующими},
требует существенно более продвинутого математического аппарата.
Многие результаты оказываются проще для искусственных \emph{короткодействующих} потенциалов~\cite{Grad1970}:
\begin{equation*}\label{eq:angular_cutoff}
    \int_0^\pi B(|\bzeta - \bzeta_*|, \cos\theta)\sin\theta\dd\theta < +\infty.
\end{equation*}
Для кулоновского потенциала вклад скользящих столкновений становится решающим,
поэтому уравнение Больцмана в пределе \(\nu\to2\) сводится к уравнению Ландау~\cite{Arseniev1990}.

%%% модельные потенциалы
% VHS~\cite{Bird1981vhs} VSS~\cite{Koura1991}

\subsubsection{Вывод на основе законов механики}

Уравнение Больцмана может быть получено в предельном переходе из уравнения Лиувилля
при некоторых предположениях. Детальное изложение такого перехода можно найти в~\cite{Grad1970, Sone2007}
Вопрос существования и единственности решения уравнения Больцмана в общем случае до настоящего времени открыт,
интересующийся читатель может обратится к~\cite{Cercignani1994a, Villani2002, Strain2011}.

%%% Вывод из уравнения Лиувилля
Обоснование своего кинетического уравнения, исходя из законов механики, было дано ещё Больцманом в эвристической форме.
Гильберт в своей знаменитой шестой проблеме поставил задачу строгого математического обоснования
процессов предельного перехода от атомистического понимания к моделям сплошной среды.
Такой переход может быть выполнен через промежуточный \emph{мезоскопический} уровень описания
на основе одночастичной функции распределения.
Х.~Грэду принадлежит первая математическая формулировка вывода уравения Больцмана из уравнения Лиувилля~\cite{Grad1949}.
Единственный строгий результат в общей постановке принадлежит О.~Лэнфорду~\cite{Lanford1975}.
Ему удалось показать, что на коротком промежутке времени (порядка времени среднего пробега)
цепочка уравнений ББГКИ для газа твёрдых сфер сходится в пределе Грэда"--~Больцмана почти везде к кинетическому уравнению.
Глобальная по времени сходимость известна только для очень частного случая распространения газа в вакуум~\cite{Illner1989}
или для малых флуктуаций отдельных частиц~\cite{Beijeren1980}.
В.\,И.~Герасименко и академик Д.\,Я.~Петрина предоставили первые количественные оценки на множество патологических
траекторий в оригинальном доказательстве Лэнфорда~\cite{Petrina1990}.
С.~Юкай упростил доказательство, используя теорему Коши"--~Ковалевской~\cite{Ukai2001}.
Обобщение теоремы Лэнфорда для короткодействующих потенцилов потребовал существенно более деликатного
анализа событий асимптотически нулевой меры~\cite{Raymond2013, Pulvirenti2014}.
Недавно французские математики Т.~Бодино, И.~Г\'{а}ллахер и Л.~Сэн-Ремон получили количественные оценки для флуктуационного режима,
что позволило обосновать линейное уравнение Больцмана на временах порядка \(\ln\ln{N}\)~\cite{Raymond2016}
и линеаризованное при \(d=2\) на временах порядках \(\sqrt[4]{\ln\ln{N}}\)~\cite{Raymond2017},
где \(N\) "--- число частиц, заключённых в \(d\)-мерном торе.
На основе их работ был достигнут первый успех для дальнодействующих потенциалов~\cite{Ayi2017}.

% основные предположения

\subsubsection{Симметрии и законы сохранения}

Для произвольной функции \(\varphi(\bzeta)\) интеграл столкновения~\eqref{eq:integral}
удовлетворяет соотношению симметрии
\begin{equation}\label{eq:ci_symmetry}
    \int\varphi(\bzeta)J(f,g)\dzeta = \frac14\int(\varphi+\varphi_*-\varphi'-\varphi'_*) J(f,g)\dzeta.
\end{equation}
Поскольку \(\zeta_i+\zeta_{i*}=\zeta'_i+\zeta'_{i*}\) и \(\zeta_i^2+\zeta_{i*}^2={\zeta'_i}^2+{\zeta'_{i*}}^{\!\!2}\),
то интеграл столкновений обладает несколькими инвариантами \(\psi_r\) (\(r=0, 1, 2, 3, 4\)):
\begin{equation}\label{eq:conser_moments}
    \int \psi_r J(f,g)\dzeta = 0,
\end{equation}
\begin{equation}\label{eq:def_psi}
    \psi_0 = 1, \quad \psi_i = \zeta_i, \quad \psi_4 = \zeta_i^2.
\end{equation}
Другими словами, в результате столкновений сохраняются масса, импульс и энергия.
Умножая уравнение Больцмана~\eqref{eq:Boltzmann} на \(\psi_r\)
и интегрируя результат по всему пространству \(\bzeta\), получим \emph{уравнения сохранения}:
\begin{gather}\label{eq:conservation}
    \pder[\rho]{t} + \pder{x_i}(\rho v_i) = 0, \\
    \pder{t}(\rho v_i) + \pder{x_j}\left( \rho v_i v_j + \frac{p_{ij}}2 \right) = \rho F_i, \\
    \pder{t}\left[ \rho\left( v_i^2 + \frac32T \right) \right]
        + \pder{x_j}\left[ \rho v_j\left( v_i^2 + \frac32T \right) + v_i p_{ij} + q_j \right]
        = 2\rho v_j F_j, \label{eq:conservation1}
\end{gather}
В классической газовой динамике уравнения сохранения~\eqref{eq:conservation}--\eqref{eq:conservation1}
замыкаются соответствующими феномелогическими выражениями для \(p_{ij}\) и \(q_i\). Например,
\begin{equation}\label{eq:Euler}
    p_{ij} = p\delta_{ij}, \quad q_i = 0,
\end{equation}
или
\begin{equation}\label{eq:Navier-Stokes}
    p_{ij} = p\delta_{ij} - \Gamma_1(T)\left( \pder[v_i]{x_j} + \pder[v_j]{x_i}
        - \frac23\pder[v_k]{x_k}\delta_{ij} \right)k, \quad
    q_i = -\frac54\Gamma_2(T)\pder[T]{x_i}k,
\end{equation}
где \(\Gamma_1(T)\) и \(\Gamma_2(T)\) "--- безразмерные коэффициенты \emph{вязкости}
и \emph{теплопроводности} газа соответственно, зависящие от температуры.
Множество уравнений сохранения~\eqref{eq:conservation}--\eqref{eq:conservation1}
с тензором напряжений и вектором теплового потока, выраженными как~\eqref{eq:Euler},
называются \emph{уравнениями Эйлера}. Выражения~\eqref{eq:Navier-Stokes},
называемые законами Ньютона и Фурье соответственно,
приводят к \emph{уравнениям Навье"--~Стокса}.

\subsubsection{Равновесное состояние}

Определим так называемый \(H\)-функционал от функции распределения \(f\) как
\begin{equation}\label{eq:H_function}
    H(f) = -\int f \ln f \dzeta \dx.
\end{equation}
Тогда \emph{\(H\)-теорема} Больцмана гласит, что
\begin{equation}\label{eq:H_theorem}
    \der[H]{t} = \int D(f) \dx \geq 0,
\end{equation}
причём равенство выполняется для \emph{распределения Максвелла}:
\begin{equation}\label{eq:Maxwell_distribution}
    f_e = \frac{\rho}{(\pi T)^{3/2}} \exp \left( -\frac{(\zeta_i-v_i)^2}{T} \right),
\end{equation}
для которого \(J(f_e,f_e) = 0\).
\(H\)-теорема отражает второе начало термодинамики.
Согласно \emph{гипотезе Черчиньяни}~\cite{Cercignani1982, Desvillettes2011} неотрицательный функционал
производства энтропии, определённый как
\begin{equation}\label{eq:D_function}
    D(f) = -\int \ln{f} J(f,f) \dzeta
\end{equation}
убывает во времени экспоненциально. Гипотеза верна для достаточно гладкой \(f\)
и реального физического межмолекулярного потенциала~\cite{Villani2003}.

\subsubsection{Линеаризация}

Рассмотрим стационарный газ (\(\Pder{t}=0\)) в отсутствие внешних сил (\(F_i=0\))
с функцией распределения близкой к референсной \(f_0\):
\begin{equation}\label{eq:linear_condition}
    \phi = \frac{f}{E} - 1 \ll 1, \quad E(\zeta) = \frac1{\pi^{3/2}}\exp\left(-\zeta^2\right).
\end{equation}
В таком случае справедливо \emph{линеаризованное уравнение Больцмана}
\begin{equation}\label{eq:linear_Boltzmann}
    \zeta_i \pder[\phi]{x_i} = \frac1k \mathcal{L}(\phi),
\end{equation}
с \emph{линеаризованным интегралом столкновения}
\begin{equation}\label{eq:linear_ci}
    \mathcal{L}(\phi) = \int E_*(\phi'+\phi'_*-\phi-\phi_*) B
    \dd \Omega(\boldsymbol{\alpha}) \boldsymbol{\dd \zeta_*}.
\end{equation}
Возмущённое локальное максвелловское распределение
\begin{equation}\label{eq:linear_maxwellian}
    \phi_e = \omega + 2\zeta_i v_i + \left(\zeta_i^2-\frac32\right)\tau
\end{equation}
удобно выражается через возмущённые макроскопические переменные:
\begin{equation}\label{eq:linear_macro}
    \omega = \rho-1, \quad \tau = T-1, \quad P = p-1, \quad P_{ij} = p_{ij} - \delta_{ij},
\end{equation}
которые вычисляются как моменты \(\phi\):
\begin{gather}\label{eq:linear_macro_variables}
    \begin{gathered}
        \omega = \int \phi E\dzeta, \quad
        v_i = \int \zeta_i \phi E\dzeta, \quad
        \tau = \frac23 \int \left( \zeta_i-\frac32 \right)^2 \phi E\dzeta, \\
        P = \tau + \omega, \quad
        P_{ij} = 2\int \zeta_i\zeta_j\phi E\dzeta, \quad
        q_i = \int \zeta_i\zeta_j^2\phi E\dzeta - \frac52v_i
    \end{gathered}
\end{gather}

Линейный оператор \(\mathcal{L}(\phi)\), также как и \(J(f,g)\), удовлетворяет соотношению симметрии
\begin{equation}\label{eq:linear_symmetry}
    \int\psi(\bzeta)\mathcal{L}(\varphi)E\dzeta = \frac14\int EE_* (\psi+\psi_*-\psi'-\psi'_*)
        (\varphi'+\varphi'_*-\varphi-\varphi_*) B\dd\Omega\dzeta_*\dzeta.
\end{equation}
С помощью~\eqref{eq:linear_symmetry} несложно показать, что в гильбертовом пространстве
со скалярным произведением \( \inner{\varphi}{\psi} = \int\varphi\psi E\dzeta\)
оператор \(\mathcal{L}(\phi)\) самосопряжён и неположителен:
\begin{equation}\label{eq:linear_properties}
    \inner{\varphi}{\mathcal{L}\psi} = \inner{\mathcal{L}\varphi}{\psi}, \quad
    \inner{\varphi}{\mathcal{L}\varphi} \leq 0,
\end{equation}
причём равенство выполняется только для инвариантов столкновения \(\psi_r\),
которые являются собственными векторами для собственного значения \(\lambda = 0\).
Подробное описание математических свойств линеаризованного оператора можно найти в~\cite{Cercignani2006}.

%%% спектральная щель, классификация спектров

\subsection{Граничные условия}

%%% какие модели
%    Экспериментальные данные свидетельствуют о том, что для негиперзвуковых течений такая модель
%    даёт удовлетворительные результаты при коэффициенте аккомодации около \(0.9\)~\cite{Ewart2007}.

Для описания взаимодействия газа с твёрдой непроницаемой границей обычно прибегают
к граничному условию \emph{диффузного отражения} (максвелловское условие с полной аккомодацией):\footnote{
    Экспериментальные данные свидетельствуют, что в большинстве случаев коэффициент аккомодации близок к единице~\cite{Agrawal2008}.
}
\begin{equation}\label{eq:diffuse_bc}
    \begin{gathered}
        f(x_i, \zeta_i, t) = \frac{\sigma_B}{(\pi T_B)^{3/2}}
            \exp\left( -\frac{(\zeta_i-v_{Bi})^2}{T_B} \right) \quad \left[(\zeta_j-v_{Bj}) n_j\right] > 0, \\
        \sigma_B = -2\sqrt\frac{\pi}{T_B}
            \int_{(\zeta_j-v_{Bj})n_j < 0} (\zeta_j-v_{Bj}) n_j f(x_i, \zeta_i, t) \dzeta,
    \end{gathered}
\end{equation}
где \(n_i\) "--- единичный вектор нормали, направленный внутрь области, занимаемой газом.
Для практических расчётов часто прибегают также к модели Черчиньяни"--~Лампис~\cite{Cercignani1971, Cercignani2006}.

Граничные условия диффузного отражения~\eqref{eq:diffuse_bc} линеаризуются следующим образом:
\begin{equation}\label{eq:linear_diffuse_bc}
    \begin{gathered}
        \varphi(x_i, \zeta_i) = \check\sigma_B + 2\zeta_j v_{Bj} + \left(\zeta_i-\frac32\right)^2\tau_B,
            \quad (\zeta_j-v_{Bj}) n_j > 0, \\
        \check\sigma_B = \sqrt\pi v_{Bj}n_j - \frac{\tau_B}2 -2\sqrt\pi
            \int_{(\zeta_j-v_{Bj})n_j < 0} \zeta_j n_j \varphi E\dzeta.
    \end{gathered}
\end{equation}

\subsection{Модельные уравнения}

Для анализа течений разреженного газа зачастую прибегают к упрощению
столкновительного члена~\eqref{eq:integral}, в частности, широко используется
модельное уравнение БКВ\footnote{
    Более распространённой считается аббревиатура БГК (Бхатнагар"--~Гросс"--~Крук),
    образованная по фамилиям авторов только одной работы~\cite{Krook1954},
    однако она не отражает истинный вклад и заслуги первооткрывателей.
    Аббревиатура БКВ (Бобылев"--~Крук"--~Ву) также используется в качестве названия одного из
    точных решений уравнений Больцмана для задачи пространственно однородной релаксации
    максвелловского газа~\cite{Bobylev1975, Krook1976},
    однако это не должно вызывать путаницы в контексте обсуждения.
} (Больцмана"--~Крука"--~Веландера)~\cite{Krook1954, Welander1954}:
\begin{equation}\label{eq:BKW}
    \pder[f]{t} + \zeta_i\pder[f]{x_i} = \frac{\rho}{k} (f_e-f).
\end{equation}
Линеазированный столкновительный член в~\eqref{eq:BKW} имеет вид
\begin{equation}\label{eq:BKW_integral}
    \mathcal{L}(\varphi) = -\varphi + \omega + 2\zeta_i v_i + \left( \zeta_i^2 - \frac32 \right)\tau.
\end{equation}
Основной недостаток модели БКВ "--- это фиксированное число Прандля \(\Pr = \Gamma_1(T)/\Gamma_2(T) = 1\).
Этого недостатка лишены ES-модель~\cite{Holway1966} со столкновительным членом
\begin{equation}\label{eq:ES_model}
    J(f,f) = \rho \left[\frac{\rho\sqrt{\det A_{ij}}}{(\pi T)^{3/2}}
        \exp\left(- \frac{A_{ij}c_i c_j}T \right) -f \right], \quad
    A_{ij}^{-1} = \frac{\delta_{ij}}{\Pr} - \frac{1-\Pr}{\Pr}\frac{p_{ij}}{p}
\end{equation}
и S-модель~\cite{Shakhov1968} со столкновительным членом
\begin{equation}\label{eq:S_model}
    J(f,f) = \rho \left[ f_e \left( 1+\frac{1-\Pr}{5}\frac{q_i c_i}{pT}
        \left(2c_i^2-5\right) \right)-f\right],
\end{equation}
где использовано сокращение \(c_i = \zeta_i-v_i\).

Для описанных модельных уравнений также справедливы уравнения сохранения и \(H\)-теорема,
однако решение, полученное с их помощью, не может считаться приближённым решением
уравнения Больцмана, поскольку нет возможности оценить степень подобных приближений.
Однако для некоторых простых задач получаемые результаты различаются не так значительно.

%\newpage
%============================================================================================================================
\section{Математическая теория задачи Коши} \label{sect:cauchy}

Математическая теория
% опишем однородное + неоднородное в R3 и T3, опустим граничные условия (объём кандидатской диссертации не позволяет)

\subsection{Пространственно-однородная задача}

%%% пространственно-однородная задача Коши
Методы численного анализа и асимптотическая теория уравнения Больцмана тесно связаны
с математической теорией задачи Коши.
Пионерские работы, посвящённые \(L^1\)-теории в пространственно однородной постановке,
принадлежат Т.~Карлеману~\autocite{Carleman1933}, Д.~Моргенштерну~\autocite{Morgenstern1954},
А.\,Я.~Повзнеру~\autocite{Povzner1962} и Л.~Аркерюду~\autocite{Arkeryd1972}.
Теория Фурье"=преобразования уравнения Больцмана Обширная программа анализа максвелловского газа на основе
была реализована А.\,В.~Бобылевым~\autocite{Bobylev1984}.
Математические инструменты для анализа сингулярных больцмановских ядер,
характерных для дальнодействующих потенциалов, появились на рубеже веков как результат труда
П.-Л.~Лионса, Р.~Александре, Л.~Девиллета, Б.~Веннберга и С.~Виллани~\autocite{Lions1989, Alexandre2000}.


Задача Коши для пространственно"=однородного уравнения Больцмана наиболее изучена.
Она имеет особую важность для численных методов, поскольку в большинстве из них используется
расщепление на транспортный и столкновительный операторы.
Кроме того, пространственная однородность устойчива в том смысле, что
слабая неоднородность распространяется во времени~\cite{Arkeryd1987}.
На сегодняшний день известно только одно семейство явных нестационарных решений,
найденное А.\,В.~Бобылевым~\cite{Bobylev1975} и независимо М.~Круком, Т.~Ву~\cite{Krook1976} для максвелловского газа.

%%% Немягкие короткодействующие потенциалы (моменты+сущ+един)
Количественные оценки для полиномиальных моментов
\begin{equation*}
    \|f(t)\|_{L^1_s} \eqdef \int_{\mathbb{R}^d} |\bzeta|^s f(t,\bzeta)\dzeta \quad (s>2)
\end{equation*}
являются базовым инструментом в пространственно"=однородной теории
и отражают поведение функции распределения для больших молекулярных скоростей.
Равномерная ограниченность полиномиальных моментов позволяет сразу же доказать
существование и единственность решения, а также H-теорему Больцмана.
Первые такие результаты принадлежат Т.~Карлеману~\cite{Carleman1933} и А.\,Я.~Повзнеру~\cite{Povzner1962}
для газа твёрдых сфер, Д.~Моргенштерну~\cite{Morgenstern1954} для макселловского газа.
Общую \(L^1\)-теорию для класса жёстких короткодействующих потенциалов развил Л.~Аркерюд в 1971 году~\cite{Arkeryd1972}.
С.~Мишлер и Б.~Веннберг максимально ослабили начальные предположения в задаче Коши,
предоставив доказательство в предположении лишь конечности массы и энергии~\cite{Mischler1999}.
В общем случае энергия неубывает со временем, однако единственность в \(L^1\) достигается
только в классе решений с постоянной энергией~\cite{Wennberg1999}.

%%% Немягкие короткодействующие потенциалы (другие свойства)
Множество других важных результатов получено для немягких короткодействующих потенциалов:
\begin{itemize}
    \item существование максвелловской нижней границы~\cite{Pulvirenti1996, Pulvirenti1997},
    \item распространение гладкости и экcпоненциальное убывание разрывов~\cite{Carlen1999, Mouhot2004},
    \item распространение максвелловской верхней границы~\cite{Gamba2009upper, Bobylev2017}.
\end{itemize}
Столкновительное ядро имеет наиболее простой вид \(B = B(\cos\theta)\) для максвелловского потенциала,
поэтому математические результаты для него, как правило, предшествуют исследованиям других потенциалов,
однако в отличие от жёстких потенциалов полиномиальные моменты максвелловского газа равномерно ограничены лишь
при условии их ограниченности в начальный момент времени.

%%% Мягкие потенциалы
Меньше известно о поведении газа с мягкими потенциалами.
Прежде всего нет доказательства равномерной ограниченности полиномиальных моментов,
есть только оценка \(\|f(t)\|_{L^1_s} < C(1+t)\) при \(\|f(0)\|_{L^1_s} < \infty\)~\cite{Carlen2009}.
При \(\gamma<0\) столкновительное ядро обретает дополнительную \emph{кинетическую сингулярность} в точке \(\bzeta = \bzeta_*\),
поэтому смысл в \(L^1\) имеют только слабые формы столкновительного интеграла.
Его симметричные свойства (законы сохранения) позволяют построить решения при \(\gamma\geq-2\) и \(\nu<2\)~\cite{Arkeryd1981, Goudon1997}.
Используя конечность производства энтропии, удаётся регуляризовать столкновительный оператор при \(\gamma>-4\)~\cite{Villani1998}.

\subsubsection{Сходимость к равновесию}

%%% Сходимость к равновесию в L^2(M^-1)
Долгую историю имеет проблема сходимости решения к равновесному максвелловскому распределению \(M\).
Она тесно связана со спектральными свойствами линеаризованного столкновительного оператора.
Как известно, он имеет пятикратное нулевое собственное значение, остальные отрицательные.
Наличие спектральной щели (конечное расстояние между максимальным отрицательным собственным значением и нулевым)
для немягких потенциалов ведёт к экспоненциальному затуханию \(\OO{e^{-\lambda t}}\) возмущённых решений~\cite{Grad1963b}.
Напротив, отсутствие спектральной щели у мягких потенциалов позволяет рассчитывать только на оценку \(\OO{e^{-\lambda t^\alpha}}\),
где \(\alpha\in(0,1)\)~\cite{Caflisch1980a}.

%%% Неконструктивная сходимость к равновесию в L^1
Теория сходимости к равновесию в \(L^1\) также берёт начало с работы Л.~Аркерюда 1988 года~\cite{Arkeryd1988},
в которой была показана экспоненциальная скорость для твёрдых потенциалов.
Однако этот результат был получен неконструктивными методами.
С физической точки зрения крайне важно иметь явные оценки сходимости, поскольку на очень большом временном промежутке
уравнение Больцмана теряет смысл (парадокс Церм\'{е}ло).

%%% Сходимость в L^1 чисто нелинейными методами
\(\mathcal{H}\)-функционал (энтропия со знаком минус) выполняет роль функционала Ляпунова для уравнения Больцмана,
поэтому в соответствии с принципом Красовского"--~Ласаля производство энтропии для уравнения Больцмана
"--- основной инструмент для контроля сходимости к равновесию в нелинейной постановке.
Более того, согласно неравенству Чисара"--~Кульбака"--~Пинскера больцмановская энтропия
гарантирует сходимость непосредственно в \(L^1\).
В связи с этим К.~Черчиньяни в 1982 году предположил, что производство энтропии связано
линейным неравенством с самой энтропией~\cite{Cercignani1982, Desvillettes2011},
однако позже он вместе с А.\,В.~Бобылевым построили контрпример для \(\gamma\in(0,2)\)~\cite{Bobylev1999}.
Дж.~Тоскани и С.~Виллани получили оптимальный результат в виде~\cite{Toscani1999, Toscani2000, Villani2003}
\begin{equation*}
    \mathcal{D}(f) \geq \lambda_\varepsilon \left[ \mathcal{H}(f) - \mathcal{H}(M) \right]^{1+\varepsilon},
    \quad \varepsilon > 0,
\end{equation*}
где производство энтропии \(\mathcal{D}(f) \eqdef -\int J(f,f) \ln f \dzeta\) почти линейным неравенством связано
с функционалом \(\mathcal{H}(f) \eqdef \int f\ln f \dzeta\).
Такой результат обеспечивает полиномиальную сходимость \(\OO{t^{-\infty}}\).
Гипотеза Черчиньяни оказалась верна только для нефизичного случая \(\gamma=2\)~\cite{Villani2003}.

%%% Сходимость в L^1 для больцмановской полугруппы
Приведённые явные оценки на темп производства энтропии справедливы на чисто функциональном уровне,
поэтому они могут быть улучшены для полугруппы, порождаемой уравнением Больцмана.
В 1997 году Э.~Карлен, Э.~Габетта и Дж.~Тоскани получили оптимальный результат для максвелловского газа~\cite{Carlen1999}
\begin{equation*}
    \|f(t) - M\|_{L^1} \leq C_\varepsilon e^{-(1-\varepsilon)\lambda_g t}, \quad \varepsilon > 0,
\end{equation*}
где \(\lambda_g\) "--- ширина спектральной щели, а \(\varepsilon\) тем меньше,
чем больше полиномиальных моментов ограничено.
Если же в начальном условии ограничена лишь энергия, то существуют сколь угодно медленно сходящиеся к равновесию
решения для нежёстких потенциалов~\cite{Carlen2003, Carlen2009}.
Для жёстких потенциалов К.~Муо в 2005 году разработал инструменты расширения функционального пространства для линеаризованной полугруппы,
позволяющие соединить спектральные результаты в \(L^2(M^{-1})\) с нелинейной \(L^1\)-теорией~\cite{Mouhot2006}.
Таким образом, явные оценки на ширину спектральной щели~\cite{Baranger2005}
обеспечили оптимальную сходимость \(\OO{e^{-\lambda_g t}}\).

\subsubsection{Дальнодействующие потенциалы}

%%% Дальнодействующие потенциалы в XX веке
В XX веке дальнодействующим потенциалам было посвящено считанное количество работ.
Ё.-П.~Пао ещё в 1974 году с помощью теории псевдо"=дифференциальных операторов показал,
что спектр линеаризованного уравнения полностью дискретен~\cite{Pao1974}.
Давно было известно, что сингулярные операторы способны повышать гладкость решения,
однако соответствующие результаты для уравнения Больцмана долгое время были недоступны
из-за высокой технической сложности.
В 1994 году Л.~Девиллет смог доказать, что решение модельного уравнения Каца лежит
в пространстве Соболева при \(t>0\)~\cite{Desvillettes1995}.
В 1997 году П.-Л.~Лионс впервые установил функциональную связь между производством энтропии
и гладкостью функции распределения~\cite{Lions1998}.
В 1999 году Р.~Александре, Л.~Девиллет, Б.~Веннберг и С.~Виллани довели этот результат до оптимальной
в некотором смысле оценки~\cite{Alexandre2000}
\begin{equation*}
    \|\sqrt{f}\|_{H^{\nu/2}(|\bzeta|<R)} \leq C_R \left[ \mathcal{D}(f) + \|f\|^2_{L_2^1} \right], \quad R > 0.
\end{equation*}
Другими словами, сингулярный больцмановский столкновительный оператор ведёт себя
как дробный лапласиан \(-(-\Delta)^{\nu/2}\).
С физической точки зрения это означает, что процесс межмолекулярного взаимодействия в больцмановском пределе
является диффузно"=столкновительным.
Важную роль в доказательстве этого функционального неравенства играет преобразование Фурье,
приложение которого к уравнению Больцмана было систематически изучено А.\,В.~Бобылевым~\cite{Bobylev1984}.

%%% Дальнодействующие потенциалы в XXI веке
Единственность решения в глобальном смысле \(t\in[0,+\infty]\) в соболевском пространстве установлена
для максвелловского газа~\cite{Toscani1999maxw} и жёстких потенциалов в случае
умеренной угловой сингулярности \(\nu\in(0,1)\)~\cite{Mouhot2009} и сильной \(\nu\in[1,2)\)~\cite{He2012}.
При условии гладкости \(b(\cos\theta)\) оказывается, что функция распределения лежит в пространстве Шварца,
пока её полиномиальные моменты ограничены~\cite{Desvillettes2005, Alexandre2012}.
Более того, для максвелловского газа известно, что регулярность по Жевре распространяется~\cite{Desvillettes2009}.
Экспоненциальная сходимость с явными оценками в \(L^1\) для дальнодействующих немягких потенциалов
была доказана совсем недавно~\cite{Tristani2014, Meng2017}.
Сходимость для мягких потенциалов известна лишь в усреднённом смысле~\cite{Carlen2009}.

\subsection{Пространственно-неоднородная задача}

%%% пространственно-неоднородная задача Коши
Для полного уравнения Больцмана первая и единственная на сегодняшний день \(L^1\)-теория
построена на основе ренормализованных решений Р.~ди-Перна и П.-Л.~Лионсом~\autocite{Lions1989}.
Почти экспоненциальная сходимость к равновесию \(\OO{t^{-\infty}}\) с явными оценками
для любого достаточно гладкого решения была доказана Л.~Девиллетом и С.~Виллани~\autocite{Villani2005}.
Оба этих результата отмечены Филдсовскими премиями 1994 и 2010 годов.
Строгая линеаризованная теория была построена Х.~Грэдом~\autocite{Grad1963b}.
Вопрос асимптотической устойчивости глобального максвелловского распределения
получил первый положительный ответ для короткодействующих жёстких потенциалов ещё в 1974 году~\autocite{Ukai1974},
но был окончательно закрыт в недавних работах Ф.~Грессмана, Р.~Стрейна~\autocite{Strain2011} и независимо
Р.~Александре, Ё.~Моримото, С.~Юкая, Ч.-Ц.~Сюя, Т.~Янга~\autocite{Alexandre2012soft}.
Т.-П.~Лю и Ш.-С.~Ю внесли важный вклад в понимании больцмановской динамики
на основе анализа функции Грина линеаризованного уравнения Больцмана~\autocite{Liu2004green, Liu2006}.

%%% асимптотическая устойчивость
Пространственно"=неоднородная задача принципиально сложнее для анализа.
В настоящее время полностью доказана асимптотическая устойчивость максвелловского распределения
в классе достаточно гладких решений.
Первые результаты в рамках нелинейной теории возмущения принадлежат Х.~Грэду, который в 1964 году показал
существование, единственность возмущённого решения, стремление его к термодинамическому равновесию,
однако для локального интервала времени~\cite{Grad1965}.
В 1974 году С.~Юкай представил глобальный результат в ограниченной области
для жёстких короткодействующих потенциалов~\cite{Ukai1974}.
Р.~Кафлиш обобщил его для \(\gamma>-1\)~\cite{Caflisch1980b}.
Спектральный анализ не позволяет продвинуться дальше границы \(\gamma=-2\).
Я.~Го предложил альтернативный метод, который позволил рассмотреть очень мягкие потенциалы~\cite{Guo2003},
Экспоненциальная сходимость в ограниченной области для них была доказана позже совместно с Р.~Стрейном~\cite{Strain2008}.
В 2010 году Ф.~Грессман и Р.~Стрейн, используя нетривиальные анизотропные cоболевские нормы~\cite{Mouhot2007} для метода Го,
показали асимптотическую устойчивость для всех дальнодействующих потенциалов~\cite{Strain2011}.
Оказалось, что в присутствии угловой сингулярности \(\nu>0\) спектральная щель и
соответствующая ей экспоненциальная сходимость имеют место при \(\gamma+\nu\geq0\).
Такой же результат, но для неограниченной области был независимо получен
гонконгской группой~\cite{Alexandre2012soft, Alexandre2011hard, Alexandre2011properties}.
Сходимость в \(\mathbb{R}^d\) алгебраическая. Для \(\gamma>-1\) С.~Юкай и К.~Асано доказали это ещё в 1982 году~\cite{Ukai1982}.
Р.~Стрейн получил оптимальную оценку \(\OO{t^{-\frac{n}2+\frac{n}{2r}}}\) в \(L^r_\bx L^2_\bzeta\) при \(r\in[2,\infty]\)~\cite{Strain2012}.

%%% теория ди-Перна"--~Лионса
Единственная на сегодняшний день теория, описывающая решения полного уравнения Больцмана без
дополнительных предположений об их малости, принадлежит Р.~ди~Перна и П.-Л.~Лионсу~\cite{Lions1989}
и отмечена Филдсовской премией 1994 году.
Им удалось доказать существование слабого решения в \emph{ренормализованной} форме
\begin{equation*}
    \pder[\beta(f)]{t} + \zeta_i\pder[\beta(f)]{x_i} = \beta'(f)J(f,f),
\end{equation*}
где \(\beta'(f) \leq C/(1+f)\).
Позже П.-Л.~Лионс упростил доказательство, используя теорию интегральных операторов Фурье~\cite{Lions1994},
а Р.~Александре и С.~Виллани обобщили его для дальнодействующих потенциалов~\cite{Alexandre2002}.
Основная идея ренормализации "--- получить в правой части сублинейный оператор вместо квадратичного \(J(f,f)\).
В такой постановке априорных оценок для массы, энергии и энтропии оказывается достаточно
для построения сходящейся последовательности глобальных решений.
Несмотря на достигнутый успех, связанный с устойчивостью, теория ди-Перна"--~Лионса ничего не говорит о
единственности решения, его положительности, сохранении энергии, стремлении к равновесию.
Упомянем также результат Р.~Иллнера и М.~Шинброта о существовании и единcтвенности решения вблиза вакуума~\cite{Illner1984}.

%%% velocity-averaging lemmas
В теории ди-Перна"--~Лионса центральную роль играют леммы об осреднении в скоростном пространстве (velocity-averaging lemmas),
обеспечивающие гладкость макроскопических величин для достаточно произвольной функции распределения~\cite{Sentis1988}.
В.\,И.~Агошкову принадлежит первый подобный результат для уравнения переноса~\cite{Agoshkov1984}.
Независимо лемму открыли Ф.~Гольс, Б.~Пертам и Р.~Сентис~\cite{Sentis1985}.
Ф.~Гольс и Л.~Сэн-Ремон нашли доказательство в \(L^1\)~\cite{Raymond2002}.

%%% сходимость к равновесию
Несмотря на отсутствие общей теории сходимости к равновесию,
Л.~Девиллет и С.~Виллани смогли получить условный результат в ограниченной области
через явные оценки на поведение \(\mathcal{H}\)-функционала~\cite{Villani2005}.
Они доказали, что если все полиномиальные моменты равномерно ограничены,
то бесконечно гладкое строго положительное решение уравнения Больцмана стремится к равновесию
по меньшей мере с полиномиальной скоростью \(\OO{t^{-\infty}}\).
Характерной особенностью динамики больцмановского газа на больших временах является чередование режимов
близких с гидродинамическому и пространственно"=однородному, вследствие чего
образуются временн\'{ы}е осцилляции производства энтропии~\cite{Filbet2006}.
Позже С.~Виллани на основе полученного результата развил абстракную теорию \emph{гипокоэрцитивности}~\cite{Villani2009}
для анализа сходимости полугрупп, порождаемых вырожденными операторами,
по аналогии с теорией гипоэллиптичности Колмогорова"--~Хёрмандера.
За эти работы, а также исследование нелинейного затухания Ландау, С.~Виллани в 2010 году был удостоен Филдсовской медали.
М.~Гуалдани, С.~Мишлер и К.~Муо обобщили пространственно"=однородный результат Муо~\cite{Mouhot2006}
и разработали абстрактный метод сочетания количественного спектрального анализа с энтропийными методами~\cite{Gualdani2013}.

%\newpage
%============================================================================================================================
\section{Асимптотическая теория} \label{sect:asymptotic}

\subsection{Строгие математические результаты}

%%% строгая асимптотическая теория
Строгая асимптотическая теория тесно связана с развитием математической теории самих гидродинамических уравнений.
Существование слабых решений уравнений Навье"--~Стокса для несжимаемых течений\footnote{
Обычно говорят об уравнениях НС для несжимаемой жидкости, но ... посмотреть у Соне
} было показано Ж.~Лер\'{е}~\autocite{Leray1934}.
Сходимость к ним ренормализованных решений ди-Перна"--~Лионса установлена в трудах
К.~Бардоса, Ф.~Гольса, Д.~Левермора~\autocite{Bardos1993},
П.-Л.~Лионса, Н.~Масмоуди~\autocite{Masmoudi2001} и Л.~Сен-Ремон~\autocite{Golse2004}.
Строгая асимптотическая теория для сжимаемых течений далека от своей зрелости.
Частичные результаты о сходимости к гладким решениям уравнений Эйлера принадлежат
Т.~Нишиде~\autocite{Nishida1978} и Р.~Кафлишу~\autocite{Caflisch1980limit}.
В присутствии процессов высокой частоты (сравнимой с частотой столкновений молекул) больцмановская динамика
качественно отличается от классической гидродинамики на основе линейных законов Ньютона и Фурье.
На основе множества работ И.\,В.~Карлина, А.\,Н.~Горбаня, М.~Слемрода совместно с другими авторами,
можно сделать вывод, что в пределе малых чисел Кнудсена и конечных числах Маха корректные уравнения
гидродинамического типа должны демонстрировать константную диссипацию высокочастотных мод
и существенно нелокальный характер~\autocite{Gorban2014}.



Р.~Кафлиш получил аналогичный результат, установив, что усечённое разложение Гильберта
сходится к гладкому решению уравнений Эйлера~\cite{Caflisch1980limit}.
% нет позитивности
Т.~Нишида показал, что для любого гладкого решения уравнений Эйлера существует сходящаяся к нему
последовательность решений уравнения Больцмана~\cite{Nishida1978}.
%Nishida’s idea is to apply the Nirenberg-Ovsyannikov [61, 62] abstract variant of the Cauchy-Kovalevska theorem. упрощено Юкаем/Асано
% initial layer due to Ukai and Asano1983
%The only rigorous results have been proved by  and Ukai and Asano1983, then improved by Liu and Yu [79], [80]


% golse acoustic small-amplitude limit

one can say that any solution of Burnett equations (in particular, the constant equilibrium solution) is unstable with respect to short- wave perturbations (the instability paradox, in terminology of Jin-Slemrod [14]).

dispersive dominance and saturation of dissipation rate of the exact hydrodynamics in the short-wave limit and the viscosity modifica- tion at high divergence of the flow velocity are indicated as severe obstacles to the resolution of Hilbert’s 6th Problem.

Saturation of dissipation at high frequencies is a universal effect that does not appear in the classical hydrodynamic equations.
Nevertheless, some effects persist: the saturation of dissipation for high frequencies and the nonlocal character of the hydrodynamic equations.

Slemrod suggested that the proper exact hydrodynamic equation should have the form of the Korteweg hydrodynamics [32,99,133] rather than of Euler or Navier–Stokes ones.
The capillarity-like terms appear, indeed, in the energy balance for all hydrody- namic equations found as a projection of the kinetic equations onto the exact or approximate invariant hydrodynamic manifolds.
In the highly nonequilibrium gas, the capillarity energy becomes significant, and it tends to infinity for high-velocity gradients.

together with the low-frequency, low-gradient Chapman–Enskog asymptotics the high–frequency and high–gradient asymptotics of the hydrodynamic equations are also achievable in a constructive simple form


% Liu, Yu 2013: Ши-Сянь Ю, Тай-Пин Лю
% The solutions of the nonlinear problem by Ukai et al. [38] are straightforward generalizations of the linear solutions, using an effective tool, "upwind damping", by the second author of the current paper. The analysis there is basically linear. This upwind damping tool is also one of the basic components for the construction of linear stable manifolds in this paper. In a study independent of the present one, Golse [11] recently succeeded in obtaining the first non-trivial nonlinear Knudsen-type layer solution for the problem at Mach number zero with a generalized penalized functional (generalized upwind damping) to screen out the fluid components.
% The positivity of the weak Boltzmann shock profile was proved by Liu and Yu [20] using the time-asymptotic and energy methods. The original attempt in [20] was to make use of the monotonicity property. However, the authors soon realized that the monotonicity property was a highly non-trivial question which demanded an entirely new way of thinking, and so was left open. It eventually led to the present consideration of invariant manifolds. The monotonicity property is now established using the center manifold reduction.
% Liu, Yu 2004pos: In [Caflisch, Nicolaenko1982] approach, the Boltzmann shock profiles are approximated by the Navier-Stokes shock profiles. The approach does not yield the positivity of the shocks due to the polynomial perturbation to the exponentially small Gaussian tail in the momentum variable \xi . Our idea is to show the positivity using the time-asymptotic stability analysis.
% Yu2010: A Green’s function pointwise estimate approach was initiated in [13, 14] for the purpose of better understand- ing the qualitative and quantitative behaviors of perturbations of a viscous shock profile. We generalize and refine this approach here to have a better handling of the local wave interactions.

% Filbet, Mouhot, Pareschi 2006: A recent work [Liu,Yu2004Green] gave a detailed pointwise study of the Green function for the linearized Boltzmann equation in the domain \(x\in\mathbb{r}\). In particular in this case, there study shows that the long-time behavior is governed by "fluid-like waves" (corresponding to the waves of the linearized Euler and Navier-Stokes equations) whose amplitude decreases polynomially, whereas the amplitude of the “kinetic part” of the Green function decreases exponentially. We think it likely that this study could be extended to the torus, where the amplitude of the fluid and kinetic parts of the Green function should both decrease exponentially. Moreover the rate of decay of the kinetic part should not depend on the size of the box, whereas the rate of decay of the fluid part should do. Hence for a box small enough, the long-time behavior should be governed by the kinetic part of the Green function (that is like the spatially homogeneous Boltzmann equation), whereas for a box big enough, the long-time behavior should be governed by the “fluid-like waves”. This is precisely what we observe numerically, and thus this theoretical study could provide a rigorous proof of the numerical observations above, at least in the linearized regime.

% For the hard sphere potential, positivity of profiles, and the improved estimate (1.19) were shown by [Liu,Yu2004positivity] by a "macro-micro decomposition" method in which fluid (macroscopic, or equilib- rium) and transient (microscopic) effects are separated and estimated by different techniques. This was used in [Liu,Yu2004positivity] to establish time-evolution ary stability of profiles with respect to perturbations of zero fluid-dynamical mass, u(x)dx = 0, and thus, assuming the existence result of [Caflish,Nicolaenko], to establish positivity of Boltzmann profiles by the positive maximum principle for the Boltzmann equation (1.2) together with convergence to the Boltzmann profile of its own Maxwellian approximation: by definition, a perturbation of zero relative mass in fluid-dynamical variables.




%%% краевая задача для линеаризованного уравнения Больцмана
на основе линеаризованной теории, заложенной ещё Х.~Грэдом~\cite{Grad1963b}
кинетического пограничного слоя, описывает
течение газа над плоскостью под действием заданных нормальных потоков
импульса и энергии.

В частности, А.\,В.~Бобылев показал,
что уравнения Барнетта, получаемые в третьем приближении, неустойчивы~\cite{Bobylev1982}.


% Сложность численного анализа кнудсеновского слоя --- сингулярность вида \(\eta\log\eta\)~\cite{Chen2016}.

%%% Гидродинамическое описание
Состояние газа в уравнении Больцмана определяется функцией распределения \(f(x_i,\zeta_i)\).
Приняв, что число Кнудсена определённым образом стремится к нулю, можно перейти к менее детальному
\emph{гидродинамическому описанию}, которое требует задания лишь первых пяти моментов от \(f\):
плотности \(\rho(x_i)\), скорости \(v_i(x_i)\) и температуры \(T(x_i)\).
В общем случае при таком асимптотическом переходе \(f(x_i,\zeta_i)\) является функционалом от
\(\rho(x_i)\), \(v_i(x_i)\) и \(T(x_i)\). Если ограничиться нулевым порядком по \(k\)
в уравнении Больцмана~\eqref{eq:Boltzmann}, то \(f(\zeta_i)\) в таком пределе станет точечной
функцией от \(\rho\), \(v_i\) и \(T\). Другими словами, мы получим уравнения Эйлера,
описывающие локально максвелловское распределение.
Если левая часть уравнения Больцмана остаётся конечной при малом \(k\),
тогда столкновительный член \(J(f,f) = \OO{k}\), таким образом асимптотическая теория
изучает малое отклонение от локально максвелловского распределения.

%%% Два подхода, их различие
Формальное разложение уравнения Больцмана по степеням некоторого параметра было впервые
предложено Д.~Гильбертом~\cite{Hilbert1912, Hilbert1924}:
\begin{equation}\label{eq:Hilbert_sum}
    f = \sum_{n=0}^\infty k^n f_n(\bx,\bzeta,t).
\end{equation}
Используя теорему Гильберта о единственности, несложно показать, что всякое такое разложение
однозначно определяется определяется гидродинамическим состоянием.
Если известно, что \(f(x_i,\zeta_i)\) зависит только от макроскопических переменных \(\rho_r\):
\begin{equation}\label{eq:Enskog_macro}
    \rho_0 = \rho, \quad \rho_i = v_i, \quad \rho_4 = T
\end{equation}
и их градиентов произвольного порядка \(\nabla\rho_r = \Pder[\rho_r]{x_i}, \Pderder[\rho_r]{x_i}{x_j},\dots\),
то задача получения совместимых гидродинамических уравнений может быть решена проще с
помощью разложения Чемпена"--~Энскога~\cite{Enskog1917, Chapman1960}:
\begin{equation}\label{eq:Enskog_sum}
    f = \sum_{n=0}^\infty k^n f_n(\rho_r,\nabla\rho_r,\bzeta).
\end{equation}
В общем, метод Чемпена"--~Энскога является частным случаем метода сокращения информации,
широко применяемого в нелинейной теории возмущения~\cite{Bogaevski1987,Bogaevski1991}.
Детальный сравнительный анализ этих двух подходов можно найти в~\cite{Grad1970, Cercignani1973}.

%%% Недостатки Барнетта и старше
В зависимости от максимального порядка рассматриваемых членов из разложения Чемпена"--~Энскога
можно получить уравнения Эйлера, уравнения Навье"--~Стокса, уравнения Барнетта, супербарнеттовские уравнения и т.\,д.
В отличие от разложения Гильберта, в этих системах растёт порядок дифференциальных уравнений
с увеличением порядка учитываемых членов. При этом известно, что задача Коши для уравнений Барнетта
и следующих за ним является некорректно поставленной.
В частности, амплитуда акустических волн, описываемых этими уравнениями для максвелловских молекул,
растёт со временем~\cite{Bobylev1982}.
Для краевой задачи уравнения Барнетта и следующие за ним могут также давать нефизичные решения~\cite{Cercignani1973}.
Причина этих проблем заключается в том, что в уравнениях, получаемых с помощью разложения Чемпена"--~Энскога,
происходит смешивание членов различного порядка по \(k\).
По этой же причине, уравнения Навье"--~Стокса не приводят в общем случае
к асимптотическому решению уравнения Больцмана,
они являются поправкой первого порядка к уравнениям Эйлера.
Строгое асимптотическое решение можно получить с помощью разложения Гильберта.

%%% Разделение временных масштабов
Существенной особенностью перехода к \emph{континуальному пределу} (\(k\to0\)) в уравнении Больцмана
является разделение двух различных временных масштабов времени. Первый масштаб "--- средний интервал
времени между столкновениями \(\OO{k}\), второй "--- время макроскопического распада посредством
механизмов диффузии и теплопередачи \(\OO{k^{-1}}\).
С физической точки зрения, первый период соответствует релаксации \(H\)-функции до термодинамической
энтропии, а второй "--- релаксации энтропии к своему максимальному значению.

%%% Слои с экспоненциальным ростом функции распределения
Асимптотическое решение уравнения Больцмана подразумевает достаточно гладкое поведение
функции распределения во всей области, занимаемой газом, однако в \emph{пограничных}\footnote{
    Чтобы не вызывать путаницы с общепринятым в гидродинамике термином <<пограничный слой>>,
    будем последний называть <<слоем Прандтля>>.
} слоях она может описываться экспонентой по \(1/k\):
\begin{itemize}
    \item начальный слой, возникающий в момент времени \(t=0\);
    \item пристеночный слой, возникающий возле физической границы;
    \item ударный слой (волна), возникающий непосредственно в газе.
\end{itemize}
Явное выделение этих слоёв возможно лишь для малых \(k\),
когда оба временн\'{ы}х масштаба отличаются существенно.

%%% Обобщённое разложение Гильберта

%%% Стационарные задачи

%%% Пристеночные слои
\begin{itemize}
    \item слой Соне толщиной \(\OO{k^2}\);
    \item слой Кнудсена толщиной \(\OO{k}\);
    \item слой Прандтля толщиной \(\OO{\sqrt{k}}\).
\end{itemize}

%%% Нестационарные задачи
\todo{Функции Грина}

%%% Физика асимптотической теории
В фундаментальных работах математиков Д.~Гильберта, Д.~Энск\'{о}га, Х.~Грэда
были заложены основы асимптотической теории для малых \(\Kn\).
6-ая проблема
Решение уравнения Больцмана для слаборазреженного газа допускает отделение гидродинамической части
от существенно неравновесных пространственно"=временн\'{ы}х пограничных областей.
Например, начальный слой возникает на масштабе времени свободного пробега,
кнудсеновский и ударный слои "--- на масштабе длины свободного пробега,
около граничной поверхности и во внутреннем объёме, соотвественно.
При обтекании тел с конечными числами Маха в приграничной области вне слоя Кнудсена
формируется также слой Прандтля. У выпуклых тел также выделяется пограничный подслой Соне (С-слой),
куда распространяются разрывы функции распределения вдоль характеристик.

%%%Also, these expansions are not expected to be convergent, but only "asymptotic". In fact, a solution of the Boltzmann equation which could be represented as the sum of such a series would be a very particular one (normal in Grad's terminology) : it would be entirely determined by the fields of local density, mean velocity and pressure associated with it.
%%% Villani2006, p.41

%%% Математика асимптотической теории
%%% Bardos, Golse & Levermore (1991): Systematic program for the proof of hydrodynamic limits of weak solutions, in particular in the incompressible regime;
%%% Golse & Saint-Raymond (2004): Rigorous proof of the incompressible hydrodynamic limit for weak solutions of the Boltzmann equation; coming after works by Golse, Levermore, Masmoudi and others. this program would take 20 years to be completed

%%% кинетические слои
\emph{Пограничные} кинетические слои моделируются краевыми задачами в полупространстве~\autocite{Grad1969}.
Соответствующие им теоремы существования и единственности решения линеаризованного уравнения Больцмана для газа твёрдых сфер
были доказаны Н.\,Б.~Масловой~\autocite{Maslova1982} и независимо К.~Бардосом, Р.~Кафлишем, Б.~Николаенко~\autocite{Bardos1986}.
Пограничный слой с конденсацией и испарением изучен К.~Черчиньяни~\autocite{Cercignani1986}
и учениками К.~Бардоса~\autocite{Coron1988}.
Нелинейная теория заложена в трудах Л.~Аркерюда, А.~Нури~\autocite{Arkeryd2000},
С.~Юкая, Т.~Янга, Ш.-С.~Ю~\autocite{Ukai2003} и Ф.~Гольса~\autocite{Golse2008}.
Кинетическая теория ударных волн развита только для малых амплитуд.
Для жёстких короткодействующих потенциалов ударный профиль впервые был построен Р.~Кафлишем и Б.~Николаенко~\autocite{Caflisch1982}.
Cтабильность, позитивность~\autocite{Liu2004} и монотонность~\autocite{Liu2013} была показана Т.-П.~Лю и Ш.-С.~Ю.


Теорема о существовании и единственности решения линеаризованных краевых задач в полупространстве
(гипотеза Х.~Грэда, доказательства Н.\,Б.~Масловой, позже К.~Бардоса и др.)
позволила определять граничные условия для соответствующих уравнений гидродинамического типа
на основе численного анализа кнудсеновского слоя.
Асимптотическая теория нашла широкое применение в трудах Киотской группы
(Ё.~Соне, К.~Аоки, Ш.~Таката и др.) для систематического анализа множества классических задач.
Говоря об асимптотической теории часто подразумевают формальные разложения Гильберта и Чепмена"--~Энск\'{о}га,
не имеющие строгого математического обоснования. В частности, А.\,В.~Бобылев показал,
что уравнения Барнетта, получаемые в третьем приближении, неустойчивы.
Гидродинамические пределы уравнения Больцмана очень многообразны, их исследованию
посвящено множество трудов, в частности, Парижской группы (К.~Бардос, Ф.~Гольс, Л.~Сен-Ремон),
С.~Юкая, Ч.\,Д.~Левермора.
%% Golse,Saint-Raymond2001 сходимость ренормализованных диПерна-Лионса к слабым решениям Лере для несжим. НС.
%Она схожа с теорией Ж.~Лер\'{е} существования слабых решений уравнений Навье"--~Стокса~\cite{Leray1934}.

В некоторых случаях решение системы уравнений гидродинамического типа
проявляет кинетические эффекты даже в континуальном пределе.
Ё.~Соне ввёл понятие призрак"=эффекта (ghost effect) для описания этого феномена.


\subsection{Линеаризованные течения}

%%% Линеализованные течения
В отсутствие значительных температурных градиентов медленные течения описываются
линеаризованным уравнением Больцмана. Первые аналитические решения отдельных задач на его основе
для модельных уравнений появились в 1960-х годах в трудах П.~Веландера, Д.\,Р.~Виллиса, К.~Черчиньяни и др.
Непосредственные численные решения высокой точности линеаризованного уравнения Больцмана
для газа твёрдых сфер были получены в 1989--92 годах усилиями Киотской группы.
%%%%%%%%% нужны цитирования

\subsubsection{Гидродинамические уравнения}
\todo{Разложение Грэда"--~Гильберта}
\subsubsection{Коррекция в кнудсеновском слое}


\subsection{Медленные неизотермические течения}

%%% Медленные неизотермические течения
Для \emph{медленных неизотермических течений} (малые числа Маха, значительные перепады температур)
нелинейная асимптотическая теория приводит к \emph{уравнениям Когана"--~Галкина"--~Фридлендера} (КГФ)~\cite{Kogan1976},
описывающим поведение газа в гидродинамической области.
В 1970-х годах они были получены и подробно изучены советской группой ЦАГИ
(М.\,Н.~Коган, В.\,С.~Галкин, О.\,Г.~Фридлендер).
Несмотря на присутствие некоторых барнеттовских членов,
уравнения КГФ не теряют устойчивости ввиду медленности течений,
а движение газа под их действием называется \emph{нелинейным термострессовым течением}.
В первых работах эти уравнения были получены наиболее простым способом,
на основе разложения Чепмена"--~Энскога~\cite{Kogan1970, Kogan1971}.
Такие же уравнения получаются из разложения Гильберта~\cite{Galkin1974}.
Кроме коэффициентов вязкости и теплопроводности, в уравнения КГФ входят некоторые
термострессовые транспортные коэффициенты. Для некоторых молекулярных потенциалов они
были впервые вычислены с помощью полиномов Сонина~\cite{Burnett1935, Chapman1960}.
Для газа твёрдых сфер более точные значения получены с помощью
непосредственного численного решения соответствующих интегральных уравнений~\cite{Sone1996}.
В результате многолетнего труда под руководством О.\,Г. Фридлендера теория медленных неизотермических течений
была подтверждена экспериментально~\cite{Friedlander1997, Friedlander2003}.

\subsubsection{Гидродинамические уравнения}
\subsubsection{Коррекция в кнудсеновском слое}
\subsubsection{Классификация течений, вызываемых температурными напряжениями}

\subsection{Плоско-параллельные течения при конечных чисел Маха}

\subsubsection{Гидродинамические уравнения}
\subsubsection{Коррекция в кнудсеновском слое}

\clearpage
