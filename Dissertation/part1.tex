\chapter{Кинетическая теория газов} \label{chapt:theory}

\section{Основные понятия и уравнения} \label{sect:fundamentals}

Кинетическая теория основывается на представлении о молекулярном строении вещества.
Газом называется совокупность молекул, находящихся на столь больших расстояниях друг от друга,
что молекулы большую часть времени слабо взаимодействуют друг с другом.
Короткие промежутки времени, в течение которых молекулы сильно взаимодействуют, рассматриваются как \emph{столкновения}.
Если усредненной по времени потенциальной энергией взаимодействия молекул можно пренебречь
по сравнению с их кинетической энергией, то газ называется \emph{идеальным}.
Практически газы из нейтральных молекул при давлениях до сотен атмосфер могут рассматриваться как идеальные.
До этих же давлений вероятность тройных столкновений мала по сравнению с вероятностью двойных (или парных).
В идеальном газе объём, занятый молекулами, мал по сравнению с объёмом, занятым газом.
Другими словами, если \(d_m\) "--- эффективный диаметр молекулы,
\(n_0\) "--- число молекул в единице объёма, то в пределе \(d_m\to0\), \(n_0\to\infty\)
в идеальном газе \(n_0 d_m^3\to0\).
Если при этом конечной остаётся \emph{длина свободного пробега} молекул между столкновениями
\begin{equation}\label{eq:ell}
    \ell = \frac1{\sqrt2\pi d_m^2 n_0},
\end{equation}
то такой предельный континуум принято называть \emph{газом Больцмана}.

\subsection{Границы применимости физической модели}

Предполагается, что движение молекул может быть описано с помощью классической ньютоновской механики.
Квантовые эффекты существенны лишь при очень низких температурах и для легких молекул (водород, гелий, электроны).
Для водорода и гелия квантовые поправки существенны уже при нормальных условиях.
Большинство же газов сжижается при температуре,
при которой ещё нет необходимости применять квантовую теорию столкновения молекул.
Квантовые эффекты необходимо учитывать при неупругих столкновениях атомов и молекул
(возбуждение внутренних степеней свободы молекул, возбуждение электронных уровней и т.\,п.).
Потенциалы упругих взаимодействий молекул также могут быть вычислены лишь с помощью квантовой механики.
Однако при известном потенциале взаимодействия упругие столкновения могут быть рассмотрены классически.
Релятивистские эффекты существенны лишь при очень больших температурах (больших скоростях молекул).
Практически их можно не учитывать при температурах порядка десятков и сотен тысяч градусов.

% Nordheim1928: Boltzmann--Nordheim integral for Bose-Einstein condensate
% ZNG formalism for Bose-Einstein condensate
% Landau1936: Landau equation for Coulomb potential
% Vlasov1938: Vlasov colisionless kinetic equation
% Cerci&Kremer2002: relativistic Boltzmann equation
% Bobylev&Illner1997: Vlasov--Manev force based on the quasirelativistic Lagrangian formalism

\subsection{Функция распределения скоростей}

В диссертации повсеместно используются безразмерные переменные.
Соответствующие размерные референсные значения содержат верхний индекс \({}^{(0)}\).
Пусть \(x_iL\) (или \(\bx L\)) "--- прямоугольные координаты в физическом пространстве,
а \(\zeta_i\sqrt{2RT^{(0)}}\) (или \(\bzeta\sqrt{2RT^{(0)}}\)) "--- молекулярная скорость.
Здесь \(L\) и \(T^{(0)}\) "--- референсные длина и температура, \(R = k_B/m\) "--- удельная газовая постоянная,
где \(k_B=1.380658\times 10^{-23} \text{Дж·К}^{-1}\) "--- постоянная Больцмана,
\(m\) "--- масса отдельной молекулы.
В шестимерном объёме \(\dx\dzeta\) в момент времени \(t\) находятся
\begin{equation}\label{eq:distrib_function}
    \dd{N} = \frac{\rho^{(0)}L^3}{m} f(\bx,\bzeta,t)\dx\dzeta
\end{equation}
молекул. \(f\) "--- \emph{функция распределения} скоростей молекул в газе.
\(\rho^{(0)}\) "--- референсная плотность, \(f^{(0)} = \rho^{(0)}/(2RT^{(0)})^{3/2}\).

Макроскопические переменные: плотность \(\rho\rho^{(0)}\), скорость \(v_i\sqrt{2RT^{(0)}}\),
температура \(TT^{(0)}\), тензор напряжений \(p_{ij}p^{(0)} = p_{ij}\rho^{(0)}RT^{(0)}\)
и вектор теплового потока \(q_i p^{(0)}\sqrt{2RT^{(0)}}\), "--- определяются
как соответствующие моменты функции распределения:
\begin{gather}\label{eq:macro_variables}
    \rho = \int f(\bx,\bzeta,t)\dzeta, \\
    \rho v_i = \int \zeta_i f(\bx,\bzeta,t)\dzeta, \\
    \rho T = \frac23 \int (\zeta_i-v_i)^2 f(\bx,\bzeta,t)\dzeta, \\
    p_{ij} = 2\int (\zeta_i-v_i)(\zeta_j-v_j) f(\bx,\bzeta,t)\dzeta, \\
    q_i = \int (\zeta_i-v_i)(\zeta_j-v_j)^2 f(\bx,\bzeta,t)\dzeta.
\end{gather}
Трехмерное интегрирование по \(\bzeta\) здесь и далее проводится во всём пространстве \(\bzeta\).
Безразмерное давление \(p = p_{ii}/3 = \rho T\).

\subsection{Уравнение Больцмана}

Поведение функции распределения \(f\) определяется \emph{уравнением Больцмана}
\begin{equation}\label{eq:Boltzmann}
    \pder[f]{t} + \zeta_i\pder[f]{x_i} + F_i\pder[f]{\zeta_i} = \frac1k J(f,f),
\end{equation}
где билинейный оператор \(J(f,g)\) выражается как
\begin{equation}\label{eq:integral}
    J(f,g) = \frac12 \int(f'g'_*+f'_*g'-fg_*-f_*g) B\left(\frac{|\alpha_j V_j|}{V},V\right)
        \dd\Omega(\boldsymbol\alpha)\dzeta_*,
\end{equation}
\begin{equation*}
    \begin{array}{l l}
        f = f(\bx,\bzeta,t), & f_* = f(\bx,\bzeta_*,t), \\
        f' = f(\bx,\bzeta',t), & f'_* = f(\bx,\bzeta'_*,t), \\
        \zeta_i' = \zeta_i+\alpha_i\alpha_j V_j, & \zeta_{i*}' = \zeta_{i*}-\alpha_i\alpha_j V_j, \\
        \boldsymbol{V} = \bzeta_*-\bzeta, & V = \sqrt{V_i^2} = |V_i|.
    \end{array}
\end{equation*}
Предполагается, что внешняя сила \(2F_i k_B T^{(0)}/L\), действующая на молекулу,
не зависит от молекулярной скорости \(\bzeta\),
\(\boldsymbol\alpha\) "--- единичный вектор,
выражающий изменение направления молекулярной скорости из-за столкновения молекул,
\(\Omega(\boldsymbol\alpha)\) "--- элемент телесного угла в направлении~\(\boldsymbol\alpha\),
\emph{столкновительное ядро} \(B(|\alpha_i V_i|/V,V)\) "--- неотрицательная функция,
определяемая межмолекулярным потенциалом.
Интегрирование в~\eqref{eq:integral} проводятся по всему пространству \(\zeta_{i*}\) и
по всем направлениям \(\alpha_i\) (всей сферической поверхности) соответственно.
Интеграл \(J(f,f)\) называется \emph{интегралом столкновения}
или \emph{столкновительным членом} уравнения Больцмана.
Модифицированное число Кнудсена \(k\) вычисляется как
\begin{equation}\label{eq:modified_kn}
    k = \frac{\sqrt\pi}2\Kn = \frac{\sqrt\pi}2 \frac{\ell^{(0)}}L = \frac{m}{2\sqrt{2\pi} d_m^2 \rho^{(0)}L}.
\end{equation}
В безразмерном виде~\eqref{eq:Boltzmann} неявно учтено, что число Струхаля равно единице,
т.\,е. референсный отрезок времени \(t^{(0)} = L/\sqrt{2RT^{(0)}}\).

\subsubsection{Классификация столкновительных ядер}

В упрощённом виде столкновительный интеграл можно переписать как
\begin{equation*}
    B = V^\gamma b(\cos\theta), \quad b(\cos\theta) \underset{\theta\to0+}{\sim} \theta^{-2-\nu},
\end{equation*}
где \(\theta\) "--- угол отклонения частиц, сталкивающихся со скоростями \(\bzeta\) и \(\bzeta_*\).
Степенному потенциалу межмолекулярного взаимодействия \( U(r)=r^{1-s}\:(s>2) \) соответствуют параметры
\begin{equation*}
    \gamma = \frac{s-5}{s-1}\in(-3,1), \quad \nu = \frac{2}{s-1}\in(0,2).
\end{equation*}
В зависимости из значения \(\gamma\) содержательна следующая классификация:
\begin{itemize}\label{eq:power_classification}
    \item модель твёрдых сфер \(\gamma=1\),
    \item жёсткие потенциалы \(\gamma\in(0,1)\),
    \item максвелловский потенциал \(\gamma=0\),
    \item мягкие потенциалы \(\gamma\in(-3,0)\),
    \item кулоновский потенциал \(\gamma=-3\).
\end{itemize}
Все степенные потенциалы порождают неинтегрируемую \emph{угловую сингулярность}
вследствие слишком большого числа \emph{скользящих} столкновений (при малых \(\theta\)).
Рассмотрение таких сингулярных потенциалов, называемых \emph{дальнодействующими},
требует существенно более продвинутого математического аппарата.
Многие результаты оказываются проще для \emph{короткодействующих} потенциалов с обрезанием по углу~\cite{Grad1970}:
\begin{equation*}\label{eq:angular_cutoff}
    \int_0^\pi B(V, \cos\theta)\sin\theta\dd\theta < +\infty.
\end{equation*}
Для кулоновского потенциала вклад скользящих столкновений становится решающим,
поэтому уравнение Больцмана в пределе \(\nu\to2\) сводится к уравнению Ландау~\cite{Arseniev1990}.

На практике часто используют модельные потенциалы.
Например, для газа, состоящего из упругих жёстких сфер,
\begin{equation}\label{eq:ci_kernel_hs}
    B_\mathrm{HS} = \frac{|\alpha_i V_i|}{4\sqrt{2\pi}}, \quad
    B^{(0)} = 4\sqrt\pi d_m^2 \sqrt{RT^{(0)}}
\end{equation}
Эффективный диаметр молекул \(d_m\) обычно связывается с референсной вязкостью газа при \(T^{(0)}\).
Г.~Бёрд предложил однопараметрическую модель~\cite{Bird1981vhs}, в которой вязкость газа пропорциональна \(T^s\),
\begin{equation}\label{eq:ci_kernel_vhs}
    B_\mathrm{VHS} = \frac{|\alpha_i V_i|}{4\sqrt{2\pi}\Gamma(\frac52-s)}\left(\frac{V}2\right)^{1-2s},
\end{equation}
где \(\Gamma(z) = \int_0^\infty t^{z-1}e^{-t}\dd{t}\) "--- гамма"=функция.
Среди двухпараметрических моделей наиболее распространена модель Леннарда"--~Джонса~\cite{Jones1924}.
В настоящий момент с высокой точностью вычислены квантовомеханические \textit{ab initio} потенциалы
для всех стабильных инертных газов~\cite{Hellmann2017} и некоторых их смесей~\cite{Jager2017}.

\subsubsection{Вывод на основе законов механики}

%%% Вывод из уравнения Лиувилля
Обоснование своего кинетического уравнения, исходя из законов механики, было дано ещё Больцманом в эвристической форме
на основании гипотезы \emph{молекулярного хаоса} (\emph{Stosszahlansatz}).
Гильберт в своей знаменитой шестой проблеме поставил задачу строгого математического обоснования
процессов предельного перехода от атомистического понимания к моделям сплошной среды.
Такой переход может быть выполнен через промежуточный \emph{мезоскопический} уровень описания
на основе одночастичной функции распределения.
Х.~Грэду принадлежит первая математическая формулировка вывода уравения Больцмана из уравнения Лиувилля~\cite{Grad1949}.
Единственный строгий результат в общей постановке принадлежит О.~Лэнфорду~\cite{Lanford1975}.
Ему удалось показать, что на коротком промежутке времени (порядка времени среднего пробега)
цепочка уравнений ББГКИ для газа твёрдых сфер сходится в пределе Грэда"--~Больцмана почти везде к кинетическому уравнению.
Глобальная по времени сходимость известна только для очень частного случая распространения газа в вакуум~\cite{Illner1989}
или для малых флуктуаций отдельных частиц~\cite{Beijeren1980}.
В.\,И.~Герасименко и академик Д.\,Я.~Петрина предоставили первые количественные оценки на множество патологических
траекторий в оригинальном доказательстве Лэнфорда~\cite{Petrina1990}.
С.~Юкай упростил и формализовал доказательство, используя теорему Коши"--~Ковалевской~\cite{Ukai2001}.
Обобщение теоремы Лэнфорда для короткодействующих потенцилов потребовало существенно более деликатного
анализа событий асимптотически нулевой меры~\cite{Raymond2013, Pulvirenti2014}.
Недавно французские математики Т.~Бодино, И.~Г\'{а}ллахер и Л.~Сэн-Ремон получили количественные оценки для флуктуационного режима,
что позволило обосновать линейное уравнение Больцмана на временах порядка \(\ln\ln{N}\)~\cite{Raymond2016}
и линеаризованное при \(d=2\) на временах порядках \(\sqrt[4]{\ln\ln{N}}\)~\cite{Raymond2017},
где \(N\) "--- число частиц, заключённых в \(d\)-мерном торе.
На основе их работ был достигнут первый успех для дальнодействующих потенциалов~\cite{Ayi2017}.

\subsubsection{Симметрии и законы сохранения}

Для произвольной функции \(\phi(\bzeta)\) интеграл столкновения~\eqref{eq:integral}
удовлетворяет соотношению симметрии
\begin{equation}\label{eq:ci_symmetry}
    \int\phi(\bzeta)J(f,g)\dzeta = \frac14\int(\phi+\phi_*-\phi'-\phi'_*) J(f,g)\dzeta.
\end{equation}
Поскольку \(\zeta_i+\zeta_{i*}=\zeta'_i+\zeta'_{i*}\) и \(\zeta_i^2+\zeta_{i*}^2={\zeta'_i}^2+{\zeta'_{i*}}^{\!\!2}\),
то интеграл столкновений обладает несколькими \emph{сумматорными инвариантами} \(\psi_r\) (\(r=0, 1, 2, 3, 4\)):
\begin{equation}\label{eq:conser_moments}
    \int \psi_r J(f,g)\dzeta = 0,
\end{equation}
\begin{equation}\label{eq:summational_invariants}
    \psi_0 = 1, \quad \psi_i = \zeta_i, \quad \psi_4 = \zeta_i^2.
\end{equation}
Другими словами, в результате столкновений сохраняются масса, импульс и энергия.
Умножая уравнение Больцмана~\eqref{eq:Boltzmann} на \(\psi_r\)
и интегрируя результат по всему пространству \(\bzeta\), получим \emph{уравнения сохранения}:
\begin{gather}\label{eq:conservation}
    \pder[\rho]{t} + \pder{x_i}(\rho v_i) = 0, \\
    \pder{t}(\rho v_i) + \pder{x_j}\left( \rho v_i v_j + \frac{p_{ij}}2 \right) = \rho F_i, \\
    \pder{t}\left[ \rho\left( v_i^2 + \frac32T \right) \right]
        + \pder{x_j}\left[ \rho v_j\left( v_i^2 + \frac32T \right) + v_i p_{ij} + q_j \right]
        = 2\rho v_j F_j, \label{eq:conservation1}
\end{gather}
В классической гидрогазодинамике уравнения сохранения~\eqref{eq:conservation}--\eqref{eq:conservation1}
замыкаются соответствующими феномелогическими выражениями для \(p_{ij}\) и \(q_i\). Например,
\begin{equation}\label{eq:Euler}
    p_{ij} = p\delta_{ij}, \quad q_i = 0
\end{equation}
приводят к \emph{уравнениям Эйлера}, а соотношения
\begin{equation}\label{eq:Navier-Stokes}
    p_{ij} = p\delta_{ij} - \Gamma_1(T)\left( \pder[v_i]{x_j} + \pder[v_j]{x_i}
        - \frac23\pder[v_k]{x_k}\delta_{ij} \right)k, \quad
    q_i = -\frac54\Gamma_2(T)\pder[T]{x_i}k,
\end{equation}
называемые законами Ньютона и Фурье соответственно,
приводят к \emph{уравнениям Навье"--~Стокса}.
\(\Gamma_1(T)\) и \(\Gamma_2(T)\) "--- безразмерные коэффициенты \emph{вязкости}
и \emph{теплопроводности} газа, зависящие от температуры \(T\).

\subsubsection{Равновесное состояние}

Столкновительный член в уравнении Больцмана является диссипативным оператором,
вызывающим релаксацию любого распределения к равновесному.
Л.~Больцман ввёл понятие \(\mathcal{H}\)-функционала
\begin{equation}\label{eq:H_function}
    \mathcal{H}(f) = \int f \ln f \dzeta \dx.
\end{equation}
и доказал знаменитую \emph{\(\mathcal{H}\)-теорему}, гласящую, что
\begin{equation}\label{eq:H_theorem}
    \der[\mathcal{H}]{t} = -\int D(f) \dx \leq 0,
\end{equation}
где определён функционал \emph{производства энтропии}
\begin{equation}\label{eq:entropy_production}
    \mathcal{D}(f) = -\int J(f,f) \ln f \dzeta.
\end{equation}
Равенство в~\eqref{eq:H_theorem} достигается только для \emph{распределения Максвелла}
\begin{equation}\label{eq:Maxwell_distribution}
    f_M = \frac{\rho}{(\pi T)^{3/2}} \exp \left( -\frac{(\zeta_i-v_i)^2}{T} \right),
\end{equation}
для которого \(J(f_M,f_M) = 0\).
\(\mathcal{H}\)-теорема отражает второе начало термодинамики.

\subsubsection{Линеаризация}

Рассмотрим стационарный газ (\(\Pder{t}=0\)) при отсутствии внешних сил (\(F_i=0\))
с функцией распределения близкой к абсолютному распределению Mаксвелла \(E(\zeta)\):
\begin{equation}\label{eq:linear_condition}
    \phi = \frac{f}{E} - 1 \ll 1, \quad E(\zeta) = \frac1{\pi^{3/2}}\exp\left(-\zeta^2\right).
\end{equation}
В таком случае справедливо \emph{линеаризованное уравнение Больцмана}
\begin{equation}\label{eq:linear_Boltzmann}
    \zeta_i \pder[\phi]{x_i} = \frac1k \mathcal{L}(\phi),
\end{equation}
с \emph{линеаризованным интегралом столкновения}
\begin{equation}\label{eq:linear_ci}
    \mathcal{L}(\phi) = \int E_*(\phi'+\phi'_*-\phi-\phi_*) B
    \dd \Omega(\boldsymbol{\alpha}) \dzeta_*.
\end{equation}
Возмущённое локальное максвелловское распределение
\begin{equation}\label{eq:linear_maxwellian}
    \phi_M = \omega + 2\Z v_i + \left(\Z^2-\frac32\right)\tau
\end{equation}
удобно выражается через возмущённые макроскопические переменные:
\begin{equation}\label{eq:linear_macro}
    \omega = \rho-1, \quad \tau = T-1, \quad P = p-1, \quad P_{ij} = p_{ij} - \delta_{ij},
\end{equation}
которые вычисляются как моменты \(\phi\):
\begin{gather}\label{eq:linear_macro_variables}
    \begin{gathered}
        \omega = \int \phi E\dzeta, \quad
        v_i = \int \Z \phi E\dzeta, \quad
        \tau = \frac23 \int \left( \Z-\frac32 \right)^2 \phi E\dzeta, \\
        P = \tau + \omega, \quad
        P_{ij} = 2\int \ZZ \phi E\dzeta, \quad
        Q_i = \int \ZZ^2 \phi E\dzeta - \frac52v_i
    \end{gathered}
\end{gather}

Линейный оператор \(\mathcal{L}(\phi)\), также как и \(J(f,g)\), удовлетворяет соотношению симметрии
\begin{equation}\label{eq:linear_symmetry}
    \int\psi(\bzeta)\mathcal{L}(\phi)E\dzeta = \frac14\int EE_* (\psi+\psi_*-\psi'-\psi'_*)
        (\phi'+\phi'_*-\phi-\phi_*) B\dd\Omega\dzeta_*\dzeta.
\end{equation}
С помощью~\eqref{eq:linear_symmetry} несложно показать, что в гильбертовом пространстве
со скалярным произведением \( \inner{\phi}{\psi} = \int\phi\psi E\dzeta\)
оператор \(\mathcal{L}(\phi)\) самосопряжён и неположителен:
\begin{equation}\label{eq:linear_properties}
    \inner{\phi}{\mathcal{L}\psi} = \inner{\mathcal{L}\phi}{\psi}, \quad
    \inner{\phi}{\mathcal{L}\phi} \leq 0,
\end{equation}
причём равенство выполняется только для инвариантов столкновения \(\psi_r\),
которые являются собственными векторами для собственного значения \(\lambda = 0\).

\newcommand{\DrawSpectrum}[2][]{% line, points
    \vspace{.2em}
    \begin{tikzpicture}
        \begin{axis}[
            axis x line=center, axis y line=center,
            xmin=-3.5, xmax=.5, ymin=-.5, ymax=.5,
            ticks=none, unit vector ratio=1 1 1, footnotesize, thin
        ]
            \addplot[scatter, only marks]table[y expr=0, row sep=\\]{#2\\0\\};
            \addplot[blue, line width=3.5]table[y expr=0, row sep=\\]{#1\\};
        \end{axis}
    \end{tikzpicture}
}

\begin{table}
    \centering
    \caption{Спектр линеаризованного уравнения Больцмана для степенного потенциала.}\label{table:spectra}
    \newcommand{\ColW}{3.5cm}
    \begin{tabular}{ >{\centering\arraybackslash}c | >{\centering\arraybackslash}m{\ColW} | >{\centering\arraybackslash}m{\ColW} | >{\centering\arraybackslash}m{\ColW} }
        & мягкие & максвелловский & жёсткие \\\hline
        истинные & \DrawSpectrum{-3\\-2\\-1.2\\-.6\\-.2\\-.08\\-.03} & \DrawSpectrum{-3\\-1.8\\-1\\-.4} & \DrawSpectrum{-3\\-1.8\\-1\\-.4} \\\hline
        обрезанные & \DrawSpectrum[-1.8\\0]{-3\\-2.4\\-2\\-1.8} & \DrawSpectrum{-1.8\\-1.6\\-1.3\\-.9} & \DrawSpectrum[-1.8\\-3.5]{-1.8\\-1.6\\-1.3\\-.9}
    \end{tabular}
\end{table}

Спектр линеаризованного оператора существенно зависит от столкновительного ядра.
Качественное изменение спектра степенного потенциала происходит, когда молекулы становятся максвелловскими (табл.~\ref{table:spectra}).
Для немягких потенциалов характерна спектральная щель, соответствующая модулю наибольшего отрицательного собственного числа.
Спектр максвелловского потенциала всегда дискретен, но ограничен при обрезании по углу,
в то время как спектр мягких и жёстких потенциалов становится непрерывным на соответствующем отрезке.

\subsection{Граничные условия}

Для газа заключённого с физическую область \(\Omega\) граничные условия можно записать в достаточно общем виде
\begin{equation}\label{eq:scatterring_kernel}
    f_B(\bx,\bzeta,t) = \int_{(\zeta_{*i}-v_{Bi})n_i<0} \mathcal{R}(\bzeta,\bzeta_*,\bx,t)f(\bx,\bzeta_*,t)\dzeta_*
    \quad \left( x\in\partial\Omega, (\zeta_i-v_{Bi})n_i>0 \right),
\end{equation}
где \(n_i\) "--- единичная нормаль, направленная в сторону газа,
\(v_{Bi}\) "--- скорость перемещения граничной поверхности,
а на \emph{ядро рассеяния} \(\mathcal{R}\) может быть наложены следующие условия:
\begin{gather}
    \mathcal{R}(\bzeta,\bzeta_*) \geq 0, \label{eq:R:positivity}\\
    -\int_{(\zeta_i-v_{Bi})n_i>0} \frac{(\zeta_k-v_{Bk})n_k}{(\zeta_{j*}-v_{Bj})n_j}
        \mathcal{R}(\bzeta,\bzeta_*)\dzeta = 1, \label{eq:R:noflux}\\
    \int_{(\zeta_{*i}-v_{Bi})n_i<0} \mathcal{R}(\bzeta,\bzeta_*)f_B(\bzeta_*)\dzeta_* = f_B(\bzeta),
        \quad f_B = f_M(\rho_B, \bv_B, T_B). \label{eq:R:reciprocity}
\end{gather}
Условие~\eqref{eq:R:noflux} соответствует непористой и неабсорбирующей граничной поверхности
(поток массы через неё равен нулю).
Условие~\eqref{eq:R:reciprocity} требуется только от ядра рассеяния,
зависящего от плотности \(\rho_B\), скорости \(\bv_B\) и температуры \(T_B\) граничной поверхности.

На практике для твёрдой непроницаемой границы наиболее распространена однопараметрическая \emph{модель Максвелла}
\begin{equation}\label{eq:scatter_Maxwell}
    \begin{aligned}
        \mathcal{R}_\mathrm{M}(\bzeta,\bzeta_*) &= (1-\alpha_\mathrm{M})\delta\left[ \zeta_{i*} - \zeta_i + 2(\zeta_j-v_{Bj})n_jn_i \right] \\
        &-\frac{2\alpha_\mathrm{M}}{\pi T_B^2} (\zeta_{j*}-v_{Bj})n_j \exp\left( -\frac{(\zeta_k-v_{Bk})^2}{T_B} \right),
    \end{aligned}
\end{equation}
где \(\alpha_\mathrm{M}\in[0,1]\) "--- коэффициент аккомодации, \(\delta(\bzeta)\) "--- дельта"=функция.
В настоящем исследовании повсеместно используются граничные условия \emph{диффузного отражения},
соответствующие \(\alpha_\mathrm{M}=1\). Они линеаризуются следующим образом:
\begin{equation}\label{eq:linear_diffuse_bc}
    \begin{gathered}
        \phi_B(\bx, \bzeta) = \sigma_B + 2\zeta_j v_{Bj} + \left(\zeta_i-\frac32\right)^2\tau_B
            \quad (\zeta_j-v_{Bj}) n_j > 0, \\
        \sigma_B = \sqrt\pi v_{Bj}n_j - \frac{\tau_B}2 -2\sqrt\pi
            \int_{(\zeta_j-v_{Bj})n_j < 0} \zeta_j n_j \phi E\dzeta.
    \end{gathered}
\end{equation}
\emph{Зеркальное отражение} осуществляется при \(\alpha_\mathrm{M}=0\).

М.~Лампис и К.~Черчиньяни предложили двухпараметрическую модель~\cite{Cercignani1971}:
\begin{equation}\label{eq:scatter_CL}
    \begin{aligned}
        \mathcal{R}_\mathrm{CL}(\bzeta,\bzeta_*) &= \frac{2\zeta_n}{\pi\alpha_n\alpha_t(2-\alpha_t)T_B^2}
            I_0\left( \frac{2\sqrt{1-\alpha_n} \zeta_n\zeta_{n*}}{\alpha_n T_B} \right) \\
        &\times\exp\left(
            -\frac{[\zeta_n^2+(1-\alpha_n)\zeta_{n*}^2]^2}{\alpha_n T_B}
            -\frac{[\zeta_{tk}+(1-\alpha_t)\zeta_{tk*}]^2}{\alpha_t(2-\alpha_t) T_B}
        \right),
    \end{aligned}
\end{equation}
где \(\zeta_n = (\zeta_i-v_{Bi})n_i\) и \(\zeta_{ti} = \zeta_j(\delta_{ij}-n_in_j)\),
\begin{equation}\label{eq:CL_I0}
    I_0 = \frac1{2\pi}\int_0^{2\pi} \exp(x\cos t)\dd{t}.
\end{equation}
В~\eqref{eq:scatter_CL} используются коэффициенты аккомодации
\begin{itemize}
    \item касательной компоненты импульса \(\alpha_t\in[0,2]\),
    \item нормальной кинетической энергии \(\alpha_n\in[0,1]\).
\end{itemize}

\subsection{Модельные уравнения}

Широкий круг нелинейных задач изучен с высокой точностью для упрощённого (модельного) столкновительного оператора
\begin{equation}\label{eq:BGK}
    J_\mathrm{BGK}(f,f) = \frac{\rho}{k} (f_M-f),
\end{equation}
предложенного М.~Круком~\cite{Krook1954} и независимо П.~Веландером~\cite{Welander1954}.
В линеаризованной постановке
\begin{equation}\label{eq:linear_BGK}
    \mathcal{L}_\mathrm{BGK}(\phi) = -\phi + \omega + 2\zeta_i v_i + \left( \zeta_i^2 - \frac32 \right)\tau.
\end{equation}
Основной недостаток \emph{модели Крука"--~Веландера} "--- это фиксированное число Прандтля
\(\Pr \eqdef \Gamma_1(T)/\Gamma_2(T)\) (\(\Pr_\mathrm{BGK}=1\)).
Этого недостатка лишены \emph{модель Холвея}~\cite{Holway1966}
\begin{equation}\label{eq:ES_model}
    J_\mathrm{ES}(f,f) = \rho \left[\frac{\rho\sqrt{\det A_{ij}}}{(\pi T)^{3/2}}
        \exp\left(- \frac{A_{ij}c_i c_j}T \right) -f \right], \quad
    A_{ij}^{-1} = \frac{\delta_{ij}}{\Pr} - \frac{1-\Pr}{\Pr}\frac{p_{ij}}{p},
\end{equation}
подстраивающая тензор напряжений, и \emph{модель Шахова}~\cite{Shakhov1968}
\begin{equation}\label{eq:S_model}
    J_\mathrm{S}(f,f) = \rho \left[ f_M \left( 1+\frac{1-\Pr}{5}\frac{q_i c_i}{pT}
        \left(2c_i^2-5\right) \right)-f\right],
\end{equation}
корректирующая вектор потока тепла.
В~\eqref{eq:ES_model} и~\eqref{eq:S_model} использовано сокращение \(c_i = \zeta_i-v_i\).

Модель Шахова, вообще говоря, применима только для слабонеравновесного газа,
в противном случае функция распределения может принимать отрицательные значения.
Модель Холвея, как видно из определения, больше подходит для медленных неизотермических течений,
где процессы теплопроводности существенно превалируют над вязкостными.
Её популярность резко возрасла, после того как в 1999 году для неё была доказана \(\mathcal{H}\)-теорема~\cite{Perthame2000}.
Простота модельного столкновительного оператора обуславливает широкое разнообразие соответствующих численных методов.
Практические расчёты показывают, что разреженный газ, особенно с невысокой степенью неравновесности, достаточно адекватно
описывается модельными уравнениями, однако в настоящее время не представляется возможным произвести какие-либо
априорные оценки отклонения от истинного решения уравнения Больцмана.

%\newpage
%============================================================================================================================
\section{Математическая теория задачи Коши} \label{sect:cauchy}

Методы численного анализа и асимптотическая теория уравнения Больцмана тесно связаны
с математической теорией задачи Коши.
Строгий анализ краевых задач представляет существенные трудности, поэтому в текущем разделе
представлены известные результаты прежде всего для однородной в физическом пространстве задачи,
а также для бесконечного пространства \(\Omega=\mathbb{R}^d\) и тора \(\Omega=\mathbb{T}^d\).

\subsection{Пространственно-однородная задача}

%%% Важность задачи
Задача Коши для пространственно"=однородного уравнения Больцмана наиболее изучена.
Она имеет особую важность для численных методов, поскольку в большинстве из них используется
расщепление на транспортный и столкновительный операторы.
Кроме того, пространственная однородность устойчива в том смысле, что
слабая неоднородность распространяется во времени~\cite{Arkeryd1987}.
На сегодняшний день известно только одно семейство явных нестационарных решений,
найденное А.\,В.~Бобылевым~\cite{Bobylev1975} и независимо М.~Круком, Т.~Ву~\cite{Krook1976} для максвелловского газа.

%%% Немягкие короткодействующие потенциалы (моменты+сущ+един)
Количественные оценки для полиномиальных моментов
\begin{equation*}
    \|f(t)\|_{L^1_s} \eqdef \int_{\mathbb{R}^d} |\bzeta|^s f(t,\bzeta)\dzeta \quad (s>2)
\end{equation*}
являются базовым инструментом в пространственно"=однородной теории
и отражают поведение функции распределения для больших молекулярных скоростей.
Равномерная ограниченность полиномиальных моментов позволяет сразу же доказать
существование и единственность решения, а также \(\mathcal{H}\)-теорему Больцмана.
Первые такие результаты принадлежат Т.~Карлеману~\cite{Carleman1933} и А.\,Я.~Повзнеру~\cite{Povzner1962}
для газа твёрдых сфер, Д.~Моргенштерну~\cite{Morgenstern1954} для макселловского газа.
Общую \(L^1\)-теорию для класса жёстких короткодействующих потенциалов развил Л.~Аркерюд в 1971 году~\cite{Arkeryd1972}.
С.~Мишлер и Б.~Веннберг максимально ослабили начальные предположения в задаче Коши,
предоставив доказательство в предположении лишь конечности массы и энергии~\cite{Mischler1999}.
В общем случае энергия неубывает со временем, однако единственность в \(L^1\) достигается
только в классе решений с постоянной энергией~\cite{Wennberg1999}.

%%% Немягкие короткодействующие потенциалы (другие свойства)
Множество других важных результатов получено для немягких короткодействующих потенциалов:
\begin{itemize}
    \item существование максвелловской нижней границы~\cite{Pulvirenti1996, Pulvirenti1997},
    \item распространение гладкости, экcпоненциальное убывание разрывов~\cite{Carlen1999, Mouhot2004},
    \item распространение максвелловской верхней границы~\cite{Gamba2009upper, Bobylev2017}.
\end{itemize}
Столкновительное ядро имеет наиболее простой вид \(B = B(\cos\theta)\) для максвелловского потенциала,
поэтому математические результаты для него, как правило, предшествуют исследованиям других потенциалов,
однако в отличие от жёстких потенциалов полиномиальные моменты максвелловского газа равномерно ограничены лишь
при условии их ограниченности в начальный момент времени.

%%% Мягкие потенциалы
Меньше известно о поведении газа с мягкими потенциалами.
Прежде всего нет доказательства равномерной ограниченности полиномиальных моментов,
есть только оценка \(\|f(t)\|_{L^1_s} < C(1+t)\) при \(\|f(0)\|_{L^1_s} < \infty\)~\cite{Carlen2009}.
При \(\gamma<0\) столкновительное ядро обретает дополнительную \emph{кинетическую сингулярность} в точке \(\bzeta = \bzeta_*\),
поэтому смысл в \(L^1\) имеют только слабые формы столкновительного интеграла.
Его симметричные свойства (законы сохранения) позволяют построить решения при \(\gamma\geq-2\) и \(\nu<2\)~\cite{Arkeryd1981, Goudon1997}.
Используя конечность производства энтропии, удаётся регуляризовать столкновительный оператор при \(\gamma>-4\)~\cite{Villani1998}.

\subsubsection{Сходимость к равновесию}

%%% Сходимость к равновесию в L^2(M^-1)
Долгую историю имеет проблема сходимости решения к равновесному максвелловскому распределению \(M\).
Она тесно связана со спектральными свойствами линеаризованного столкновительного оператора.
Как известно, он имеет пятикратное нулевое собственное значение, остальные отрицательные.
Наличие спектральной щели (конечное расстояние между максимальным отрицательным собственным значением и нулевым)
для немягких потенциалов ведёт к экспоненциальному затуханию \(\OO{e^{-\lambda t}}\) возмущённых решений~\cite{Grad1963b}.
Напротив, отсутствие спектральной щели у мягких потенциалов позволяет рассчитывать только на оценку \(\OO{e^{-\lambda t^\alpha}}\),
где \(\alpha\in(0,1)\)~\cite{Caflisch1980a}.

%%% Неконструктивная сходимость к равновесию в L^1
Теория сходимости к равновесию в \(L^1\) также берёт начало с работы Л.~Аркерюда 1988 года~\cite{Arkeryd1988},
в которой была показана экспоненциальная скорость для твёрдых потенциалов.
Однако этот результат был получен неконструктивными методами.
С физической точки зрения крайне важно иметь явные оценки сходимости, поскольку на очень большом временном промежутке
уравнение Больцмана теряет смысл (парадокс Церм\'{е}ло).

%%% Сходимость в L^1 чисто нелинейными методами
\(\mathcal{H}\)-функционал (энтропия со знаком минус) выполняет роль функционала Ляпунова для уравнения Больцмана,
поэтому в соответствии с принципом Красовского"--~Ласаля производство энтропии для уравнения Больцмана
"--- основной инструмент для контроля сходимости к равновесию в нелинейной постановке.
Более того, согласно неравенству Чисара"--~Кульбака"--~Пинскера больцмановская энтропия
гарантирует сходимость непосредственно в \(L^1\).
В связи с этим К.~Черчиньяни в 1982 году предположил, что производство энтропии связано
линейным неравенством с самой энтропией~\cite{Cercignani1982, Desvillettes2011},
однако позже он вместе с А.\,В.~Бобылевым построили контрпример для \(\gamma\in(0,2)\)~\cite{Bobylev1999}.
Дж.~Тоск\'{а}ни и С.~Виллан\'{и} получили оптимальный результат в виде~\cite{Toscani1999, Toscani2000, Villani2003}
\begin{equation*}
    \mathcal{D}(f) \geq \lambda_\epsilon \left[ \mathcal{H}(f) - \mathcal{H}(M) \right]^{1+\epsilon},
    \quad \epsilon > 0.
\end{equation*}
Такой результат обеспечивает полиномиальную сходимость \(\OO{t^{-\infty}}\).
Первоначальная гипотеза Черчиньяни оказалась верна только для нефизичного случая \(\gamma=2\)~\cite{Villani2003}.

%%% Сходимость в L^1 для больцмановской полугруппы
Приведённые явные оценки на темп производства энтропии справедливы на чисто функциональном уровне,
поэтому они могут быть улучшены для полугруппы, порождаемой уравнением Больцмана.
В 1997 году Э.~Карлен, Э.~Габетта и Дж.~Тоск\'{а}ни получили оптимальный результат для максвелловского газа~\cite{Carlen1999}
\begin{equation*}
    \|f(t) - M\|_{L^1} \leq C_\epsilon e^{-(1-\epsilon)\lambda_g t}, \quad \epsilon > 0,
\end{equation*}
где \(\lambda_g\) "--- ширина спектральной щели, а \(\epsilon\) тем меньше,
чем больше полиномиальных моментов ограничено.
Если же в начальном условии ограничена лишь энергия, то существуют сколь угодно медленно сходящиеся к равновесию
решения для нежёстких потенциалов~\cite{Carlen2003, Carlen2009}.
Для жёстких потенциалов К.~Муо в 2005 году разработал инструменты расширения функционального пространства для линеаризованной полугруппы,
позволяющие соединить спектральные результаты в \(L^2(M^{-1})\) с нелинейной \(L^1\)-теорией~\cite{Mouhot2006}.
Таким образом, явные оценки на ширину спектральной щели~\cite{Baranger2005}
обеспечили оптимальную сходимость \(\OO{e^{-\lambda_g t}}\).

\subsubsection{Дальнодействующие потенциалы}

%%% Дальнодействующие потенциалы в XX веке
В XX веке дальнодействующим потенциалам было посвящено считанное количество работ.
Ё.-П.~Пао ещё в 1974 году с помощью теории псевдо"=дифференциальных операторов показал,
что спектр линеаризованного уравнения полностью дискретен~\cite{Pao1974}.
Давно было известно, что сингулярные операторы способны повышать гладкость решения,
однако соответствующие результаты для уравнения Больцмана долгое время были недоступны
из-за высокой технической сложности.
В 1994 году Л.~Девиллет смог доказать, что решение модельного уравнения Каца лежит
в пространстве Соболева при \(t>0\)~\cite{Desvillettes1995}.
В 1997 году П.-Л.~Лионс впервые установил функциональную связь между производством энтропии
и гладкостью функции распределения~\cite{Lions1998}.
В 1999 году Р.~Александре, Л.~Девиллет, Б.~Веннберг и С.~Виллани довели этот результат до оптимальной
в некотором смысле оценки~\cite{Alexandre2000}
\begin{equation*}
    \|\sqrt{f}\|_{H^{\nu/2}(|\bzeta|<R)} \leq C_R \left[ \mathcal{D}(f) + \|f\|^2_{L_2^1} \right], \quad R > 0.
\end{equation*}
Другими словами, сингулярный больцмановский столкновительный оператор ведёт себя
как дробный лапласиан \(-(-\Delta)^{\nu/2}\).
С физической точки зрения это означает, что процесс межмолекулярного взаимодействия в больцмановском пределе
является диффузно"=столкновительным.
Важную роль в доказательстве этого функционального неравенства играет преобразование Фурье,
приложение которого к уравнению Больцмана было систематически изучено А.\,В.~Бобылевым~\cite{Bobylev1984}.

%%% Дальнодействующие потенциалы в XXI веке
Единственность решения в глобальном смысле \(t\in[0,+\infty]\) в соболевском пространстве установлена
для максвелловского газа~\cite{Toscani1999maxw} и жёстких потенциалов в случае
умеренной угловой сингулярности \(\nu\in(0,1)\)~\cite{Mouhot2009} и сильной \(\nu\in[1,2)\)~\cite{He2012}.
При условии гладкости \(b(\cos\theta)\) оказывается, что функция распределения лежит в пространстве Шварца,
пока её полиномиальные моменты ограничены~\cite{Desvillettes2005, Alexandre2012}.
Более того, для максвелловского газа известно, что регулярность по Жевре распространяется~\cite{Desvillettes2009}.
Экспоненциальная сходимость с явными оценками в \(L^1\) для дальнодействующих немягких потенциалов
была доказана совсем недавно~\cite{Tristani2014, Meng2017}.
Сходимость для мягких потенциалов известна лишь в усреднённом смысле~\cite{Carlen2009}.

\subsection{Пространственно-неоднородная задача}

%%% асимптотическая устойчивость
Пространственно"=неоднородная задача принципиально сложнее для анализа.
Строгая линеаризованная теория была построена Х.~Грэдом~\cite{Grad1963b}.
Он же в 1964 году получил первые результаты в рамках нелинейной теории возмущения, доказав
существование, единственность возмущённого решения и стремление его к термодинамическому равновесию,
однако только для локального интервала времени~\cite{Grad1965}.
В 1974 году С.~Юкай представил глобальный результат в ограниченной области
для жёстких короткодействующих потенциалов~\cite{Ukai1974}.
Р.~Кафлиш обобщил его для \(\gamma>-1\)~\cite{Caflisch1980b}.
Спектральный анализ не позволяет продвинуться дальше границы \(\gamma=-2\).
Я.~Го предложил альтернативный метод, который позволил рассмотреть очень мягкие потенциалы~\cite{Guo2003},
Экспоненциальная сходимость в ограниченной области для них была доказана позже совместно с Р.~Стрейном~\cite{Strain2008}.
В 2010 году Ф.~Грессман и Р.~Стрейн, используя нетривиальные анизотропные cоболевские нормы~\cite{Mouhot2007} для метода Го,
показали асимптотическую устойчивость для всех дальнодействующих потенциалов~\cite{Strain2011}.
Оказалось, что при наличии угловой сингулярности \(\nu>0\) спектральная щель и
соответствующая ей экспоненциальная сходимость имеют место при \(\gamma+\nu\geq0\).
Такой же результат, но для неограниченной области был независимо получен
гонконгской группой~\cite{Alexandre2012soft, Alexandre2011hard, Alexandre2011properties}.
Сходимость в \(\mathbb{R}^d\) алгебраическая. Для \(\gamma>-1\) С.~Юкай и К.~Асано доказали это ещё в 1982 году~\cite{Ukai1982}.
Р.~Стрейн получил оптимальную оценку \(\OO{t^{-\frac{n}2+\frac{n}{2r}}}\) в \(L^r_\bx L^2_\bzeta\) при \(r\in[2,\infty]\)~\cite{Strain2012}.

%%% теория ди-Перна"--~Лионса
Единственная на сегодняшний день \(L^1\)-теория, описывающая решения полного уравнения Больцмана без
дополнительных предположений об их малости, принадлежит Р.~ди~Перна и П.-Л.~Лионсу~\cite{Lions1989}
и отмечена Филдсовской премией 1994 году.
Им удалось доказать существование слабого решения в \emph{ренормализованной} форме
\begin{equation*}
    \pder[\beta(f)]{t} + \zeta_i\pder[\beta(f)]{x_i} = \beta'(f)J(f,f),
\end{equation*}
где \(\beta'(f) \leq C/(1+f)\).
Позже П.-Л.~Лионс упростил доказательство, используя теорию интегральных операторов Фурье~\cite{Lions1994},
а Р.~Александре и С.~Виллани обобщили его для дальнодействующих потенциалов~\cite{Alexandre2002}.
Основная идея ренормализации "--- получить в правой части сублинейный оператор вместо квадратичного \(J(f,f)\).
В такой постановке априорных оценок для массы, энергии и энтропии оказывается достаточно
для построения сходящейся последовательности глобальных решений.
Несмотря на достигнутый успех, связанный с устойчивостью, теория ди-Перна"--~Лионса ничего не говорит о
единственности решения, его положительности, сохранении энергии, стремлении к равновесию.
Стоит упомянуть также частный результат Р.~Иллнера и М.~Шинброта о существовании и единcтвенности решения вблиза вакуума~\cite{Illner1984}.

%%% velocity-averaging lemmas
В теории ди-Перна"--~Лионса центральную роль играют леммы об осреднении в скоростном пространстве (velocity-averaging lemmas),
обеспечивающие гладкость макроскопических величин для достаточно произвольной функции распределения~\cite{Sentis1988}.
В.\,И.~Агошкову принадлежит первый подобный результат для уравнения переноса~\cite{Agoshkov1984}.
Независимо лемму открыли Ф.~Гольс, Б.~Пертам и Р.~Сентис~\cite{Sentis1985}.
Ф.~Гольс и Л.~Сэн-Ремон нашли доказательство в \(L^1\)~\cite{Raymond2002}.

%%% сходимость к равновесию
Несмотря на отсутствие общей теории сходимости к равновесию,
Л.~Девиллет и С.~Виллани смогли получить условный результат в ограниченной области
через явные оценки на поведение \(\mathcal{H}\)-функционала~\cite{Villani2005}.
Они доказали, что если все полиномиальные моменты равномерно ограничены,
то бесконечно гладкое строго положительное решение уравнения Больцмана стремится к равновесию
по меньшей мере с полиномиальной скоростью \(\OO{t^{-\infty}}\).
Характерной особенностью динамики больцмановского газа на больших временах является чередование режимов
близких с гидродинамическому и пространственно"=однородному, вследствие чего
образуются временн\'{ы}е осцилляции производства энтропии~\cite{Filbet2006}.
Позже С.~Виллани на основе полученного результата развил абстракную теорию \emph{гипокоэрцитивности}~\cite{Villani2009}
для анализа сходимости полугрупп, порождаемых вырожденными операторами,
по аналогии с теорией гипоэллиптичности Колмогорова"--~Хёрмандера.
За эти работы, а также исследование нелинейного затухания Ландау, С.~Виллани в 2010 году был удостоен Филдсовской медали.
М.~Гуалдани, С.~Мишлер и К.~Муо обобщили пространственно"=однородный результат Муо~\cite{Mouhot2006}
и разработали абстрактный метод сочетания количественного спектрального анализа с энтропийными методами~\cite{Gualdani2013}.

%%% функции Грина
Ещё один важный результат был получен Т.-П.~Лю и Ш.-С.~Ю в рамках теории функции Грина~\cite{Liu2004green,Liu2006}.
Они продемонстрировали, что затухание любого возмущения, описываемого линеаризованным уравнением Больцмана,
может быть разложено на кинетическую составляющую, убывающую экспоненциально,
и гидродинамическую, амплитуда которой спадает полиномиально в \(\mathbb{R}^3\).

%\newpage
%============================================================================================================================
\section{Асимптотическая теория} \label{sect:asymptotic}

%%% Гидродинамическое описание
Состояние газа в уравнении Больцмана определяется функцией распределения \(f(\bx,\bzeta)\).
Приняв, что число Кнудсена определённым образом стремится к нулю, можно перейти к менее детальному
\emph{гидродинамическому описанию}, которое требует задания лишь первых пяти моментов от \(f\):
плотности, скорости и температуры.
В общем случае при таком асимптотическом переходе \(f(\bx,\bzeta)\) является функционалом от
\(\rho(\bx)\), \(v_i(\bx)\) и \(T(\bx)\). Если ограничиться нулевым порядком по \(k\)
в уравнении Больцмана~\eqref{eq:Boltzmann}, то \(f(\bzeta)\) в таком пределе станет точечной
функцией от \(\rho\), \(v_i\) и \(T\). Другими словами, мы получим уравнения Эйлера,
описывающие локально максвелловское распределение.
Если левая часть уравнения Больцмана остаётся конечной при малом \(k\),
тогда столкновительный член \(J(f,f) = \OO{k}\), таким образом асимптотическая теория
изучает малое отклонение от локально максвелловского распределения.

%%% Разложение Гильберта
Формальное разложение уравнения Больцмана по степеням некоторого параметра было впервые
предложено Д.~Гильбертом~\cite{Hilbert1912, Hilbert1924}:
\begin{equation}\label{eq:Hilbert_sum}
    f = \sum_{n=0}^\infty k^n f_n(\bx,\bzeta,t).
\end{equation}
В общем случае нельзя ожидать сходимость этого ряда.
Более того решение вида~\eqref{eq:Hilbert_sum} представляет собой весьма узкий класс решений,
поскольку всякое такое разложение однозначно определяется гидродинамическим состоянием,
что несложно показать используя теорему Гильберта о единственности.

%%% Разложение Чемпена"--~Энскога
Если известно, что \(f(\bx,\bzeta)\) зависит только от макроскопических переменных \(\rho_r\):
\begin{equation}\label{eq:Enskog_macro}
    \rho_0 = \rho, \quad \rho_i = v_i, \quad \rho_4 = T,
\end{equation}
и их градиентов произвольного порядка \(\nabla\rho_r = \Pder[\rho_r]{x_i}, \Pderder[\rho_r]{x_i}{x_j},\dots\),
то задача получения совместимых гидродинамических уравнений может быть решена проще с
помощью \emph{разложения Чемпена"--~Энскога}~\cite{Enskog1917, Chapman1960}:
\begin{equation}\label{eq:Enskog_sum}
    f = \sum_{n=0}^\infty k^n f_n(\rho_r,\nabla\rho_r,\bzeta).
\end{equation}
В общем подход Чемпена"--~Энскога является методом сокращения информации,
широко применяемого в нелинейной теории возмущения~\cite{Bogaevski1987, Bogaevski1991}.

%%% Разделение временных масштабов
Существенной особенностью перехода к \emph{континуальному пределу} (\(k\to0\)) в уравнении Больцмана
является разделение двух различных временных масштабов времени. Первый масштаб "--- средний интервал
времени между столкновениями \(\OO{k}\), второй "--- время макроскопического распада посредством
механизмов диффузии и теплопередачи \(\OO{k^{-1}}\).
С физической точки зрения, первый период соответствует релаксации \(\mathcal{H}\)-функционала
до термодинамической энтропии, а второй "--- релаксации энтропии к своему максимальному значению.

%%% Слои с экспоненциальным ростом функции распределения
Асимптотическое решение уравнения Больцмана для слаборазреженного газа допускает отделение достаточно гладкой гидродинамической части
от существенно неравновесных пространственно"=временн\'{ы}х \emph{кинетических пограничных слоёв}:
\begin{itemize}
    \item \emph{начальный слой}, возникающий в момент времени \(t=0\);
    \item \emph{пристеночный слой}, возникающий возле физической границы;
    \item \emph{ударный слой}, возникающий непосредственно в газе.
\end{itemize}
Явное выделение этих слоёв возможно лишь для малых \(k\), когда оба временн\'{ы}х масштаба отличаются существенно.
Функция распределения в них содержит экспоненциальный фактор с множителем \(1/k\).
Пристеночный кинетический слой можно разделить на
\begin{itemize}
    \item \emph{слой Кнудсена} толщиной \(\OO{k}\),
    \item \emph{слой Соне} толщиной \(\OO{k^2}\)~\cite{Sone1973}.
\end{itemize}
Последний характеризуется проникновением тангенциальных разрывов в скоростном пространстве внутрь области с газом
и возникает исключительно вокруг выпуклых поверхностей, поскольку все разрывы распространяются вдоль характеристик.
Кроме кинетических можно выделить \emph{вязкий пограничный слой} (\emph{слой Прандтля}) толщиной \(\OO{\sqrt{k}}\).

\begin{figure}
    \centering
    \begin{tikzpicture}[
        arrow/.style={>={Stealth[scale length=1.5]}, thin, shorten <=.3pt}
    ]
        \footnotesize
        \pgftransformscale{3}

        \clip (-0.5,2) rectangle (1.5,5);

        \fill[fill=black!20, draw=black, very thick] (0,0) circle [radius=2.5];
        \foreach \i in {.65,1,2}
            \draw[semithick] (0,0) circle [radius=2+\i];

        \node at (0,2.58) {Соне};
        \node at (0,2.83) {Кнудсен};
        \node at (0,3.5) {Прандтль};
        \node at (0,4.5) {Эйлер};

        \draw[<->, arrow] (81:2.5) -- (81:4) node [right,midway] {\(\OO{\sqrt{k}}\)};
        \draw[<->, arrow] (73:2.5) -- (73:3) node [right,midway] {\(\OO{k}\)};
        \draw[thin] (65:2.3) node [above left=-2pt] {\(\OO{k^2}\!\)}  -- (65:2.85);
        \draw[>-<, arrow, shorten <=-7pt, shorten >=-7pt] (65:2.5) -- (65:2.65) ;
    \end{tikzpicture}
    \caption[
        Многослойная (многопалубная) структура течения слаборазреженного газа около выпуклого тела.
    ]{
        Многослойная (многопалубная) структура течения слаборазреженного газа около выпуклого тела:
        в невязкой области \(n_i\Pder[f]{x_i}=\OO{f}\);
        в слое Прандтля \(\sqrt{k}n_i\Pder[f]{x_i}=\OO{f}\);
        в слое Кнудсена \(kn_i\Pder[f]{x_i}=\OO{f}\);
        в слое Соне \(f\) разрывна.
    }\label{fig:asymptotic_structure}
\end{figure}

%%% Недостатки Барнетта и старше
В зависимости от максимального порядка рассматриваемых членов из разложения Чемпена"--~Энскога можно получить
уравнения Эйлера, уравнения Навье"--~Стокса для сжимаемых течений, уравнения Барнетта, супербарнеттовские уравнения и т.\,д.
В отличие от разложения Гильберта, в этих системах растёт порядок дифференциальных уравнений
с увеличением порядка учитываемых членов.
При этом известно, что задача Коши для уравнений Барнетта и следующих за ним является некорректно поставленной.
В частности, амплитуда акустических волн, описываемых этими уравнениями для максвелловских молекул,
растёт со временем~\cite{Bobylev1982}.
Другими словами, получаемые решения нестабильны по отношению к коротковолновым возмущениям.
Для краевой задачи уравнения Барнетта и следующие за ним могут также давать нефизичные решения~\cite{Cercignani1973}.
Причина этих проблем заключается в том, что в уравнениях, получаемых с помощью разложения Чемпена"--~Энскога,
происходит смешивание членов различного порядка по \(k\).
По этой же причине, уравнения Навье"--~Стокса для сжимаемых течений являются поправкой первого порядка к уравнениям Эйлера,
но не асимптотическим решением уравнения Больцмана.
А.\,В.~Бобылев предложил метод регуляризации уравнений Барнетта,
основанный на дополнительном перемешивании членов старшего порядка~\cite{Bobylev2006}.

Строгое асимптотическое решение можно получить с помощью разложения Гильберта,
однако при конечных числах Маха оно имеет весьма сложную структуру (рис.~\ref{fig:asymptotic_structure}),
поскольку требуется сращивание решений уравнений Эйлера, уравнений Прандтля в вязких пограничных слоях
и непосредственно уравнения Больцмана в кинетических пограничных слоях~\cite{Sone2000}.
Поэтому на практике численный анализ течений проводится с помощью только лишь уравнений Навье"--~Стокса для сжимаемых течений,
сочетающих несколько упомянутых временных масштабов.

%%% ghost effect
При различном пространственно"=временн\'{о}м масштабировании асимптотическая система уравнений гидродинамического типа
может содержать макроскопические переменные разного порядка~\cite{Bardos2008}.
В некоторых случаях это приводит к тому, что инфинитезимальные в континуальном пределе величины
конечным образом влияют на поля макроскопических переменных нулевого порядка.
Поскольку эти величины ненулевого порядка формально не существуют в континуальном мире,
то Ё.~Соне предложил ввести понятие призрак"=эффекта (ghost effect) для описания такого предельного поведения~\cite{Sone2000ghost}.

\subsection{Обзор строгих математических результатов}

%%% incompressible Navier--Stokes
Строгая асимптотическая теория тесно связана с развитием математической теории самих гидродинамических уравнений.
Ж.~Лер\'{е} в своих классических трудах доказал существование слабых решений уравнений Навье"--~Стокса
для несжимаемой жидкости~\cite{Leray1934}.
К.~Бардос, Ф.~Гольс и Д.~Левермор впервые поставили задачу сходимости к ним ренормализованных решений ди-Перна"--~Лионса~\cite{Bardos1991}
и получили первые частные результаты~\cite{Bardos1993} вместе с П.-Л.~Лионсом и Н.~Масмоуди~\cite{Masmoudi2001}.
В 2004 году Ф.~Гольс и Л.~Сен-Ремон построили доказательство для ограниченных ядер~\cite{Golse2004}
и позже обобщили для неограниченных~\cite{Golse2009}.
Окончательно \emph{программу Бардоса"--~Гольса"--~Левермора} завершил тунисский математик Н.~Масмоуди,
распространив результат для дальнодействующих потенциалов~\cite{Masmoudi2010}.

%%% compressible Euler
Строгая асимптотическая теория для сжимаемых течений далека от своей зрелости.
Частичные результаты о сходимости к гладким решениям уравнений Эйлера принадлежат
Т.~Нишиде~\cite{Nishida1978} и Р.~Кафлишу~\cite{Caflisch1980limit}.
Больцмановская динамика процессов высокой частоты (сравнимой с частотой столкновений молекул)
качественно отличается от классической гидродинамики на основе линейных законов Ньютона и Фурье.
% the saturation of dissipation for high frequencies and the nonlocal character of the hydrodynamic equations
На основе множества работ И.\,В.~Карлина, А.\,Н.~Горбаня, М.~Слемрода совместно с другими авторами,
можно сделать вывод, что в пределе малых чисел Кнудсена и конечных числах Маха корректные уравнения
гидродинамического типа должны демонстрировать константную диссипацию высокочастотных мод
и существенно нелокальный характер~\cite{Gorban2014}.

%%% кинетические слои
Пограничные кинетические слои моделируются краевыми задачами в полупространстве~\cite{Grad1969}.
Соответствующие им теоремы существования и единственности решения линеаризованного уравнения Больцмана для газа твёрдых сфер
были доказаны Н.\,Б.~Масловой~\cite{Maslova1982} и независимо К.~Бардосом, Р.~Кафлишем, Б.~Николаенко~\cite{Bardos1986}.
Пограничный слой с конденсацией и испарением изучен К.~Черчиньяни~\cite{Cercignani1986}
и учениками К.~Бардоса~\cite{Coron1988}.
Нелинейная теория заложена в трудах Л.~Аркерюда, А.~Нури~\cite{Arkeryd2000},
С.~Юкая, Т.~Янга, Ши-Сянь~Ю~\cite{Ukai2003} и Ф.~Гольса~\cite{Golse2008}.
Кинетическая теория ударных волн развита только для малых амплитуд.
Для жёстких короткодействующих потенциалов ударный профиль впервые был построен Р.~Кафлишем и Б.~Николаенко~\cite{Caflisch1982}.
Его стабильность, позитивность~\cite{Liu2004} и монотонность~\cite{Liu2013} была показана Тай-Пин~Лю и Ши-Сянь.~Ю.

\subsection{Слабовозмущённые течения}

Рассмотрим слаборазреженный газ, описываемый стационарным линеаризованным уравнением Больцмана~\eqref{eq:linear_Boltzmann}.
Абстрагируясь сначала от граничных условий, будем искать решение в виде степенного ряда,
называемого \emph{разложением Грэда"--~Гильберта} (будем приписывать индекс \(G\)),
\begin{equation}\label{eq:gh:phi_sum}
    \phi_G(\bx,\bzeta) = \sum_{m=0}^\infty \phi_{Gm}(\bx,\bzeta)k^m,
\end{equation}
подразумевая, что
\begin{equation}\label{eq:gh:fluid_assumption}
    \pder[\phi_G]{x_i} = \OO{\phi_G}.
\end{equation}
Соответствующие макроскопические величины \(h_G=\omega_G, v_{iG}, \tau_G,\dots\) также могут быть разложены в ряд
\begin{equation}\label{eq:gh:macro_sum}
    h_G(\bx) = \sum_{m=0}^\infty h_{Gm}(\bx)k^m.
\end{equation}
Подставляя~\eqref{eq:gh:phi_sum} в~\eqref{eq:linear_Boltzmann} и приравнивая члены одного порядка по \(k\),
получаем ряд интегральных уравнений для \(\phi_{Gm}\):
\begin{align}
    \mathcal{L}(\phi_{G0}) &= 0, \label{eq:gh:integral_eqn_0}\\
    \mathcal{L}(\phi_{Gm}) &= \zeta_i\pder[\phi_{Gm-1}]{x_i} \quad (m=1,2,3,\dots). \label{eq:gh:integral_eqn_m}
\end{align}
Однородное уравнение~\eqref{eq:gh:integral_eqn_0} имеет решение
в виде возмущённого распределения Максвелла~\eqref{eq:linear_maxwellian},
а для неоднородных уравнений~\eqref{eq:gh:integral_eqn_m} должны выполняться условия разрешимости
\begin{equation}\label{eq:gh:solvability}
    \int\psi_r\zeta_i\pder[\phi_{Gm-1}]{x_i}E\dzeta = 0,
\end{equation}
где \(\psi_r\) "--- инварианты столкновений~\eqref{eq:summational_invariants}.
В силу изотропных свойств оператора \(\mathcal{L}\) первые члены разложения Грэда"--~Гильберта имеют вид
\begin{multline}\label{eq:gh:phi_explicit}
    \phi_G = \phi_{GM} - \left( \ZZ B(\zeta)\pder[v_{iG}]{x_j}
        + \Z A(\zeta)\pder[\tau_G]{x_i}
        + \frac1{\gamma_1}\Z D_1(\zeta)\pder[P_G]{x_i}
    \right)k \\ + \left( \ZZZ D_2(\zeta)\pderder[v_{iG}]{x_j}{x_k}
        - \ZZ F(\zeta)\pderder[\tau_G]{x_i}{x_j}
    \right)k^2 + \OO{k^3}
\end{multline}
а соответствующие условия разрешимости дают \emph{уравнения Стокса}:
\begin{equation}\label{eq:gh:stokes}
    \left.\begin{gathered}
        \pder[P_{G0}]{x_i} = 0, \quad \pder[v_{iG0}]{x_i} = 0, \\
        \pder[P_{Gm+1}]{x_i} = \gamma_1\pderdual[v_{iGm}]{x_j}, \quad
        \pderdual[\tau_{Gm}]{x_i} = 0.
    \end{gathered}\quad\right\} \quad (m=0,1,2,\dots)
\end{equation}
Тензор напряжения и вектор потока могут быть вычислены из~\eqref{eq:gh:phi_explicit}:
\begin{gather}
    P_{ijG} = P_G\delta_{ij} - \gamma_1 \left( \pder[v_{iG}]{x_j} + \pder[v_j]{x_i} \right)k
        + \left( \gamma_3\pderder[\tau_G]{x_i}{x_j}
            - \frac{\gamma_6}{\gamma_1}\pderder[P_G]{x_i}{x_j} \right)k^2 + \OO{k^4} \label{eq:gh:stress}\\
    Q_{iG} = - \left( \frac54\gamma_2\pder[\tau_G]{x_i}
        - \frac{\gamma_3}{2\gamma_1}\pder[P_G]{x_i} \right)k + \OO{k^4} \label{eq:gh:heat_flux}
\end{gather}

Функции \(A(\zeta)\), \(B(\zeta)\), \(D_1(\zeta)\), \(D_2(\zeta)\), \(F(\zeta)\)
являются решениями следующих интегральных уравнений:
\begin{gather}
    \mathcal{L}\left[\Z A(\zeta)\right] = -\Z \left(\zeta^2-\frac52\right), \label{eq:gh:A}\\
    \mathcal{L}\left[\left(\ZZ-\frac13\zeta^2\delta_{ij}\right)B(\zeta)\right] =
        -2\left(\ZZ-\frac13\zeta^2\delta_{ij}\right), \label{eq:gh:B}\\
    \mathcal{L}\left[\left(\ZZ-\frac13\zeta^2\delta_{ij}\right)F(\zeta)\right] =
        \left(\ZZ-\frac13\zeta^2\delta_{ij}\right)A(\zeta), \label{eq:gh:F}\\
    \begin{multlined}
        \mathcal{L}\left[(\ZD{i}{jk}+\ZD{j}{ki}+\ZD{k}{ij})D_1(\zeta) + \ZZZ D_2(\zeta)\right] \\
            = \gamma_1(\ZD{i}{jk}+\ZD{j}{ki}+\ZD{k}{ij}) - \ZZZ B(\zeta),
    \end{multlined}\label{eq:gh:D}
\end{gather}
при дополнительных условиях:
\begin{gather}
    \int_0^\infty \zeta^4 A(\zeta)\exp(-\zeta^2)\dd\zeta = 0, \\
    \int_0^\infty \left[ 5\zeta^4 D_1(\zeta) + \zeta^6 D_2(\zeta) \right]\exp(-\zeta^2)\dd\zeta = 0.
\end{gather}
Транспортные коэффициенты вычисляются через соответствующие интегралы от этих функций:
\begin{equation}\label{eq:gh:gammas}
    \left.\begin{aligned}
        \gamma_1 &= I_6(B), \quad \gamma_2 = 2I_6(A), \\
        \gamma_3 &= I_6(AB) = -2I_6(F), \\
        \gamma_6 &= \frac12 I_6(BD_1) + \frac{3}{14} I_8(BD_2),
    \end{aligned}\quad\right\}
\end{equation}
где
\begin{equation}\label{eq:gh:I_Z}
    I_n(Z) = \frac{8}{15\sqrt\pi}\int_0^\infty \zeta^n Z(\zeta) \exp(-\zeta^2)\dd\zeta.
\end{equation}
Для газа твёрдых сфер
\begin{equation}\label{eq:gh:gammas_hs}
    \left.\begin{aligned}
        \gamma_1 &= 1.270042427, \quad \gamma_2 &= 1.922284066, \\
        \gamma_3 &= 1.947906335, \quad \gamma_6 &= 1.419423836,
    \end{aligned}\quad\right\}
\end{equation}
а для модели Крука"--~Веландера
\begin{equation}\label{eq:gh:gammas_bgk}
    \gamma_1 = \gamma_2 = \gamma_3 = \gamma_6 = 1.
\end{equation}

\subsubsection{Слой Кнудсена и граничные условия}

Разложение Грэда"--~Гильберта не обладает достаточным количеством свободных параметров,
чтобы удовлетворить кинетическим граничным условиям, например диффузного отражения~\eqref{eq:linear_diffuse_bc}.
При малых \(k\) решение стационарного уравнения Больцмана допускает разделение пространственных масштабов,
так что оно может быть найдено в форме
\begin{equation}\label{eq:gh:sum_solutions}
    \phi = \phi_G + \phi_K,
\end{equation}
где \(\phi_G\) "--- \emph{гидродинамическая часть} решения с масштабом порядка единицы,
\(\phi_K\) "--- \emph{кнудсеновская часть} или \emph{поправка кнудсеновского слоя} с масштабом порядка \(k\).
Оказывается, что кинетические граничные условия могут быть удовлетворены в предположении
\begin{equation}\label{eq:gh:knudsen_assumption}
    kn_i\pder[\phi_K]{x_i} = \OO{\phi_K}.
\end{equation}

Вводя естественную для слоя Кнудсена локальную систему координат \((\eta,\chi_1,\chi_2)\),
\begin{equation}\label{eq:eta_definition}
    x_i = k\eta n_i(\chi_1,\chi_2) + x_{Bi}(\chi_1,\chi_2),
\end{equation}
где \(x_{Bi}\) "--- граничная поверхность, \(\eta\) "--- координата, растянутая вдоль нормали \(n_i\),
\(\chi_1\) и \(\chi_2\) "--- ортогональные координаты поверхности \(\eta=const\),
получаем уравнение для \(\phi_K\):
\begin{equation}\label{eq:gh:knudsen_equation}
    \zeta_in_i\pder[\phi_K]{\eta} = \mathcal{L}(\phi_K)
    - k\zeta_i\left(\pder[\chi_1]{x_i}\pder[\phi_K]{\chi_1} + \pder[\chi_2]{x_i}\pder[\phi_K]{\chi_2}\right).
\end{equation}
Раскладывая \(\phi_K\) в аналогичный степенной ряд
\begin{equation}\label{eq:gh:phiK_sum}
    \phi_K(\bx,\bzeta) = \sum_{n=0}^\infty k^n \phi_{Kn}(\bx,\bzeta)
\end{equation}
и приравнивая члены одного порядка по \(k\), получаем одномерные по физическому пространству
уравнения для \(\phi_{K0}\) и \(\phi_{K1}\):
\begin{gather}
    \zeta_in_i\pder[\phi_{K0}]{\eta} = \mathcal{L}(\phi_{K0}), \label{eq:gh:knlayer0}\\
    \zeta_in_i\pder[\phi_{K1}]{\eta} = \mathcal{L}(\phi_{K1})
        - \zeta_i\left[ \onwall{\pder[\chi_1]{x_i}} \pder[\phi_{K0}]{\chi_1}
            + \onwall{\pder[\chi_2]{x_i}} \pder[\phi_{K0}]{\chi_2}
        \right], \label{eq:gh:knlayer1}
\end{gather}
с граничными условиями:
\begin{equation}\label{eq:gh:phiK_bc}
    \onwall{\phi_{Km}} = \phi_{Bm} - \onwall{\phi_{Gm}} \quad (\zeta_in_i>0),
        \quad \lim_{\eta\to\infty} \phi_{Km} = 0,
\end{equation}
где \(\onwall{\cdots}\) означает значение на границе (\(\eta=0\)).
Решение поставленных задач в полупространстве существует при определённых функциональных соотношениях
между \(\omega_{Gm}\), \(v_{iGm}\), \(\tau_{Gm}\) и локальными параметрами \(\phi_{Bm}\).
При дополнительном требовании
\begin{equation}\label{eq:gh:phiK_exp_decay}
    \phi_{Km} = \OO{\eta^{-\infty}}, \quad \eta\to\infty
\end{equation}
достигается единственность, определяющая однозначность декомпозиции~\eqref{eq:gh:sum_solutions}.

Для диффузного отражения~\eqref{eq:linear_diffuse_bc} получаются
следующие гидродинамические граничные условия и поправки кнудсеновского слоя:
\begin{gather}
    \begin{aligned}
    \begin{bmatrix} (v_{jG} - v_{Bj}) \\ v_{jK} \end{bmatrix} \deltann{i}{j} =
        &- \onwall{\pder[v_{jG}]{x_k} + \pder[v_k]{x_j}} \deltann{i}{j}n_k
            \begin{bmatrix} k_0 \\ Y_0(\eta) \end{bmatrix}k \\
        &- \onwall{\pder[\tau_G]{x_j}} \deltann{i}{j}
            \begin{bmatrix} K_1 \\ \frac12 Y_1(\eta) \end{bmatrix}k + \OO{k^2},
    \end{aligned} \label{eq:gh:boundary_vt}\\
    \begin{bmatrix} v_{iG} \\ v_{iK} \end{bmatrix} n_i = \OO{k^2}, \label{eq:gh:boundary_vn}\\
    \begin{bmatrix} \tau_G - \tau_B \\ \tau_K \\ \omega_K \end{bmatrix}
        = \onwall{\pder[\tau_G]{x_i}} n_i
            \begin{bmatrix} d_1 \\ \Theta_1(\eta) \\ \Omega_1(\eta) \end{bmatrix}k
        + \OO{k^2}. \label{eq:gh:boundary_tau}
\end{gather}
За каждым из полученных коэффициентов закреплен соответствующий термин, отражающий его физический смысл.
\(k_0\) и \(d_1\) соответствуют \emph{скоростному и температурному скачкам}, а \(K_1\) \emph{тепловому скольжению}.
Для газа твёрдых сфер~\cite{Ohwada1989creep, Ohwada1989jump, Takata2015}
\begin{equation}\label{eq:slip_coeffs_first}
    k_0 = -1.25395, \quad d_1 = 2.40014, \quad K_1 = -0.64642.
\end{equation}
Поскольку \(K_1<0\), направление теплового скольжения совпадает
с направлением градиента температуры граничной поверхности.
Функции \(Y_0(\eta)\), \(Y_1(\eta)\), \(\Theta_1(\eta)\), \(\Omega_1(\eta)\) убывают экспоненциально от \(\eta\)
и табулированы для газа твёрдых сфер в~\cite{Ohwada1989creep, Ohwada1989jump, Sone2002, Sone2007, Takata2015}.

\subsection{Медленные неизотермические течения}

Для \emph{медленных неизотермических течений} (малые числа Маха, значительные перепады температур)
нелинейная асимптотическая теория приводит к \emph{уравнениям Когана"--~Галкина"--~Фридлендера} (КГФ)~\cite{Kogan1976},
описывающим поведение газа в гидродинамической области.
В 1970-х годах они были получены и подробно изучены советской группой ЦАГИ
(М.\,Н.~Коган, В.\,С.~Галкин, О.\,Г.~Фридлендер).
Несмотря на наличие некоторых барнеттовских членов,
уравнения КГФ не теряют устойчивости ввиду медленности течений,
а движение газа под их действием называется \emph{нелинейным термострессовым течением}.
В первых работах эти уравнения были получены наиболее простым способом,
на основе разложения Чепмена"--~Энскога~\cite{Kogan1970, Kogan1971}.
Такие же уравнения получаются из разложения Гильберта~\cite{Galkin1974}.
Кроме коэффициентов вязкости и теплопроводности, в уравнения КГФ входят некоторые
термострессовые транспортные коэффициенты. Для некоторых молекулярных потенциалов они
были впервые вычислены с помощью полиномов Сонина~\cite{Burnett1935, Chapman1960}.
Для газа твёрдых сфер более точные значения получены с помощью
непосредственного численного решения соответствующих интегральных уравнений~\cite{Sone1996}.
В результате многолетнего труда под руководством О.\,Г. Фридлендера теория медленных неизотермических течений
была подтверждена экспериментально~\cite{Friedlander1997, Friedlander2003}.

Далее излагаются основные результаты нелинейной асимптотической теории на основе разложения Гильберта.
Решение, как в случае слабовозмущённых течений, может быть найдено в виде суммы
\begin{equation}\label{eq:sum_solutions}
    f = f_H + f_K,
\end{equation}
где гидродинамическая часть (\emph{разложение Гильберта}) \(f_H\),
и поправка кнудсеновского слоя \(f_K\) обладают следующими свойствами:
\begin{align}
    \pder[f_H]{x_i} = \OO{f_H}, &\quad k\to0, \label{eq:Hilbert_assumption}\\
    k n_i\pder[f_K]{x_i} = \OO{f_K}, &\quad k\to0, \label{eq:Knudsen_assumption}\\
    f_K = \OO{\eta^{-\infty}}, &\quad \eta\to\infty. \label{eq:Knudsen_decay}
\end{align}

\subsubsection{Гидродинамические уравнения}

Функция распределения \(f_H\) и макроскопические переменные \(h_H = \rho_H, v_{iH}, T_H, \dots\)
также разлагаются в ряд по \(k\):
\begin{equation}\label{eq:Hilbert_expansion}
    f_H = \sum_{m=0}^\infty f_{Hm} k^m, \quad h_H = \sum_{m=0}^\infty h_{Hm} k^m.
\end{equation}
Будем искать решение уравнения Больцмана в предположении
медленности течения (\(v_{iH0} = 0\)) и слабости поля внешних сил (\(F_{iH0} = F_{iH1} = 0\)).
Подставляя~\eqref{eq:Hilbert_expansion} в уравнение Больцмана~\eqref{eq:Boltzmann} и приравнивая члены
при равных степенях \(k\), получаем систему интегро"=дифференциальных уравнений,
для которой должны выполняться условия разрешимости. В нулевом порядке
\begin{equation}
    \pder[p_{H0}]{x_i} = 0. \label{eq:kgf:asymptotic0_p}
\end{equation}
В первом
\begin{gather}
    \pder{x_i}\left(\frac{u_{iH1}}{T_{H0}}\right) = 0, \label{eq:kgf:asymptotic1_u} \\
    \pder[p_{H1}]{x_i} = 0, \label{eq:kgf:asymptotic1_p} \\
    \pder[u_{iH1}]{x_i} = \frac12\pder{x_i}\left(\Gamma_2\pder[T_{H0}]{x_i}\right). \label{eq:kgf:asymptotic1_T}
\end{gather}
Во втором при \(p_{H1} = 0\)
\begin{gather}
    \pder{x_i}\left(\frac{u_{iH2}}{T_{H0}}\right)
        = \frac{u_{iH1}}{T_{H0}}\pder{x_i}\left(\frac{T_{H1}}{T_{H0}}\right), \label{eq:kgf:asymptotic2_u} \\
    \begin{aligned}
        \pder{x_j}\left(\frac{u_{iH1}u_{jH1}}{T_{H0}}\right)
        &-\frac12\pder{x_j}\left[\Gamma_1\left(
            \pder[u_{iH1}]{x_j} + \pder[u_{jH1}]{x_i} - \frac23\pder[u_{kH1}]{x_k}\delta_{ij}
        \right)\right] \\
        &-\left[
            \frac{\Gamma_7}{\Gamma_2}\frac{u_{jH1}}{T_{H0}}\pder[T_{H0}]{x_j}
            + \frac{\Gamma_2^2}4 \left(\frac{\Gamma_7}{\Gamma_2^2}\right)'
                \left(\pder[T_{H0}]{x_j}\right)^2
        \right]\pder[T_{H0}]{x_i} \\
        &= -\frac12\pder[p_{H2}^\dag]{x_i} + \frac{p_{H0}^2 F_{iH2}}{T_{H0}},
    \end{aligned} \label{eq:kgf:asymptotic2_p} \\
    \pder[u_{iH2}]{x_i} = \frac12\pderdual{x_i}\left(\Gamma_2T_{H1}\right). \label{eq:kgf:asymptotic2_T}
\end{gather}
Здесь введены следущие обозначения: \(u_{iH1} = p_{H0}v_{iH1}\), \(u_{iH2} = p_{H0}v_{iH2}\) и
\begin{equation}\label{eq:kgf:dag_pressure}
    p_{H2}^\dag = p_0 p_{H2}
        + \frac23\pder{x_k}\left(\Gamma_3\pder[T_{H0}]{x_k}\right)
        - \frac{\Gamma_7}{6}\left(\pder[T_{H0}]{x_k}\right)^2.
\end{equation}
Уравнения~\eqref{eq:kgf:asymptotic1_u},~\eqref{eq:kgf:asymptotic1_T},~\eqref{eq:kgf:asymptotic2_p}
для \(T_{H0}\), \(u_{iH1}\) и \(p_{H2}^\dag\) называются \emph{уравнениями Когана"--~Галкина"--~Фридлендера} (\emph{КГФ}).
Они содержат член температурных напряжений, отсутствующий в уравнениях Навье"--~Стокса.
Сравнивая его с \(p_{H0}^2F_{iH2}/T_{H0}\), можно увидеть, что на покоящуюся единицу массы газа действует сила
\begin{equation}\label{eq:kgf:gamma7_force}
    F_i = \frac{\Gamma_2^2}4\left(\frac{\Gamma_7}{\Gamma_2^2}\right)'\frac{T_H}{p_H^2}
        \left(\pder[T_H]{x_j}\right)^2 \pder[T_H]{x_i} k^2 + \OO{k^3}.
\end{equation}
Она исчезает только тогда, когда изотермические поверхности параллельны:
\begin{equation}\label{eq:kgf:parallel}
    e_{ijk}\pder[T_{H0}]{x_j}\pder{x_k}\left(\pder[T_{H0}]{x_l}\right)^2 = 0.
\end{equation}
В~\eqref{eq:kgf:parallel} использован символ Л\'{е}ви-Чив\'{и}ты \(e_{ijk}\).
Движение газа под действием этой силы называется \emph{нелинейным термострессовым} течением.
Важно отметить, что \(p_{H2}^\dag\) не входит непосредственно в уравнение состояния,
поэтому определяется с точностью до константы.

Транспортные коэффициенты \(\Gamma_i=\Gamma_i(T_{H0})\) зависят от температуры.
Первые два из них соответствуют размерным вязкости \(\mu\) и теплопроводности \(\lambda\) газа,
\begin{equation}\label{eq:mu_lambda}
    \mu = \Gamma_1(T_H) \frac{p^{(0)}L}{\sqrt{2RT^{(0)}}} k, \quad
    \lambda = \frac{5\Gamma_2(T_H)}{2} \frac{p^{(0)}RL}{\sqrt{2RT^{(0)}}} k.
\end{equation}
Для степенного молекулярного потенциала можно записать
\begin{equation}\label{eq:kgf:Gammas}
    \Gamma_{1,2}(T) = \gamma_{1,2}T^s, \quad \Gamma_3(T) = \gamma_3T^{2s},
        \quad \Gamma_7(T) = \Gamma_3' - \gamma_7T^{2s-1}.
\end{equation}
Для газа твёрдых сфер
\begin{equation}\label{eq:kgf:gammas}
    s=0.5, \quad \gamma_7 = 0.189201, \quad \Gamma_7 = 1.758705,
\end{equation}
для модели Крука"--~Веландера
\begin{equation}\label{eq:kgf:gammas_bgk}
    s=1, \quad \gamma_7 = 1, \quad \Gamma_7 = 1.
\end{equation}

Тензор напряжений и вектор потока тепла вычисляются по следующим формулам:
\begin{gather}
    \begin{aligned}
    p_{ijH} = p_H\delta_{ij} &- \frac{\Gamma_1}{p_H}\left(
            \pder[u_{iH}]{x_j} + \pder[u_{jH}]{x_i} - \frac23\pder[u_{kH}]{x_k}\delta_{ij} \right)k \\
        &+ \frac{\Gamma_3'-\Gamma_7}{p_H}\left[
            \pder[T_H]{x_i}\pder[T_H]{x_j} - \frac13\left(\pder[T_H]{x_k}\right)^2\delta_{ij} \right]k^2 \\
        &+ \frac{\Gamma_3}{p_H}\left(
            \pderder[T_H]{x_i}{x_j} - \frac13\pderdual[T_H]{x_k}\delta_{ij} \right)k^2 + \OO{k^3},
    \end{aligned} \label{eq:kgf:stress} \\
    q_{iH} = -\frac54\Gamma_2\pder[T_H]{x_i}k + \OO{k^3}. \label{eq:kgf:heat_flux}
\end{gather}

Коэффициент \(\Gamma_3\) входит в выражения температурных напряжений,
но не движущую силу. Для степенного потенциала
\begin{equation}\label{eq:kgf:gamma7_force_simple}
    F_i = -\frac{\Gamma_7}{4p_H} \left(\pder[T_H]{x_j}\right)^2 \pder[T_H]{x_i} k^2 + \OO{k^3},
\end{equation}
поэтому при \(\Gamma_7>0\) нелинейное термострессовое течение возникает в направлении
противоположном градиенту температур.

\subsubsection{Слой Кнудсена и граничные условия}

Используя локальные координаты кнудсеновского слоя~\eqref{eq:eta_definition},
находим, что \(f_K\) подчиняется уравнению
\begin{equation}\label{eq:fK_equation}
    \zeta_in_i\pder[f_K]{\eta} = 2J(f_H,f_K) + J(f_K,f_K)
    - k\zeta_i\left(\pder[\chi_1]{x_i}\pder[f_K]{\chi_1} + \pder[\chi_2]{x_i}\pder[f_K]{\chi_2}\right).
\end{equation}
В нулевом порядке гидродинамическая часть решения является максвеллианом,
удовлетворяющий граничному условию диффузного отражения при
\begin{equation}\label{eq:kgf:boundary_T0}
    T_{H0} = T_B,
\end{equation}
а значит \(f_{K0} = 0\). Используя разложения
\begin{gather}
    f_K = f_{K1}k + f_{K2}k^2 + \cdots, \label{eq:knlayer_expansion}\\
    f_H = \onwall{f_{H0}} + \left[\onwall{f_{H1}} + \eta\onwall{\pder[f_{H0}]{x_i}}n_i \right]k + \cdots, \label{eq:Honwall_expansion}
\end{gather}
где \(\onwall{\cdots}\) означает значение на границе (\(\eta=0\)),
получаем уравнения для \(f_{K1}\) и \(f_{K2}\):
\begin{gather}
    \zeta_in_i\pder[f_{K1}]{\eta} = 2J\left[\onwall{f_{H0}},f_{K1}\right], \label{eq:knlayer1}\\
    \begin{multlined}
        \zeta_in_i\pder[f_{K2}]{\eta} = 2J\left[\onwall{f_{H0}}, f_{K2}\right]
        - \zeta_i\left[\onwall{\pder[\chi_1]{x_i}}\pder[f_{K1}]{\chi_1} + \onwall{\pder[\chi_2]{x_i}}\pder[f_{K1}]{\chi_2}\right] \\
        + 2J\left[\onwall{f_{H1}}+\eta\onwall{\pder[f_{H0}]{x_i}}n_i,f_{K1}\right] + J(f_{K1},f_{K1}).
    \end{multlined}\label{eq:knlayer2}
\end{gather}
Поскольку
\begin{equation}\label{eq:knudsen_linearization}
    2J\left[\onwall{f_{H0}},f_{K1}\right] = \mathcal{L}\left(\frac{f_{K1}}{\onwall{f_{H0}}}\right)\onwall{f_{H0}},
\end{equation}
то опять получаем одномерные задачи в полупространстве \(\eta\in(0,+\infty)\) для линеаризованного
около \(\onwall{f_{H0}}\) уравнения Больцмана с граничными условиями
\begin{equation}\label{eq:fK_bc}
    \onwall{f_{Km}} = f_{Bm} - \onwall{f_{Hm}} \quad (\zeta_in_i>0),
        \quad \lim_{\eta\to\infty} f_{Km} = \onwall{f_{Hm}} \quad (m=1,2,3,\dots).
\end{equation}

Однородное уравнение~\eqref{eq:knlayer1} приводит к следующим
граничным условиям и поправкам кнудсеновского слоя:
\begin{gather}
    \frac1{\sqrt{T_{B0}}}\begin{bmatrix} (u_{jH1} - u_{Bj1}) \\ u_{jK1} \end{bmatrix} \deltann{i}{j} =
        - \onwall{\pder[T_{H0}]{x_j}} \deltann{i}{j}
        \begin{bmatrix} K_1 \\ \frac12 Y_1(\tilde\eta) \end{bmatrix}, \label{eq:kgf:boundary_u1t}\\
    \begin{bmatrix} u_{jH1} \\ u_{jK1} \end{bmatrix} n_j = 0, \label{eq:kgf:boundary_u1n}\\
    \frac{p_{H0}}{T_{B0}}\begin{bmatrix} T_{H1} - T_{B1} \\ T_{K1} \\ T_{B0}^2\rho_{K1} \end{bmatrix} =
        \onwall{\pder[T_{H0}]{x_j}} n_j
        \begin{bmatrix} d_1 \\ \Theta_1(\tilde\eta) \\ p_{H0}\Omega_1(\tilde\eta) \end{bmatrix}, \label{eq:kgf:boundary_T1}
\end{gather}
где \(\tilde\eta = \eta p_{H0}/T_{B0}\).
Неоднородное уравнение~\eqref{eq:knlayer2} приводит к следующим
граничным условиям и поправкам кнудсеновского слоя:
\begin{gather}
    \begin{aligned}
        \frac1{\sqrt{T_{B0}}}
            \begin{bmatrix} (u_{jH2} - u_{Bj2}) \\ u_{jK2} \end{bmatrix}\deltann{i}{j} =
        &- \frac{\sqrt{T_{B0}}}{p_{H0}}\onwall{\pder[u_{jH1}]{x_k}} \deltann{i}{j}n_k
            \begin{bmatrix} k_0 \\ Y_0(\tilde\eta) \end{bmatrix} \\
        - \frac{T_{B0}}{p_{H0}}\onwall{\pderder[T_{H0}]{x_j}{x_k}} \deltann{i}{j}n_k
            \begin{bmatrix} a_4 \\ Y_{a4}(\tilde\eta) \end{bmatrix}
        &- \bar\kappa\frac{T_{B0}}{p_{H0}}\onwall{\pder[T_{H0}]{x_j}} \deltann{i}{j}
            \begin{bmatrix} a_5 \\ Y_{a5}(\tilde\eta) \end{bmatrix} \\
        - \kappa_{jk}\frac{T_{B0}}{p_{H0}}\onwall{\pder[T_{H0}]{x_k}} \deltann{i}{j}
            \begin{bmatrix} a_6 \\ Y_{a6}(\tilde\eta) \end{bmatrix}
        &- \pder[T_{B1}]{x_j} \deltann{i}{j}
            \begin{bmatrix} K_1 \\ \frac12 Y_1(\tilde\eta) \end{bmatrix},
    \end{aligned}\label{eq:kgf:boundary_u2t}\\
    \begin{multlined}
        \frac1{\sqrt{T_{B0}}}
            \begin{bmatrix} (u_{jH2} - u_{Bj2}) \\ u_{jK2} \end{bmatrix} n_j = \\
        - \frac{T_{B0}}{p_{H0}}\left[ \onwall{\pderder[T_{H0}]{x_i}{x_j}}n_i n_j
            + 2\bar\kappa\onwall{\pder[T_{H0}]{x_i}}n_i \right]
            \begin{bmatrix} \frac12\int_0^\infty Y_1(\eta_0)\dd\eta_0 \\
                \frac12\int_\infty^{\tilde\eta} Y_1(\eta_0)\dd\eta_0 \end{bmatrix},
    \end{multlined}\label{eq:kgf:boundary_u2n}\\
    \begin{multlined}
        \frac{p_{H0}}{T_{B0}}
            \begin{bmatrix} T_{H2} - T_{B2} \\ T_{K2} \\ T_{B0}^2\rho_{K2} \end{bmatrix} =
        \onwall{\pder[T_{H1}]{x_j}} n_j
            \begin{bmatrix} d_1 \\ \Theta_1(\tilde\eta) \\ p_{H0}\Omega_1(\tilde\eta) \end{bmatrix} \\
        + \frac{T_{B0}}{p_{H0}}\onwall{\pderder[T_{H0}]{x_i}{x_j}} n_i n_j
            \begin{bmatrix} d_3 \\ \Theta_3(\tilde\eta) \\ p_{H0}\Omega_3(\tilde\eta) \end{bmatrix}
        + \bar\kappa\frac{T_{B0}}{p_{H0}}\onwall{\pder[T_{H0}]{x_i}} n_i
            \begin{bmatrix} d_5 \\ \Theta_5(\tilde\eta) \\ p_{H0}\Omega_5(\tilde\eta) \end{bmatrix},
    \end{multlined}\label{eq:kgf:boundary_T2}
\end{gather}
где \(\bar\kappa/L = (\kappa_1+\kappa_2)/2L\) "--- средняя кривизна граничной поверхности,
главные кривизны \(\kappa_1/L\), \(\kappa_2/L\) принимают отрицательные значения,
когда центр соответствующей кривизны лежит со стороны газа.
Безразмерный тензор кривизны \(\kappa_{ij} = \kappa_1 l_i l_j + \kappa_2 m_i m_j\)
выражается через единичные векторы соответствующих главных направлений \(l_i\) и \(m_i\).

Коэффициент \(a_4\) соответствует термострессовому скольжению второго порядка.
Для газа твёрдых сфер~\cite{Ohwada1992, Takata2015}
\begin{equation}\label{eq:a4_coeff}
    a_4 = 0.0331.
\end{equation}
Поскольку \(a_4>0\), имеет место явление отрицательного термофореза~\cite{Ohwada1992}.
Коэффициенты, стоящие перед \(\bar\kappa\) и \(\kappa_{ij}\),
вычислены недавно~\cite{Takata2015curvature, Takata2015}:
\begin{equation}\label{eq:curvature_coeffs}
    a_5 = 0.23353, \quad a_6 = -1.99878, \quad d_3 = 0.4993, \quad d_5 = 4.6180.
\end{equation}
Функции \(Y_{a4}(\eta)\), \(Y_{a5}(\eta)\), \(Y_{a6}(\eta)\), \(\Theta_3(\eta)\),
\(\Omega_3(\eta)\), \(\Theta_5(\eta)\), \(\Omega_5(\eta)\) убывают также экспоненциально от \(\eta\)
и табулированы в~\cite{Ohwada1992, Sone2002, Sone2007, Takata2015curvature, Takata2015}.

Последние два члена в~\eqref{eq:knlayer2} приводят к дополнительным нелинейным слагаемым
в~\eqref{eq:kgf:boundary_u2t} и~\eqref{eq:kgf:boundary_T2}:
\begin{equation}\label{eq:boundary_nonlinear}
    \frac1{p_{H0}^2}\onwall{\pder[T_{H0}]{x_j}\deltann{i}{j}}\onwall{\pder[T_{H0}]{x_k}n_k}, \quad
    \frac1{p_{H0}^2}\onwall{\pder[T_{H0}]{x_i}n_i}^2,
\end{equation}
однако полное решение этой неоднородной задачи кнудсеновского слоя для газа твёрдых сфер в литературе не представлено.
Для модельного уравнения Крука"--~Веландера численный анализ второго слагаемого выполнен в~\cite{Sone1970}.

\subsubsection{Использование граничных условий старшего порядка}

Уравнения следующего порядка для \(T_{H1}\), \(v_{iH2}\) и \(p_{H3}\) громоздки,
и до настоящего времени не были получены в общей форме для произвольного молекулярного потенциала.
Поэтому численный анализ медленных течений слаборазреженного газа обычно ведут
на основе уравнений КГФ~\eqref{eq:kgf:asymptotic1_u},~\eqref{eq:kgf:asymptotic1_T},~\eqref{eq:kgf:asymptotic2_p}
с граничными условиями~\eqref{eq:kgf:boundary_T0},~\eqref{eq:kgf:boundary_u1t},~\eqref{eq:kgf:boundary_u1n}.
Однако асимптотическое решение можно улучшить,
если привнести в него известные граничные условия следующего порядка.
Например, можно вычислить температурное поле \(T_H = T_{H0} + T_{H1}k + \OO{k^2}\) из уравнения
\begin{equation}\label{eq:asymptotic_T}
    \frac1k\pder[u_{iH}]{x_i} = \frac12\pder{x_i}\left(\Gamma_2\pder[T_H]{x_i}\right) + \OO{k^2},
\end{equation}
получаемого из~\eqref{eq:kgf:asymptotic1_T} и~\eqref{eq:kgf:asymptotic2_T},
при граничном условии
\begin{equation}\label{eq:kgf:boundary_T}
    T_H = T_B + d_1\frac{T_{B0}}{p_{H0}}\onwall{\pder[T_H]{x_j}}n_j k + \OO{k^2},
\end{equation}
получаемого из~\eqref{eq:kgf:boundary_T0} и~\eqref{eq:kgf:boundary_T1}.
Поскольку \(u_{iH2}\) неизвестно, то температурное поле \(T_H\) вычисляется с точностью \(\OO{k}\),
однако на границе с точностью \(\OO{k^2}\).
В~\eqref{eq:kgf:boundary_T} вместо \(\Pder[T_{H0}]{x_j}\) используется производная от \(T_H\),
что позволяет учесть температурный скачок в граничном условии следующего порядка.
Аналогично можно учесть скоростной скачок на границе:
\begin{equation}\label{eq:boundary_u}
    u_{iH} = u_{Bi1}k - \left[ K_1\sqrt{T_{B0}}\onwall{\pder[T_{H0}]{x_j}}
        + k_0\frac{T_{B0}}{p_{H0}}\onwall{\pder[u_{jH}]{x_k}}n_k \right] \deltann{i}{j}k + \OO{k^2}.
\end{equation}
Таким же образом в граничные условия могут быть включены члены
из~\eqref{eq:kgf:boundary_u2t} и~\eqref{eq:kgf:boundary_T2},
содержащие вторую производную от \(T_{H0}\), а также \(\bar\kappa\), \(\kappa_{ij}\).
Граничное условие на нормальную компоненту скорости~\eqref{eq:kgf:boundary_u2n}
несовместимо с уравнением~\eqref{eq:kgf:asymptotic1_u}, поэтому не используется.

Поля \(T_H\) и \(u_{iH}\), полученные вышеописанным способом,
качественно лучше описывают поведение разреженного газа,
поскольку учитывают дополнительные граничные эффекты.
Можно также надеяться, что они количественно лучше аппроксимируют точное решение.

Для вычисления второй производной вдоль нормали от \(T_{H0}\)
удобно воспользоваться преобразованием~\eqref{eq:kgf:asymptotic1_T} и~\eqref{eq:kgf:boundary_u1t}:
\begin{multline*}
    \pderder[T_{H0}]{x_i}{x_j}n_i n_j + 2\bar\kappa\pder[T_{H0}]{x_i}n_i =
        - \pderdual[T_{H0}]{\chi_\alpha} + \pderdual[T_{H0}]{x_k} = \\
    - \pderdual[T_{H0}]{\chi_\alpha} - \frac1{\Gamma_2} \left[
        \Gamma_2'\left(\pder[T_{H0}]{x_i}n_i\right)^2 +
        \left(\Gamma_2'+\frac{2K_1}{\sqrt{T_{H0}}}\right) \left(\pder[T_{H0}]{\chi_\alpha}\right)^2
    \right],
\end{multline*}
где подразумевается суммирование по парам повторяющихся индексов \(\alpha=1,2\),
а также \(|\Pder[\chi_\alpha]{x_i}| = 1\).

\subsubsection{Силы, действующие на обтекаемые тела}

Вследствие неоднородных напряжений в газе, возникает сила второго порядка по \(k\),
действующая на единицу площади обтекаемого тела \(F_{iH2} = -p_{ijH2}n_j\)
С помощью формулы Остроградского"--~Гаусса член второго порядка по \(k\) в~\eqref{eq:kgf:stress}
может быть преобразован во время интегрирования по поверхности тела:
\begin{align*}
    \oint_S \Gamma_3\pderder[T_{H0}]{x_i}{x_j} &n_j\dd{S}
         = \oint_S \pder{x_i} \left( \Gamma_3\pder[T_{H0}]{x_j} \right) n_j\dd{S}
        - \oint_S \Gamma_3'\pder[T_{H0}]{x_i}\pder[T_{H0}]{x_j} n_j\dd{S} \\
        &= \int_V \pder{x_i}\pder{x_j} \left( \Gamma_3 \pder[T_{H0}]{x_j} \right) \dd{V}
        - \oint_S \Gamma_3'\pder[T_{H0}]{x_i}\pder[T_{H0}]{x_j} n_j\dd{S} \\
        &= \oint_S \Gamma_3'\left(\pder[T_{H0}]{x_j}\right)^2 n_i\dd{S}
        + \oint_S \Gamma_3\pderdual[T_{H0}]{x_j} n_i\dd{S}
        - \oint_S \Gamma_3'\pder[T_{H0}]{x_i}\pder[T_{H0}]{x_j} n_j\dd{S},
\end{align*}
где интегрирование производится по всему объему тела \(V\) и всей его поверхности \(S\).
Один из вязкостных членов может быть также преобразован аналогичным образом:
\begin{multline*}
    \oint_S \Gamma_1\pder[u_{jH1}]{x_i} n_j\dd{S}
        = \oint_S \pder{x_i} \left( \Gamma_1 u_{jH1} \right) n_j\dd{S} \\
        = \int_V \pder{x_i}\pder{x_j} \left( \Gamma_1 u_{jH1} \right) \dd{V}
        = \oint_S \left( \Gamma_1 + \Gamma_1'T_{H0} \right) \pder[u_{jH1}]{x_j} n_i\dd{S}.
\end{multline*}
Здесь использовано граничное условие~\eqref{eq:kgf:boundary_u1n} и уравнение непрерывности~\eqref{eq:kgf:asymptotic1_u}.

Таким образом, полная сила, действующая на обтекаемое тело,
\begin{align}\label{eq:kgf:force_general}
    p_{H0} \oint_S F_{i2} \dd{S} =
        &- \oint_S p_{H2}^\dag n_i \dd{S} \notag\\
        &+ \oint_S \left( \Gamma_1 + \Gamma_1'T_{H0} \right) \pder[u_{jH1}]{x_j} n_i \dd{S}
        +  \oint_S \Gamma_1 \pder[u_{iH1}]{x_j} n_j \dd{S} \notag\\
        &- \oint_S \frac{\Gamma_7}2\left(\pder[T_{H0}]{x_j}\right)^2 n_i \dd{S}
        +  \oint_S \Gamma_7\pder[T_{H0}]{x_i}\pder[T_{H0}]{x_j} n_j \dd{S}.
\end{align}
В частности, если тело равномерно нагрето (\(\Pder[T_B]{x_i}=0\)) и покоится (\(u_{Bi}=0\)),
то на него действует сила, состоящая из трёх компонент,
\begin{equation}\label{eq:kgf:force}
    p_{H0} \oint_S F_{i2} \dd{S} =
        - \oint_S p_{H2}^\dag n_i \dd{S} \\
        + \oint_S \Gamma_1\pder[u_{iH1}]{x_j} n_j \dd{S}
        + \oint_S \frac{\Gamma_7}2\left(\pder[T_{H0}]{x_j}\right)^2 n_i \dd{S}.
\end{equation}
Поправка слоя Кнудсена исключена из рассмотрения, поскольку вносит нулевой вклад в значение полной силы.
Это легко доказывается, сдвигая область интегрирования за пределы слоя Кнудсена.

\subsubsection{Электростатическая аналогия}

\begin{figure}[ht]
    \centering
    \subbottom[газ твёрдых сфер: \(s_2=1/2\) и \(s_7=0\).
        Асимптоты: \(\frac34\tau^2\),~\(\tau\to0\) и \(\tau^2\),~\(\tau\to\infty\)]{%
        \centering
        \begin{tikzpicture}
            \begin{loglogaxis}[
                xlabel=\(\tau\),
                ylabel=\(F\)]
                \addplot gnuplot [raw gnuplot, mark=none, color=blue, thick]{
                    set logscale xy;
                    set xrange [1e-2:1e3];
                    plot ((x+1)**.5-1)*((x+1)**1.5-1);
                };
                \addplot gnuplot [raw gnuplot, mark=none, color=black, very thin, dashed]{
                    set xrange [1e0:1e3];
                    plot x**2;
                };
                \addplot gnuplot [raw gnuplot, mark=none, color=black, very thin, dashed]{
                    set xrange [1e-2:1e+2];
                    plot 3./4*x**2;
                };
            \end{loglogaxis}
        \end{tikzpicture}
        \label{fig:snit_force:hs}}
    \subbottom[максвелловские молекулы или модель БГК: \(s_2=1\) и \(s_7=1\).
        Асимптоты: \(2\tau^2\),~\(\tau\to0\) и \(\tau^3\),~\(\tau\to\infty\)]{%
        \centering
        \begin{tikzpicture}
            \begin{loglogaxis}[
                xlabel=\(\tau\),
                ylabel=\(F\)]
                \addplot gnuplot [raw gnuplot, mark=none, color=blue, thick]{
                    set logscale xy;
                    set xrange [1e-2:1e3];
                    plot ((x+1)**1-1)*((x+1)**2-1);
                };
                \addplot gnuplot [raw gnuplot, mark=none, color=black, very thin, dashed]{
                    set samples 2;
                    set xrange [1e0:1e3];
                    plot x**3;
                };
                \addplot gnuplot [raw gnuplot, mark=none, color=black, very thin, dashed]{
                    set samples 2;
                    set xrange [1e-2:1e+1];
                    plot 2*x**2;
                };
            \end{loglogaxis}
        \end{tikzpicture}
        \label{fig:snit_force:maxwell}}
    \caption{Зависимость силы притяжения двух тел \(F\) от разности температур \(\tau=T_2-T_1\) при \(T_1=1\).
        Тонкие пунктирные линии соответствуют асимптотам.}
    \label{fig:snit_force}
\end{figure}

Сила взаимодействия между равномерно нагретыми телами оказывается подобна электростатической.
Впервые на это обратили внимание М.\,Н.~Коган, В.\,С.~Галкин и О.\,Г.~Фридлендер,
рассмотрев линейное приближение~\cite{Kogan1976}.
Их результат можно естественным образом обобщить в нелинейной постановке.

Если ограничится рассмотрением степенного потенциала,
то коэффициент теплопроводности \(\Gamma_2\propto T^s\) (см.~\eqref{eq:kgf:Gammas}),
а соответствующее уравнение теплопроводности нелинейно,
которое, однако, является линейным уравнением Лапласа для \(T^{1+s}\).
Здесь и далее мы для простоты предполагаем \(s>0\).
В соответствии с электростатической теорией можно ввести аналог заряда
\begin{equation}\label{eq:charge_ell}
    e_a \eqdef T_a^s \oint_{S_a} \pder[T]{x_i}n_i\dd{S} = C'_{ab} T_b^{1+s}, \quad \sum_a e_a = 0.
\end{equation}
\(T^{1+s}\) можно считать потенциалом, \(C_{aa}\) коэффициентами ёмкости,
а \(C_{ab}\,(a\neq b)\) коэффициентами электростатической индукции.
Несложно также построить аналог энергии
\begin{equation}\label{eq:energy_ell}
    U \eqdef \int \left[ \frac{\Gamma_7(T)}2\left(\pder[T]{x_i}\right)^2 - p \right]\dd{V} =
    \gamma_7 \sum_a T_a^{2s} \oint_{S_a} \pder[T]{x_i}n_i\dd{S} =
    C_{ab} T_a^s T_b^{1+s},
\end{equation}
тогда действующая сила находится как вариационная производная
\begin{equation}\label{eq:force_ell}
    F^a_i = \left(\varder[U]{r^a_i}\right)_T = \pder[C_{ab}]{r^a_i} T_a^s T_b^{1+s}.
\end{equation}
В частности, для двух тел с температурами \(T_1\) и \(T_2\) можно записать
\begin{gather}
    e = C \left( T_2^{1+s} - T_1^{1+s} \right), \label{eq:charge_2bodies}\\
    U = C \left( T_2^s - T_1^s \right)\left( T_2^{1+s} - T_1^{1+s} \right), \label{eq:energy_2bodies}\\
    F_i = \pder[C]{r^a_i} \left( T_2^s - T_1^s \right)\left( T_2^{1+s} - T_1^{1+s} \right), \label{eq:force_2bodies}
\end{gather}
поскольку \(e_1+e_2=0\) и \(e_1=e_2=0\) при \(T_1=T_2\). \(C\) "--- аналог электростатической ёмкости.
При \(T=1+o(1)\) задача сводится к линейной, где сила притяжения \(F \propto (T_2-T_1)^2\).
На рис.~\ref{fig:snit_force} показаны соответствующие зависимости для некоторых частных случаев.

\subsubsection{Континуальный предел}

В классической гидродинамике уравнения Навье"--~Стокса (\(\Gamma_7=0\))
с неподвижными границами (\(v_{Bi}=0\)) и условиями без скольжения (\(K_1=0\))
приводят к нулевому полю \(v_{iH1}=0\) и к уравнению теплопроводности
\begin{equation}\label{eq:heat_equation}
    \pder{x_i}\left(\sqrt{T_{H0}}\pder[T_{H0}]{x_i}\right) = 0.
\end{equation}
В общем случае корректное распределение температур в континуальном пределе (\(k\to0\))
находится из уравнений КГФ с соответствующими граничными условиями.
Оно будет совпадать с решением~\eqref{eq:heat_equation} только для узкого класса задач, где \(v_{iH1}=0\).

В континуальном мире (\(k=0\)) не существует величин \(u_{iH1}\) и \(p_{H2}^\dag\),
тем не менее инфинитезимальное поле скоростей \(v_i\) конечным образом влияет на \(T\).
Такое асимптотическое поведение получило название призрак"=эффекта (ghost effect)~\cite{Sone2002, Sone2007}.

\subsection{Одномерные течения при конечных чисел Маха}

Как было указано выше, построение асимптотического решения при конечных числах Маха в общем случае
дополнительно требует сращивания вязких пограничных слоёв с решением уравений Эйлера.
Рассмотрение одномерных течений, описываемых уравнением
\begin{equation}\label{eq:1d}
    \zeta_y\pder[f]{y} = \frac1kJ(f,f), \quad \int\zeta_yf\dzeta = \int\zeta_zf\dzeta = 0,
\end{equation}
в этом смысле существенно проще, поскольку вырождаются
\begin{itemize}
    \item члены, содержащие кривизну слоя Прандтля,
    \item область невязкого течения.
\end{itemize}

Вплоть до членов второго порядка по \(k\) гидродинамическая часть решения одномерных задач
совпадает с уравнениями Навье"--~Стокса
\begin{gather}
    \der{y}\left(\Gamma_1\der[v_H]{y}\right) = \OO{k^2}, \label{eq:1d:asymptotic_v}\\
    \Gamma_1\left(\der[v_H]{y}\right)^2
        + \frac54\der{y}\left(\Gamma_2\der[T_H]{y}\right) = \OO{k^2}. \label{eq:1d:asymptotic_T}
\end{gather}
Давление \(p_H\) константно вплоть до второго порядка.
Если масса газа \(M = \int\rho\dx\) постоянна, то давление \(p_H\) может быть получено из равенства
\begin{equation}\label{eq:1d:pH}
    p_H \int\frac{\dx}{T_H} = M + \OO{k^2},
\end{equation}
поскольку \(\rho = \rho_H + \rho_{K1}k + \OO{k^2}\) и \(p_H = \rho_H T_H\).
\(\rho_{K1}\) не фигурирует в~\eqref{eq:1d:pH}, так как \(\int\rho_{K1}\dx = \OO{k}\).
Гидродинамическое давление второго порядка \(p_{H2}\) вычисляется из уравнения
\begin{equation}\label{eq:1d:pH2}
    \frac{3p_H}{2}\der[p_{H2}]{y}
        + \der{y}\left[ \Gamma_3\derdual[T_H]{y} + \Gamma_7\left(\der[T_H]{y}\right)^2 \\
            + (\Gamma_8-2\Gamma_9)\left(\der[v_H]{y}\right)^2 \right] = \OO{k}.
\end{equation}

Если газ ограничен неподвижными пластинами с граничными условиями диффузного отражения,
то справедливо следующее асимптотическое решение
\begin{gather}
    v = v_H - \sum_a\frac{Y_0(\tilde\eta_a)}{p_H}\left(T_H\pder[v_H]{y}\right)_ak + \OO{k^2}, \label{eq:1d:v}\\
    T = T_H - \sum_a\frac{\Theta_1(\tilde\eta_a)}{p_H}\left(T_H\pder[T_H]{y}\right)_ak + \OO{k^2}, \label{eq:1d:T}\\
    p_{xy} = -\Gamma_1\pder[v_H]{y}k + \OO{k^3}, \label{eq:1d:p_xy}\\
    \begin{aligned}
        q_x = \sum_a H_A(\tilde\eta_a) &\left(T_H\pder[v_H]{y}\right)_ak + q_{xK2}k^2 \\
        &+ \frac{T_H}{p_H}\left( \frac{\Gamma_3}2 \pderdual[v_H]{y}
            + 4\Gamma_{10} \pder[T_H]{y}\pder[v_H]{y} \right)k^2 + \OO{k^3},
    \end{aligned}\label{eq:1d:q_x}\\
    q_y = -\frac54\Gamma_2\pder[T_H]{y}k + q_{yK2}k^2 + \OO{k^3}, \label{eq:1d:q_y}\\
    p = p_H - \mathcal{P}_\eta k + (p_{H2} + p_{K2})k^2 + \OO{k^3}, \label{eq:1d:p}\\
    p_{xx} - p = -\frac12 \mathcal{P}_\eta k
        + \big[ 2(\Gamma_8+\Gamma_9)\mathcal{P}_u - \mathcal{P}_T + p_{xxK2}-p_{K2}\big]k^2
        + \OO{k^3}, \label{eq:1d:Pxx}\\
    p_{yy} - p = \mathcal{P}_\eta k
        + \big[ 2(\Gamma_8-2\Gamma_9)\mathcal{P}_u + 2\mathcal{P}_T + p_{yyK2}-p_{K2} \big]k^2
        + \OO{k^3}, \label{eq:1d:Pyy}\\
    p_{zz} - p = -\frac12 \mathcal{P}_\eta k
        + \big[ 2(\Gamma_9-2\Gamma_8)\mathcal{P}_u - \mathcal{P}_T + p_{zzK2}-p_{K2} \big]k^2
        + \OO{k^3}, \label{eq:1d:Pzz}
\end{gather}
где
\begin{gather}\label{eq:mathcal_P_def}
    \mathcal{P}_\eta = \sum_a\left(\Omega_1(\tilde\eta_a)+\Theta_1(\tilde\eta_a)\right)\left(\pder[T_H]{y}\right)_a, \\
    \mathcal{P}_T = \frac1{3p_H}\left[\Gamma_3 \pderdual[T_H]{y} + \Gamma_7\left(\pder[T_H]{y}\right)^2\right], \quad
    \mathcal{P}_u = \frac1{3p_H}\left(\pder[v_H]{y}\right)^2,
\end{gather}
а граничные условия
\begin{equation}\label{eq:1d:bc}
    v_H = v_{Ba} - k_0\frac{T_H}{p_H}\pder[v_H]{y}k + \OO{k^2}, \quad
    T_H = T_{Ba} + d_1\frac{T_H}{p_H}\pder[T_H]{y}k + \OO{k^2}.
\end{equation}
Величины с индексом \(a\) вычисляются на соответствующей границе.
Функции кнудсеновского слоя (\(Y_0\), \(\Theta_1\), \(H_A\), \(\Omega_1\)) убывают экспоненциально,
\begin{equation}\label{eq:nonlinear_knudsen_functions}
    \tilde\eta_a = \frac{p_H}{T_H}\eta_a, \quad \eta_a = \frac{|y-y_a|}{k},
\end{equation}
где \(y_a\) "--- координаты пластин.

Для степенного потенциала
\begin{equation}\label{eq:gammas_nonlinear}
    \Gamma_8 = \gamma_8 T^{2s},
        \quad \Gamma_9 = \gamma_9 T^{2s},
        \quad \Gamma_{10} = \gamma_{10} T^{2s-1},
\end{equation}
а для модели твёрдых сфер
\begin{equation}\label{eq:gammas_nonlinear_hs}
    \gamma_8 = 1.495941968, \quad \gamma_9 = 1.636073459, \quad \gamma_{10} = 2.449780.
\end{equation}
Алгоритм вычисления транспортных коэффициентов~\eqref{eq:gammas_nonlinear_hs} через приближённое решение
соответствующих интегральных уравнений изложен в приложении~\ref{app:gammas}.

\clearpage
