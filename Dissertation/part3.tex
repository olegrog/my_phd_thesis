\chapter{Классические задачи молекулярной газодинамики} \label{chapt:problems}

%%% Линеализованные течения
При отсутствии значительных температурных градиентов медленные течения описываются
линеаризованным уравнением Больцмана. Первые аналитические решения отдельных задач на его основе
для модельных уравнений появились в 1960-х годах в трудах П.~Веландера, Д.\,Р.~Виллиса, К.~Черчиньяни и др.
Непосредственные численные решения высокой точности линеаризованного уравнения Больцмана
для газа твёрдых сфер были получены в 1989--92 годах усилиями Киотской группы.
%В трудах Киотской группы реализована масштабная программа высокоточного численного анализа классических задач молекулярной газодинамики
%на основе линеаризованного уравнения Больцмана для газа твёрдых сфер.
%%%%%%%%% нужны цитирования

% Вот тут и надо про солверы!
%    В рамках платформы OpenFOAM разработан солвер уравнений КГФ
%    на основе метода конечных объёмов и модифицированного алгоритма SIMPLE~\cite{Rogozin2014}.

\section{Задача Куэтта} \label{sect:couette}

\subsection{Решение на равномерной сетке}

\begin{figure}
    \centering
    \includegraphics{couette_heat/graph}
    \caption{Зависимость сдвигового напряжения от числа Кнудсена}\label{fig:couette:shear}
\end{figure}

\begin{figure}
    \centering
    \includegraphics{couette_heat/flow}
    \caption{Зависимость потока массы через половину сечения от числа Кнудсена}\label{fig:couette:flow}
\end{figure}

\begin{figure}
    \centering
    \includegraphics{couette_heat/qflow}
    \caption{Зависимость потока тепла через половину сечения от числа Кнудсена}\label{fig:couette:qflow}
\end{figure}

На рис.~\ref{fig:couette:shear} наблюдается хорошое совпадение результатов.
Однако для бесстолкновительного газа заметно превышение \(\hat{p}_{xy}\) на 0.61\%,
что может быть обусловлено только ошибкой дискретной аппроксимации функции распределения по скоростному пространству.
При интерполяции полученной кривой на интервале \(\Kn\in[0,0.2]\) асимптотическим решением
можно произвести подбор параметров нелинейным методом наименьших квадратов.
С его помощью в области континуального описания газа определяется отклонение
от коэффициентов вязкости (\(+0.83\%\)) и скольжения (\(-1.2\%\)).
Если все полученные значения нормировать на точное значение для бесстолкновительного газа (разделить на 1.0061),
то погрешность в коэффициенте вязкости снижается до \(+0.22\%\), что является уже приемлимым результатом.

На рис.~\ref{fig:couette:flow},~\ref{fig:couette:qflow} для сильно разреженного газа наблюдаются
значительные отклонения, природа которых пока не определена.

Для задачи переноса тепла проанализируем зависимость теплового потока от числа Кнудсена~\ref{fig:heat}.

Для бесстолкновительного газа
\[ \frac{\hat{q}_x}{\hat{T}_1-\hat{T}_2} = -\frac1{\sqrt{\pi}} \]
При небольших \(\Kn\) справедливо асимптотическое решение
\[ \frac{\hat{q}_x}{\hat{T}_1-\hat{T}_2} = -\frac{\hat{\lambda}\Kn}{1+\sqrt{\pi}d_1\Kn}. \]
Коэффициенты теплопроводности \(\hat{\lambda}\) и скачка температуры на стенке \(d_1\) для модели твёрдых сфер равны~\cite{Sone2007}
\[ \hat{\lambda} = 2.129475, \; d_1 = 2.4001. \]
Отбросив знаменатель, получим гидродинамическое решение \(\hat{q}_x = -\hat{\lambda}\Kn(\hat{T}_1-\hat{T}_2)\).
Численные значения точного решения табулированы в~\cite{Sone2007}.

\begin{figure}
    \centering
    \includegraphics{couette_heat/heat}
    \caption{Зависимость теплового потока от числа Кнудсена}\label{fig:heat}
\end{figure}

Как и в задаче о течении Куэтта, на рис.~\ref{fig:heat} видна хорошая сходимость к точному решению,
однако одновременно наблюдаются те же проблемы: завышение значений теплопотока на \(0.63\)\%
для бесстолкновительного газа и значительная ошибка в коэффициенте теплопроводности (\(+2.2\%\)).
По всей видимости, последняя объясняется неточной аппроксимацией угла разлёта сталкивающихся молекул,
в частности, существованием минимально разрешимого угла.

\subsubsection{Исследование погрешности}

Логичным представляется исследование зависимости указанных погрешностей от числа интерполирующих узлов
скоростного пространства. Будем анализировать ошибки аппроксимации задачи переноса тепла.
Для этого коэффициент теплопроводности вычислялся локально в точке \(x_0=\hat L/2\) по формуле
\[ \hat{q_x}(x_0) = -\hat{\lambda}\frac{\dd\hat T}{\dd x}\bigg|_{x=x_0}. \]

Соответствующие графики в логарифмическом масштабе представлены на рис.~\ref{fig:error}.
Наклон прямых говорит о квадратичной зависимости:
\[ \epsilon \propto N_\Omega^{-2} \propto V_\Omega^{-2/3}, \]
где \(N_\Omega\) "--- число узлов на радиусе скоростной сетки, \(V_\Omega\) "--- её объём.

При \(N_\Omega = 40\) точность вычисления теплового потока повышается до \(0.1\%\),
а коэффициента теплопроводности только до \(0.5\%\).

\begin{figure}
    \centering
    \includegraphics{couette_heat/error}
    \caption{
        Зависимость от числа узлов на радиусе скоростной сетки погрешностей вычисления
        теплопотока для бесстолкновительного газа и коэффициента теплопроводности для слаборазреженного газа.
    }\label{fig:error}
\end{figure}


%============================================================================================================================
\subsection{Решение на неравномерной сетке}


%\newpage
%============================================================================================================================
\section{Течение между пластинами с синусоидальным распределением температур} \label{sect:sone_bobylev}

%\newpage
%============================================================================================================================
\section{Течение между некоаксиальными цилиндрами} \label{sect:noncoaxial}

%\newpage
%============================================================================================================================
\section{Течение между эллиптическими цилиндрами} \label{sect:elliptic}

\clearpage
