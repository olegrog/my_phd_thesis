\chapter{Численное решение уравнения Больцмана} \label{chapt:numerics}

\section{Обзор вычислительных методов} \label{sect:review}

Огромное множество исследований посвящено численному решению уравнения Больцмана.
Среди них можно выделить три магистральных направления в зависимости от способа аппроксимации
функции распределения скоростей:
\begin{itemize}
    \item \emph{методы прямого статистического моделирования} (\emph{ПСМ})
    строятся на основе некоторого случайного процесса марковского типа,
    способного аппроксимировать больцмановскую динамику;
    \item \emph{методы дискретных скоростей} подразумевают фиксированный набор доступных молекулярных скоростей;
    \item \emph{проекционные методы} используют разложение по базису в определённом функциональном пространстве.
\end{itemize}
Ввиду широкого распространения методов ПСМ, распространено разделение численных методов
на стохастические и детерминистические, однако автору представляется такая классификация неконструктивной,
поскольку стохастические приёмы, привносящие численный шум в решение, весьма универсальны
и могут применяться практически для всех стандартных методов.
Эти приёмы в некоторых случаях позволяют существенно снизить алгоритмическую сложность чисто детерминистического метода,
а возникающие флуктуации решения способны в какой-то степени отражать численную невязку.
В частности, для многомерного интегрирования эффективны теоретико"=числовые методы~\cite{Korobov1963}
(квази-Монте Карло~\cite{Dick2013}).
Для методов дискретных скоростей в своё время предложены специальные ускорительные стохастические
процедуры~\cite{Buet1996, Platkowski2000}.

В результате более чем полувекового развития численных методов решения уравнения Больцмана,
было выделено три основных свойства, строгое выполнение которых делает вычислительный алгоритм
надёжным и высокоэффективным:
\begin{itemize}
    \item сохранение массы, импульса и энергии (\emph{консервативность});
    \item выполнение \(\mathcal{H}\)-теоремы (\emph{энтропийность});
    \item \emph{положительность} функции распределения.
\end{itemize}
Первые два свойства гарантируют, что на бесконечности (\(t\to\infty\)) численное решение
пространственно"=однородного уравнения Больцмана будет точно совпадать с аналитическим,
а положительность решения, как правило, является необходимым условием устойчивости метода.
Нарушение даже всех перечисленных свойств возможно, но только при использовании достаточно подробных сеток
во всем фазовом пространстве, что практически возможно только для простейших задач,
обладающих дополнительными симметриями. В таком случае степень нарушения консервативности
может служить, например, апостериорной оценкой отклонения численного решения от истинного.

\subsection{Стохастические методы}

В тех случаях, когда построение прямых численных методов сталкивается с существенными трудностями,
ПСМ нередко позволяет достичь приемлимой точности малой кровью.
В настоящее время для численного решения существенно нелинейных задач наиболее распространён
метод Бёрда (DSMC), впервые предложенный им на основе общих физических соображений~\cite{Bird1963}.
Современные его реализации используют улучшенные алгоритмы выбора сталкивающихся частиц,
разработанные позже самим Г.~Бёрдом (схема без счётчика)~\cite{Bird1989},
а также М.\,С.~Ивановым и С.\,В.~Рогазинским (схема мажорантной частоты)~\cite{Ivanov1988}.

%%% стохастические модели разреженного газа
\emph{Стохастические модели} разреженного газа впервые рассмотрены М.\,А.~Леонтовичем
ещё в 1935 году~\cite{Leontovich1935}, но только в 1992 году В.~Вагнер показал сходимость
полумарковского метода Бёрда к уравнению Больцмана~\cite{Wagner1992},
основываясь на результатах А.\,В.~Скорохода~\cite{Skorokhod1983} и С.\,Н.~Смирнова~\cite{Smirnov1989}.
До этого времени предлагались альтернативные модели, получаемые эвристически из уравнения Больцмана.
В частности, В.\,Е.~Яницкий и О.\,М.~Белоцерковский разработали стохастический метод~\cite{Yanitskij1975}
на основе строго марковского процесса эволюции модели Каца"--~Леонтовича~\cite{Kac1965, Kac1973},
которая асимптотически эквивалента уравнению Больцмана, однако строгое доказательство~\cite{Babovsky1989}
сходимости к уравнению Больцмана было получено только для метода Нанбу"--~Бабовского~\cite{Nanbu1980, Babovsky1986}.

%%% недостатки DSMC и пути решения
К основным недостаткам методов ПСМ относятся высокий уровень статистического шума
и значительный рост вычислительных затрат при стремлении к континуальному пределу,
возникающий из-за нарушения энтропийности для распределений близких к максвелловским.
%%% а) линейные течения
В последние годы предлагаются различные методы уменьшения дисперсии (variance reduction) для слабо возмущённых течений
(information preservation~\cite{Fan2001, Sun2002}, time-relaxed~\cite{Pareschi2001, Pareschi2011},
low-variance~\cite{Hadji2005, Hadji2007, Hadji2011}).
Основой этих \emph{гибридных} подходов является априорное представление о функции распределения
и применение статистического моделирования только к возмущённой части решения (алгебраическая декомпозиция).
Обобщение девиационного стохастического подхода, допускающего частицы c отрицательным весом,
на нелинейное уравнение Больцмана приводит к значительным вычислительным трудностям~\cite{Wagner2008}.
Активно развиваются гибридные методы, использующие классическую схему Бёрда
(moment-guide~\cite{Dimarco2011, Dimarco2013}, convex combination~\cite{Dimarco2008, Caflisch2016}).
%%% б) гиперзвуковые течения
Особую сложность представляют также течения с высоким перепадом плотности (например, гиперзвуковые),
поскольку очень малая часть ансамбля модельных частиц попадает в области наиболее разреженного газа (например, донную).
С вычислительной точки зрения, это многомасштабные задачи с широким диапазоном чисел Кнудсена.
Стохастический метод взвешенных частиц позволяет адаптировать уровень дисперсии посредством
их деления и аннигиляции, однако ценой высокой сложности алгоритма~\cite{Rjasanow1996, Rjasanow2005}.
Несмотря на то что в широком круге задач метод Бёрда позволяет достичь инженерной точности,
множество тонких эффектов остаются за гранью его реальной разрешающей способности.

\subsection{Методы дискретных скоростей}

%%% методы дискретных скоростей
Методы дискретных скоростей (дискретных ординат) восходят к работам 40-х годов Нобелевского лауреата С.~Чандрасекара
в области теории излучения~\cite{Chandrasekhar1950}, ещё до появления методов ПСМ.
Первые значительные успехи, связанные с применением этого подхода к численному решению уравнения Больцмана,
были получены в США А.~Нордсиком и Б.~Хиксом~\cite{Nordsieck1966, Nordsieck1970}.
Они использовали метод Монте-Карло для вычисления пятимерного интеграла столкновений
и простейшую коррекцию функции распределения для получения устойчивой численной схемы.
Начиная с 1965 года, после ввода в эксплуатацию БЭСМ-6, \emph{метод Хикса"--~Йена"--~Нордсика}~\cite{Nordsieck1966, Yen1984}
активно развивался в Вычислительном центре АН СССР (Ф.\,Г.~Черемисин, В.\,В.~Аристов и др.).
На первых ЭВМ основную трудность представляли ограничения в объёме памяти~\cite{Tcheremissine1970}.
Операторное расщепление уравнения Больцмана и полиномиальная коррекция,
обеспечивающая консервативность на макроскопическом уровне,
существенно повысили надёжность метода~\cite{Tcheremissine1980}.
Он успешно применялся во многих прикладных областях, однако в тех случаях, когда вклад,
вносимый полиномиальной коррекцией, становился значительным,
достижение приемлимой точности сильно осложнялось.

\subsubsection{Модели дискретного газа}

%%% модели дискретных скоростей на конечных решётках
Одновременно с численными методами бурное развитие получили математические \emph{модели дискретного газа}.
Простейшая такая модель, содержащая только два возможных вектора скорости,
была рассмотрена ещё в фундаметальном труде Т.~Карлемана~\cite{Carleman1957}.
Дж.~Бродуэлл в 1964 году использовал шесть и восемь скоростей для численного анализа простейших задач
разреженного газа~\cite{Broadwell1964shock, Broadwell1964shear}.
Его успех вызвал интерес у французских математиков Р.~Гатиньоль~\cite{Gatignol1975} и А.~Кабанна~\cite{Cabannes1980},
впервые представившие систематическую теорию газа дискретных скоростей.
Среди пионеров формальной теории отметим также С.\,К.~Годунова и У.\,М.~Султангазина~\cite{Sultangazin1971}.


%%% модели дискретных скоростей на равномерной решётке
Д.~Гольдштейн, Б.~Стёртевант и Дж.~Бродуелл первыми использовали модель дискретного газа
на решётке с постоянным шагом~\cite{Goldstein1989}.
Важным её преимуществом является присущая на микроскопическом уровне консервативность и энтропийность.
В 1995 году А.~Пальчевский, Ж.~Шнайд\'{е}р и А.\,В.~Бобылев показали,
что последовательность таких моделей сходится к уравнению Больцмана при стремлении шага решётки к нулю (\(h\to0\)),
однако порядок сходимости оказался не больше \(1/14\)~\cite{Palczewski1995, Palczewski1997}.
Только в 2004 году их результат удалось обобщить для двумерного газа~\cite{Fainsilber2006},
для которого сходимость вообще носит логарифмический характер \(\OO{\ln^p h^{-1}}\).
Рассмотрев интеграл столкновения в координатах Карлемана\footnote{
    В прямоугольных координатах Карлемана~\eqref{eq:Carleman_representation} задача о сходимости
    модели дискретного газа оказывается существенно проще, чем исследование равномерного покрытия
    столкновительной сферы, поскольку сводится к решению линейных диофантовых уравнений.
}, В.~Панфёров и А.~Гейнц улучшили сходимость,
но лишь вплоть до первого порядка~\cite{Panferov2002}.
Ж.~Шнайд\'{е}р совместно с Ф.~Рожье~\cite{Rogier1994} и Ф.~Мишель~\cite{Michel2000}
на основе последовательностей Фарея построили модель дискретного газа на конечной решётке
с общим числом узлов \(\OO{N^3}\) и сходимостью \(\OO{N^{-3/2}\ln{N} + h^2N^2}\).
Наконец, С.~Мишлер разработал метод доказательства сходимости конечно-разностных схем
с операторным расщеплением для дискретного газа к решениям Ди-Перна"--~Лионса~\cite{Mischler1997}.

\subsubsection{Размазывание столкновительного процесса}

Основная трудность классических моделей дискретного газа связана с малым количеством
допустимых пар разлётных скоростей для выбранной столкновительной пары.
Для построения консервативной схемы второго порядка точности необходимо так или иначе
привнести дополнительную свободу в столкновительный процесс.
Другими словами, ослабить его или размазать (to mollify).

К.~Бюе, С.~Кордье и П.~Дегон предложили несколько таких регуляризационных подходов,
сохраняющих консервативность численной схемы в слабой форме (для столкновительного оператора целиком)~\cite{Buet1998}.
Один из них основан на размазывании столкновительной сферы;
второй, напротив, при локальном сохранении импульса и энергии, допускает нарушения инвариатности массы.
Достижение \emph{макроскопической} консервативности в этих подходах создаёт зависимость
дискретного столкновительного оператора от функции распределения или по крайней мере отдельных её моментов.
Такое требование существенно ограничивает эффективность численной реализации.
Х.~Бабовски построил простейшую схему с консервативностью на \emph{мезоскопическом} уровне
(для всей столкновительной сферы)~\cite{Babovsky1998}, его подход позже развил Д.~Гёрш~\cite{Goersch2002}.
Наконец, в 1997 году Ф.\,Г.~Черемисин предложил новый класс методов дискретных скоростей,
основанных на консервативном проецировании разлётных частиц~\cite{Tcheremissine1997, Tcheremissine1998}\footnote{
    Интересно отметить, что все описанные подходы были опубликованы в одном специальном издании журнала
    \textit{Computers \& Mathematics with Applications (Vol.~35, No.~1/2)} по приглашению К.~Черчиньяни и Р.~Иллнера,
    что явилось в некотором смысле диалектическим ответом на результат Пальчевского"--~Шнайд\'{е}ра"--~Бобылева.}.
\emph{Микроскопическая} консервативность, достигнутая Ф.\,Г.~Черемисиным, позволяет построить наиболее эффективную
численную схему и может быть интерпретирована как проекционная процедура Петрова"--~Галёркина,
в которой столкновительные инварианты образуют ортогональную оболочку.
Кроме того, Специальная процедура интерполяции функции распределения обеспечивает
энтропийность метода~\cite{Tcheremissine2000}. Поэтому такой метод будем называть
\emph{консервативным проекционно"=интерполяционным методом дискретных скоростей} (\emph{KПИМДС}).

\subsection{Проекционные методы}

%%% введение
Дополнительное априорное знание о функции распределения в отдельных классах задач может служить
основой для построения более эффективных, но менее универсальных численных методов.
Проекционные методы имеют \emph{экспоненциальную сходимость} по отношению к размерности аппроксимационного пространства,
однако в общем случае весьма затруднительно добиться консервативности и положительности.
Более того, разрывные решения представляют для них особую сложность в связи с явлением Гиббса.
Последнюю проблему можно обойти, используя тот факт, что для короткодействующих потенциалов
интеграл столкновений может быть записан в форме \(J = J^+ - \nu_c f\), где
\begin{equation}\label{eq:gain_term}
    J^+(f,f) = \int f'f'_* B\dd\Omega(\boldsymbol\nu)\dzeta_*
\end{equation}
и \emph{частота столкновений}
\begin{equation}\label{eq:collision_frequency}
    \nu_c(f) = \int f_*B\dd\Omega(\boldsymbol\nu)\dzeta_*
\end{equation}
не содержат разрывов.
Гладкость \emph{интеграла обратных столкновений}~\eqref{eq:gain_term} впервые показана П.-Л.~Ли\'{о}нсом~\cite{Lions1994},
позже его результат уточнён Б.~Веннбергом~\cite{Wennberg1997}, К.~Муо и С.~Виллан\'{и}~\cite{Mouhot2004}.
Частота столкновений \(\nu_c(f)\) "--- также гладкая функция,
поскольку~\eqref{eq:collision_frequency} является свёрткой с регулярной функцией.

%%% многочлены Эрмита
Проекционные методы вычисления интеграла столкновений, по-видимому, берут своё начало с классического
моментного метода Х.~Грэда~\cite{Grad1949}, который представляет собой метод Галёркина
с многочленами \emph{Эрмита} в качестве базиса и локальным распределением Максвелла в качестве весовой функции.
Если Грэд ограничился в основном рассмотрением первых тринадцати моментов функции распределения,
то А.~Шор\'{е}н использовал для численного расчёта профиля ударной волны старшие члены эрмитового базиса~\cite{Chorin1972}.
Из-за сильно возрастающей сложности метод Эрмита"--~Галёркина применяется преимущественно для
аппроксимации линеаризованного уравнения Больцмана~\cite{Gobbert2007}.

%%% многочлены Фурье
А.\,В.~Бобылев показал, что столкновительный оператор в \emph{Фурье}"=базисе принимает
особенно простой вид для максвелловских молекул~\cite{Bobylev1975}.
Кроме того, с вычислительной точки зрения такой базис привлекален
благодаря алгоритмам быстрого преобразования (БПФ).
Первые численные результаты (анализа процессов релаксации высоких моментов функции распределения)
на основе метода Фурье"--~Галёркина для максвелловских молекул
принадлежат Ю.\,Н.~Григорьеву и А.\,Н.~Михалицыну~\cite{Grigoriev1983}.
Н.\,Х.~Ибрагимов и С.\,В.~Рязанов использовали представление Фурье
для построения консервативного метода второго порядка для произвольного молекулярного потенциала,
оставаясь в рамках модели дискретных скоростей~\cite{Ibragimov2002}.
Основная идея метода, предложенная А.\,В.~Бобылевым и С.\,В.~Рязановым~\cite{Rjasanow1999},
состоит в использовании \emph{карлемановского представления} столкновительного оператора
\begin{multline}\label{eq:Carleman_representation}
    J(f,f) = \int_{\mathbb{R}^d\times\mathbb{R}^d} B\left(|\bx+\by|, -\frac{\bx\cdot(\bx+\by)}{|\bx||\bx+\by|}\right)
        \frac{2^{d-1}}{|\bx+\by|^{d-2}} \delta(\bx\cdot\by) \\
    \big( f(\bzeta+\by)f(\bzeta+\bx) - f(\bzeta+\bx+\by)f(\bzeta) \big) \dd\bx\dd\by,
\end{multline}
позволяющего избежать интегрирования по сфере.

%%% недостатки метода
Общая теория метода Фурье"--~Галёркина была развита Л.~Парески совместно с Б.~Пертамом~\cite{Pareschi1996}.
и Дж.~Руссо~\cite{Pareschi2000method, Pareschi2000stability}.
Несмотря на проекционную точность, метод обладает существенными недостатками:
\begin{itemize}
    \item периодизация скоростного пространства,
    \item неконсервативность по импульсу и энергии,
    \item нарушение положительности.
\end{itemize}
Первая проблема частично решается из-за гауссовского затухания функции распределения,
однако чрезмерное увеличение периода приводит к значительным осцилляциям старших мод.
Вторая проблема частично решается при использовании достаточного количества мод,
поскольку консервативность гарантируется с экспоненциальной точностью.
По третьей проблеме важный результат получен Ф.~Фильбе и К.~Муо.
Для возмущённого столкновительного оператора, им удалось доказать асимптотическую стабильность
спектральной аппроксимации, не сохраняющей положительность~\cite{Filbet2011}.
%%% снижение алгоритмической сложности
В общем случае фурье"=образ столкновительного ядра не представим в виде свёртки,
поэтому алгоритмы БПФ не могут быть применены.
К.~Муо и Л.~Парески показали, что для модели твёрдых сфер при \(d=3\) это возможно
в рамках карлемановского представления~\cite{Pareschi2006}.
Таким образом им удалось понизить вычислительную сложность метода Фурье"--~Галёркина
с \(\OO{N^{2d}}\) до \(\OO{M^{d-1}N^d\ln{N}}\), где \(N\) "--- число мод на каждой координатной оси,
а \(M\) "--- число допустимых углов отклонений в столновительном процессе.
Леи~Ву, Дж.~Риз и Йонг-Хао~Жанг обобщили их результат для более общей формы столкновительного ядра~\cite{Wu2014}.

%%% многочлены Сонина
Все перечисленные недостатки метода Фурье"--~Галёркина могут быть полностью решены
при аппроксимации изотропной функции распределения полиномами Сонина.
Их первое применение восходит к работам Д.~Барнетта~\cite{Burnett1935}
для вычисления транспортных коэффициентов через решение интегральных уравнений Гильберта.
И.\,А.~Эндер и А.\,Я.~Эндер первыми предложили общий численный метод на их основе~\cite{Ender1970},
позже построили полностью консервативную схему~\cite{Ender1988}.
Э.~Фонн, Ф.~Грох и Р.~Хиптмайр обобщили метод Фурье"--~Сонина для произвольной функции распределения,
однако только в двумерном случае ввиду сильно возрастающей сложности выражений~\cite{Fonn2014}.

%%% метод дискретных скоростей и разрывный Галёркин
Наконец отметим, что метод дискретных скоростей может быть формально интерпретирован
как проекционный метод в пространстве дельта-функций.
Если же их заменить на кусочно-полиномиальные функции с конечным носителем внутри некоторых ячеек скоростного пространства,
то получается известный \emph{разрывный метод Галёркина}.
Для решения уравнения Больцмана он впервые был применён Е.\,Ф.~Л\'{и}маром~\cite{Limar1973}.
А.~Майорана предложил общую методику построения консервативного разрывного метода Галёркина~\cite{Majorana2011}.

\subsection{Методы консервативной коррекции}

Макроскопическая консервативность метода в общем может быть достигнута различными процедурами коррекции,
которые, однако, способны сильно ухудшать аппроксимационную точность используемого метода.
Самая простая идея полиномиальной коррекции (умножение на многочлен) впервые была применена
Ф.\,Г.~Черемисиным и В.\,В.~Аристовым~\cite{Tcheremissine1970}.
А.\,В.~Бобылев, Н.\,Х.~Ибрагимов и С.\,В.~Рязанов использовали представление Фурье
для построения консервативного метода второго порядка для произвольного молекулярного потенциала,
оставаясь в рамках модели дискретных скоростей~\cite{Rjasanow1999, Ibragimov2002}.
Для достижения консервативности они впервые, по-видимому, применили методы условной оптимизации.
И.\,М.~Гамба и С.\,Х.~Таркабхушанам предложили минимизировать функционал коррекции
в \(L^2\)-норме с помощью множителей Лагранжа.
Э.~Габетта, Л.~Парески и Дж.~Тоскани показали, что дискретная равновесная функция распределения
должна обладать свойством~\cite{Gabetta1997}
\begin{equation}\label{eq:dicrete_equilibrium}
    \ln f_{M\gamma} \in \spann\left\{ 1, \bzeta_\gamma, \zeta^2_\gamma \right\}
\end{equation}
для того, чтобы наравне с консервативностью добиться энтропийности.

\subsection{Неравномерные сетки}

Во многих прикладных задачах эффективная аппроксимация уравнения Больцмана
требует существенно неоднородной дискретизации в скоростном пространстве.
В частности, в краевых задачах функция распределения всегда терпит разрывы на граничной поверхности,
которые к тому же распространяются вдоль характеристик в окружающий газ при обтекании выпуклых тел.
Неравномерные решётки активно используются как в методах дискретных скоростей~\cite{Kolobov2011, Morris2012},
так и проекционных~\cite{Heintz2008, Wu2014}.
Более того, в~\cite{Kolobov2011, Kolobov2013} применяются адаптивные методы построения построения сетки,
однако, несмотря на очевидные преимущества, такой подход существенно усложняет численное решение
бесстолкновительного уравнения Больцмана.
Наконец, КПИМДС на неравномерных сетках может быть построен с помощью техники \emph{многоточечного проецирования},
впервые предложенной Ф.~Варгизом~\cite{Varghese2007}.

%%%%%%%%%%%%%%%%%%%%%%%%%%%%%%%%%%%%%%%%%%%%%%%%%%%%%%%%%%%%%%%%%%%%%%%%%%%%%%%%%%%%
\section{Консервативный проекционно"=интерполяционный метод дискретных скоростей} \label{sect:method}
%%%%%%%%%%%%%%%%%%%%%%%%%%%%%%%%%%%%%%%%%%%%%%%%%%%%%%%%%%%%%%%%%%%%%%%%%%%%%%%%%%%%

Уравнение Больцмана~\eqref{eq:Boltzmann} при отсутствии внешних сил
численно решается с помощью операторного расщепления на уравнение переноса
\begin{equation}\label{eq:split_transport}
    \pder[f]{t} + \zeta_i\pder[f]{x_i} = 0,
\end{equation}
для которого используется стандартный метод конечных объёмов с явной TVD-схемой второго порядка,
и пространственно"=однородное уравнение Больцмана
\begin{equation}\label{eq:split_collisions}
    \pder[f]{t} = J(f),
\end{equation}
для которого используется КПИМДС.
Л.~Девиллет и С.~Мишлер доказали сходимость конечно-разностных схем с операторным расщеплением
к решениям Ди-Перна"--~Лионса~\cite{Desvillettes1996},
а А.\,В.~Бобылев и Т.~Овада показали, что использование симметричной схемы расщепления позволяет
достичь второго порядка аппроксимации~\cite{Bobylev2001}.

\subsection{Дискретизация скоростного пространства}

Пусть регулярная скоростная сетка \(\mathcal{V} = \Set{\bzeta_\gamma\in\mathbb{R}^d}{\gamma\in\Gamma}\)
построена таким образом, что кубатура в пространстве \(\bzeta\) выражается в виде взвешенной суммы
\begin{equation}\label{eq:bzeta_cubature}
    \int F(\bzeta) \dzeta \approx \sum_{\gamma\in\Gamma} F_\gamma w_\gamma =
        \sum_{\gamma\in\Gamma} \hat{F_\gamma}, \quad
    \sum_{\gamma\in\Gamma} w_\gamma = V_\Gamma, \quad
    F_\gamma = F(\bzeta_\gamma),
\end{equation}
где \(F(\bzeta)\) "--- произвольная интегрируемая функция,
\(V_\Gamma\) "--- полный объём скоростной сетки, \(\Gamma\) "--- некоторое множество индексов.
Тогда кубатурная формула в пространстве \((\boldsymbol{\omega},\bzeta,\bzeta_*)\)
может быть записана как
\begin{equation}\label{eq:nu_cubature}
    \int F(\boldsymbol{\omega},\bzeta,\bzeta_*) \dd\Omega(\boldsymbol{\omega})\dzeta\dzeta_* \approx
        \frac{4\pi V_\Gamma^2}{ \sum_{\nu\in\Nu} w_{\nu}w_{*\nu} }
        \sum_{\nu\in\Nu} F(\boldsymbol{\omega}_\nu,\bzeta_{\nu},\bzeta_{*\nu}) w_{\nu}w_{*\nu},
\end{equation}
где \(F(\boldsymbol{\omega},\bzeta,\bzeta_*)\) "--- также произвольная интегрируемая функция.
\(\bzeta_{\nu}\in\mathcal{V}\), \(\bzeta_{*\nu}\in\mathcal{V}\)
и \(\boldsymbol{\omega}_\nu\in S^{d-1} = \Set{\boldsymbol{\omega}\in\mathbb{R}^d}{|\boldsymbol{\omega}| = 1}\)
получаются из некоторого \((3d-1)\)-мерного кубатурного правила,
\(\Nu\subset\mathbb{N}\) "--- его множество индексов.
Заметим, что численное интегрирование в~\eqref{eq:nu_cubature} выполняется по
дискретному спектру \((\bzeta,\bzeta_*)\) и непрерывному спектру \(\boldsymbol{\omega}\).

Интеграл столкновений, записанный в симметризованной форме,
\begin{equation}\label{eq:symm_ci}
    J(f_\gamma) = \frac14\int \left(
        \delta_\gamma + \delta_{*\gamma} - \delta'_\gamma - \delta'_{*\gamma}
    \right) (f'f'_* - ff_*)B \dd\Omega(\boldsymbol{\omega}) \dzeta\dzeta_*,
\end{equation}
где \(\delta_\gamma = \delta(\bzeta-\bzeta_\gamma)\) "--- дельта"=функция Дирака в \(\mathbb{R}^d\),
имеет следующий дискретный аналог:
\begin{equation}\label{eq:discrete_symm_ci}
    \hat{J}_\gamma(\hat{f}_\gamma) =
        \frac{\pi V_\Gamma^2}{\sum_{\nu\in\Nu} w_{\nu}w_{*\nu}}
        \sum_{\nu\in\Nu} \left(
            \delta_{\nu\gamma} + \delta_{*\nu\gamma}
            - \delta'_{\nu\gamma} - \delta'_{*\nu\gamma}
        \right)\left(
            \frac{w_{\nu}w_{*\nu}}{w'_{\nu}w'_{*\nu}}
            \hat{f}'_{\nu}\hat{f}'_{*\nu} - \hat{f}_{\nu}\hat{f}_{*\nu}
        \right)B_\nu,
\end{equation}
где \(\delta_{\varsigma\gamma}\) "--- символ Кронекера.
В общем случае \(\bzeta'_{\nu}\) и \(\bzeta'_{*\nu}\) не попадают в \(\mathcal{V}\),
поэтому величины \(\hat{f}'_{\nu}\), \(\hat{f}'_{*\nu}\), \(w'_{\nu}\), \(w'_{*\nu}\)
и функции \(\delta'_{\nu\gamma}\), \(\delta'_{*\nu\gamma}\) должны быть определены некоторым образом.

Максвелловское распределение аппроксимируется следующим образом:
\begin{equation}\label{eq:discrete_Maxwellian}
    \hat{f}_{M\gamma} = \rho\left[\sum_{\varsigma\in\Gamma}w_\varsigma\exp
            \left(-\frac{(\bzeta_\varsigma - \boldsymbol{v})^2}{T}\right)
        \right]^{-1}
        w_\gamma\exp\left(-\frac{(\bzeta_\gamma - \boldsymbol{v})^2}{T}\right).
\end{equation}

\subsection{Проекционно"=интерполяционная техника}

Если скорости после столкновения,
\(\bzeta'_{\nu}\notin\mathcal{V}\) и \(\bzeta'_{*\nu}\notin\mathcal{V}\),
заменяются, соответственно, ближайшими сеточными скоростями,
\(\bzeta_{\lambda_\nu}\in\mathcal{V}\) и \(\bzeta_{\mu_\nu}\in\mathcal{V}\),
то дискретный интеграл столкновений~\eqref{eq:discrete_symm_ci} теряет свойство консервативности,
и дискретный максвеллиан~\eqref{eq:discrete_Maxwellian} перестаёт быть равновесным состоянием.
Для решения этих проблем в КПИМДС применяются две специальные процедуры.

Во-первых, \(\bzeta'_{\nu}\) проецируется на множество сеточных скоростей
\(\Set{\bzeta_{\lambda_\nu+s_a}}{a\in\Lambda}\subset\mathcal{V}\) следующим образом:
\begin{equation}\label{eq:ci_projection}
    \delta'_{\nu\gamma} = \sum_{a\in\Lambda} r_{\lambda_\nu,a}\delta_{\lambda_\nu+s_a,\gamma},
\end{equation}
где множество индексов \(\Lambda = \Set{a}{r_{\lambda_\nu,a}\neq0}\subset\mathbb{Z}\).
Множество правил смещения \(\mathcal{S} = \Set{s_a}{a\in\Lambda}\)
называется \emph{проекционным шаблоном}.
Выражение~\eqref{eq:ci_projection} можно формально рассматривать как приближение
\(\delta(\bzeta'-\bzeta_\gamma)\) в пространстве дельта"=функций
\(\Set{\delta(\bzeta-\bzeta_\gamma)}{\bzeta_\gamma\in\Nu}\)
проекционным методом Петрова"--~Галёркина на некоторую линейную оболочку функций \(\psi_s(\bzeta)\):
\begin{equation}\label{eq:Petrov-Galerkin}
    \int \psi_s(\bzeta_\gamma) \left( \delta(\bzeta'-\bzeta_\gamma)
        - \sum_{a\in\Lambda} r_{\lambda_\nu,a} \delta(\bzeta_{\lambda_\nu+s_a}-\bzeta_\gamma) \right) \dzeta_\gamma = 0.
\end{equation}
Если множество \(\{\psi_s\}\) содержит все столкновительные инварианты, например
\begin{equation}\label{eq:collision_invariants}
    \psi_0 = 1, \quad \psi_i = \zeta_i, \quad \psi_4 = \zeta_i^2,
\end{equation}
то при найденных \emph{проекционных весах} \(r_{\lambda_\nu,a}\)
для заданных \emph{проекционных скоростей} \(\bzeta_{\lambda_\nu+s_a}\)
каждый член кубатуры~\eqref{eq:discrete_symm_ci} обеспечит сохранение массы, импульса и кинетической энергии.

Во-вторых, для того чтобы выполнить
\begin{equation}\label{eq:strict_interpolation}
    \hat{J}_\gamma\left(\hat{f}_{M\gamma}\right) = 0,
\end{equation}
подбирается необходимая интерполяция \(\hat{f}'_{\nu}\).
В достаточно общем виде, она может быть рассмотрена в виде среднего взвешенного по Колмогорову
\begin{equation}\label{eq:Kolmogorov_mean}
    \hat{f}'_{\nu} = \phi_f^{-1}\left(\sum_{a\in\Lambda} q_{\lambda,a} \phi_f\left(\hat{f}_{\lambda+s_a}\right)\right), \quad
    w'_{\nu} = \phi_w^{-1}\left(\sum_{a\in\Lambda} p_{\lambda,a} \phi_w\left(w_{\lambda+s_a}\right)\right),
\end{equation}
где соответствующие \emph{интерполяционные веса} нормированы:
\begin{equation}\label{eq:normalized_pq}
    \sum_{a\in\Lambda} q_{\lambda,a} = 1, \quad
    \sum_{a\in\Lambda} p_{\lambda,a} = 1,
\end{equation}
а \(\phi_f\) и \(\phi_w\) "--- непрерывные строго монотонные функции,
\(\phi_f^{-1}\) и \(\phi_w^{-1}\) "--- обратные к ним функции.
Если положить
\begin{equation}\label{eq:geom_interpolation}
    \phi_f(x) = \phi_w(x) = \ln(x), \quad \phi_f^{-1}(x) = \phi_w^{-1}(x) = \exp(x), \quad
    p_{\lambda,a} = q_{\lambda,a} = r_{\lambda,a},
\end{equation}
то~\eqref{eq:strict_interpolation} выполняется строго.
Кроме того, несложно показать, что среднее геометрическое вида~\eqref{eq:geom_interpolation}
приводит к выполнению дискретного аналога \(\mathcal{H}\)-теоремы (энтропийности)~\cite{Dodulad2013}.
Этот тип интерполяции требует высоких затрат с вычислительной точки зрения,
однако на практике операция возведения в степень может быть выполнена с точностью \(10^{-5}\),
что позволяет в несколько раз ускорить вычисления.
Для \(\bzeta'_{*\nu}\) и \(\hat{f}'_{*\nu}\) все формулы аналогичны.

\subsection{Решение задачи Коши}

Обратимся теперь к пространственно"=однородному уравнению Больцмана~\eqref{eq:split_collisions}.
Пусть \(f_\gamma^n\) обозначает приближённое решение~\eqref{eq:split_collisions}
для скорости \(\bzeta_\gamma\), \(\gamma\in\Gamma\) в момент времени \(t_n\), \(n\in\mathbb{N}\).
Переписывая~\eqref{eq:discrete_symm_ci} как
\begin{equation}\label{eq:discrete_short_ci}
    \hat{J}_\gamma^n\left(\hat{f}_\gamma^n\right) =
        \sum_{\nu=1}^N \hat{\Delta}_\gamma^{n+(\nu-1)/N} \left(\hat{f}_\gamma^n\right), \quad
    N=|\Nu|,
\end{equation}
где \(\hat{\Delta}_\gamma^{n+(\nu-1)/N}\) "--- это \(\nu\in\Nu_n\) член суммы~\eqref{eq:discrete_symm_ci},
можно применить явный метод Эйлера первого порядка в дробных шагах
\begin{equation}\label{eq:time_scheme}
    \hat{f}_\gamma^{n+\nu/N} = \hat{f}_\gamma^{n+(\nu-1)/N} + \Delta{t} \hat{\Delta}_\gamma^{n+(\nu-1)/N}
    \left(\hat{f}_\gamma^{n+(\nu-1)/N}\right),
\end{equation}
где \(\Delta{t} = t_{n+1} - t_n\) "--- временн\'{о}й шаг.
Схема~\eqref{eq:time_scheme} имеет порядок сходимости \(\OO{\Delta{t}|\Gamma|/|\Nu|}\),
если все дискретные скорости \(\bzeta_\gamma\) распределены равномерно
в последовательностях \((\bzeta_{\nu})_{\nu=1}^N\) и \((\bzeta_{*\nu})_{\nu=1}^N\).
Этого можно добиться случайной перестановкой кубатурной последовательности.
Если \(|\Gamma|/|\Nu| = \OO{\Delta{t}}\), то достигается второй порядок точности.

%%% On the cubature rule
Оптимальные кубатурные правила Коробова~\cite{Korobov1959, Sloan1994} используются
для аппроксимации восьмимерного интеграла в~\eqref{eq:discrete_symm_ci}.
На каждом временн\'{о}м шаге решётка сдвигается на случайный вектор,
так что получается последовательность множеств кубатурных точек \((\Nu_n)_{n\in\mathbb{N}}\).

\subsection{Сохранение положительности}

Схема~\eqref{eq:time_scheme} допускает отрицательные значения функции распределения,
при которых она теряет свойство устойчивости.
Для того чтобы сохранить положительность, достаточно потребовать
\begin{equation}\label{eq:positive_f}
    \hat{f}_\gamma^{n+(\nu-1)/N} + \frac{\Delta{t}}N \hat{\Delta}_\gamma^{n+(\nu-1)/N} > 0
\end{equation}
для всех \(\gamma\in\Gamma\) и \(\nu\in\Nu_n\).
Если \(\gamma = \nu\), то имеем
\begin{equation}\label{eq:positive_f_alpha}
    \hat{f}_{\nu} - \frac{A}{N}\hat{f}_{\nu}\hat{f}_{*\nu} > 0, \quad
    A = \frac{\pi\Delta{t} V_\Gamma^2 N B_{\max}}{\sum_{\nu\in\Nu} w_{\nu}w_{*\nu}}
\end{equation}
или
\begin{equation}\label{eq:positive_f_alpha2}
    N > A \hat{f}_{\max},
\end{equation}
где
\begin{equation}\label{eq:f_B_max}
    \hat{f}_{\max} = \max_{\gamma\in\Gamma} \hat{f}_\gamma, \quad
    B_{\max} = \max_{\substack{\gamma,\varsigma\in\Gamma\\\boldsymbol{\omega}\in S^2}}
        B(\boldsymbol{\omega}, \bzeta_{\gamma}, \bzeta_{\varsigma}) = \OO{\zeta_{\max}}, \quad
    \zeta_{\max} = \max_{\gamma\in\Gamma}|\bzeta_\gamma|.
\end{equation}
Такая же оценка справедлива, когда \(\bzeta_\gamma = \bzeta_{*\nu}\).

Проекционные узлы \(\gamma = \lambda_\nu+s_a\) (и \(\gamma = \mu_\nu+s_a\))
рассматриваются с интерполяцией~\eqref{eq:geom_interpolation}.
Дополнительно предположим, что \(r_{\lambda_\nu,a} \leq 1\).
Если \(r_{\lambda_\nu,a} \geq 0\), получаем
\begin{equation}\label{eq:positive_f_lambda2+}
    N > A \hat{f}_{\max} \epsilon_f^2 \epsilon_w^2,
\end{equation}
где
\begin{equation}\label{eq:epsilon_f}
    \epsilon_f = \max_{\substack{s_a,s_b\in\mathcal{S}\\\gamma\in\Gamma}} \frac{\hat{f}_{\gamma+s_a}}{\hat{f}_{\gamma+s_b}}, \quad
    \epsilon_w = \max_{\gamma,\varsigma\in\Gamma} \frac{w_\gamma}{w_\varsigma}.
\end{equation}
Для гладкой функции распределения \(\epsilon_f\) пропорциональна максимальному \emph{диаметру} проекционного шаблона
\begin{equation}\label{eq:stencil_diameter}
    R_\mathcal{S} = \max_{\substack{s_a,s_b\in\mathcal{S}\\\gamma\in\Gamma}}
        \left| \bzeta_{\gamma+s_a} - \bzeta_{\gamma+s_b} \right|.
\end{equation}
Если \(r_{\lambda_\nu,a} < 0\), имеем
\begin{equation}\label{eq:positive_f_lambda-}
    \hat{f}_{\lambda_\nu+s_a} + \frac{A}{N}r_{\lambda_\nu,a} \hat{f}_{\nu}\hat{f}_{*\nu} > 0.
\end{equation}
Для произвольной функции распределения получаем дорогостоящую оценку
\begin{equation}\label{eq:positive_f_lambda2-}
    N > A \hat{f}_{\max} \bar{r}_{\max} \max_{\gamma,\varsigma\in\Gamma}\frac{\hat{f_\gamma}}{\hat{f_\varsigma}}, \quad
    \bar{r}_{\max} = \max_{\gamma\in\Gamma,a\in\Lambda}( -r_{\gamma,a} ),
\end{equation}
но для максвеллиана
\begin{equation}\label{eq:positive_f_lambda2-maxw}
    N > A \hat{f}_{\max} \epsilon_f^2 \bar{r}_{\max}.
\end{equation}

Таким образом, чтобы уменьшить количество точек \(N\), достаточное для~\eqref{eq:positive_f},
скоростную сетку необходимо строить, минимизируя \(|\Gamma|\), \(\zeta_{\max}\) и \(\epsilon_w\),
а проекционный шаблон выбирать, минимизируя \(R_\mathcal{S}\) и \(\bar{r}_{\max}\).
Величина \(\epsilon_f\) уменьшается при сгущении сетки в областях больших градиентов функции распределения.

%%% Excluding of the points
На практике условие~\eqref{eq:positive_f} для всех \(\nu\) требует больших вычислительных затрат.
Для достижения приемлемой точности достаточно исключать из~\eqref{eq:time_scheme} члены,
нарушающие~\eqref{eq:positive_f}. Другими словами, столкновительный интеграл можно вычислять как
\begin{equation}\label{eq:discrete_short_ci_discarded}
    \hat{J}_\gamma^n = \sum_{\nu\in\Nu\setminus\Mu} \hat{\Delta}_\gamma^{n+(\nu-1)/N},
\end{equation}
где \(\Mu\) "--- множество кубатурных точек, исключённых из \(\Nu\).
Для того чтобы не допустить значительной ошибки при такой методике численного интегрирования,
необходимо контролировать вклад исключённых узлов в столкновительный интеграл.
Например, \(N\) может быть выбрано так, чтобы величина
\begin{equation}\label{eq:excluded_contribution}
    \epsilon_J = \frac{\pi V_\Gamma^2}{\rho\sum_{\nu\in\Nu} w_{\nu}w_{*\nu}}
        \sum_{\nu\in\Mu} \left|
            \hat{f}_{\lambda_\nu}\hat{f}_{\mu_\nu} - \hat{f}_{\nu}\hat{f}_{*\nu}
        \right|B_\nu.
\end{equation}
была достаточно мала.
Интерполяция~\eqref{eq:geom_interpolation} может приводить к огромным значениям \(\hat{f}'_{\nu}\),
когда одно из значений \(\hat{f}_{\lambda_\nu+s_a}\) очень мало,
а соответствующий ему вес \(r_{\lambda_\nu,a}\) отрицателен.
По этой причине интерполяция в~\eqref{eq:excluded_contribution} не используется.

\subsection{Проекционные шаблоны}

%%% Rectangular grid
В дальнейшем будем предполагать, что скоростная сетка прямоугольна в~\(\mathbb{R}^3\),
поэтому она может быть проиндексирована целочисленным вектором, т.\,е. \(\Gamma = \Set{\gamma}{\gamma\in\mathbb{Z}^3}\).
Правило смещений может также быть представлено как целочисленный вектор, т.\,е. \(\mathcal{S}\subset\mathbb{Z}^3\).
Тогда сумму индексов следует интерпретировать как векторную сумму в~\(\mathbb{Z}^3\).
В~\cite{Anikin2012} показано, что проекционный метод обладает вторым порядком аппроксимации
по отношению к шагу прямоугольной скоростной сетки, поэтому веса \(w_\gamma\) выбираются так,
чтобы соответствовать формуле прямоугольников со срединной точкой.

%%% Review of schemes
Благодаря симметрии равномерной сетки,
достаточно использовать два проекционных узла, чтобы обеспечить консервативность.
В общем случае пять проекционных узлов необходимо,
чтобы существовало решение~\eqref{eq:Petrov-Galerkin} для~\eqref{eq:collision_invariants}.
Диаметр шаблона \(R_\mathcal{S}\) можно уменьшить, если использовать семь проекционных узлов.
Если \(|\mathcal{S}|=n\), то схема~\eqref{eq:time_scheme} называется \(n\)-\emph{точечной схемой}.

%%% 2-point scheme
\emph{2-точечная схема} основана на симметричном проецировании
\begin{equation}\label{eq:uniform_projection}
    \delta'_{\nu\gamma} = (1-r)\delta_{\lambda\gamma} + r\delta_{\lambda+s,\gamma}, \quad
    \delta'_{*\nu\gamma} = (1-r)\delta_{\mu\gamma} + r\delta_{\mu-s,\gamma},
\end{equation}
где \(\bzeta_{\lambda+s} + \bzeta_{\mu-s} = \bzeta_{\lambda} + \bzeta_{\mu}\) и
\begin{equation}\label{eq:stencil_weights2}
    r = \frac{E_0-E_1}{E_2-E_1}, \quad
    E_0 = \bzeta_{\nu}^2 + \bzeta_{*\nu}^2, \quad
    E_1 = \bzeta_{\lambda}^2 + \bzeta_{\mu}^2, \quad
    E_2 = \bzeta_{\lambda+s}^2 + \bzeta_{\mu-s}^2.
\end{equation}
Подстрочный индекс \(\nu\) опущен для краткости.
Для этой схемы выполняются следующие соотношения:
\begin{equation}\label{eq:weights_ranges2}
    0 \leq r < 1, \quad h \leq R_{\mathcal{S}} \leq \sqrt3h,
\end{equation}
где \(h^3 = w_\gamma = V_\Gamma/|\Gamma|\).

%%% 5-point scheme
Пусть \(\boldsymbol{\eta} = \bzeta'_{\nu} - \bzeta_{\lambda}\),
а \(\bh_+\), \(\bh_-\) "--- минимальные диагональные смещения от \(\bzeta_{\lambda}\),
такие что вектор \(\bh_+\) направлен в тот же октант, что и \(\boldsymbol{\eta}\),
а вектор \(\bh_-\) лежит в противоположном.
Тогда \emph{компактная 5-точечная схема} строится на узлах
\begin{equation}\label{eq:stencil_nodes5}
    \bzeta_{\lambda+s_0} = \bzeta_{\lambda}, \quad
    \bzeta_{\lambda+s_i} = \bzeta_{\lambda} + (\bh_+\cdot \be_i)\be_i, \quad
    \bzeta_{\lambda+s_4} = \bzeta_{\lambda} + \bh_-,
\end{equation}
где \(\be_i\) "--- базис прямоугольной скоростной сетки. Проекционные веса равны
\begin{equation}\label{eq:stencil_weights5}
    r_{\lambda,0} = 1 - \sum_{i=1}^4 r_{\lambda,i}, \quad
    r_{\lambda,i} = \frac{\eta_i - r_{\lambda,4}h_{-i}}{h_{+i}}, \quad
    r_{\lambda,4} = \frac{\boldsymbol{\eta}\cdot(\boldsymbol{\eta} - \bh_+)}
        {\bh_-\cdot(\bh_- - \bh_+)}.
\end{equation}
Для равномерной сетки справедливы следующие соотношения:
\begin{equation}\label{eq:weights_ranges5}
    0 < r_{\lambda,0} \leq 1, \quad
    -\frac1{12} \leq r_{\lambda,i} < \frac{11}{24}, \quad
    -\frac18 \leq r_{\lambda,4} \leq 0, \quad
    R_\mathcal{S} = \sqrt6h.
\end{equation}

%%% 7-point scheme
\emph{Симметричная 7-точечная схема} строится на узлах
\begin{equation}\label{eq:stencil_nodes7}
    \bzeta_{\lambda+s_0} = \bzeta_{\lambda}, \quad
    \bzeta_{\lambda+s_{\pm i}} = \bzeta_{\lambda} + (\bh_\pm\cdot \be_i)\be_i.
\end{equation}
Проекционные веса равны
\begin{equation}\label{eq:stencil_weights7}
    r_{\lambda,0} = 1 - \sum_{i=1}^3 r_{\lambda,i} + r_{\lambda,-i}, \quad
    r_{\lambda,\pm i} = \pm\frac{\eta_i(\eta_i - h_{\mp i})}{h_{\pm i}(h_{+i}-h_{-i})}.
\end{equation}
В~\eqref{eq:stencil_weights7} суммирование по повторяющимся индексам не производится.
Для равномерной сетки справедливы следующие соотношения:
\begin{equation}\label{eq:weights_ranges7}
    \frac14 \leq r_{\lambda,0} \le  q 1, \quad
    0 \leq r_{\lambda,\pm i} \leq \frac38, \quad
    -\frac18 \leq r_{\lambda,\mp i} \leq 0, \quad
    R_\mathcal{S} = 2h.
\end{equation}

И 5-точечная, и 7-точечная схемы обладают \(\bar{r}_{\max}=1/8\).
Для того чтобы уменьшить это значение, необходимо использовать больше проекционных узлов~\cite{Dodulad2012}.

%\newpage
%============================================================================================================================

\clearpage
